\tikzset{
    circuits/pcb/width/.initial = 0.7cm,
    circuits/pcb/height/.initial = 0.6cm,
    circuits/pcb/chip size/.initial = 0.15cm,
    circuits/pcb/chip pins/.initial = 4,
    circuits/pcb/chip pin length/.initial = 1pt,
    circuits/pcb/out pins/.initial = 3,
    circuits/pcb/pin width/.initial = 2pt,
    circuits/pcb/pin height/.initial = 2pt,
    pcb/.style = {pcbshape, draw, circuits/general}
}

\pgfdeclareshape{pcbshape}
{
    \saveddimen{\width}{\pgf@x=\pgfkeysvalueof{/tikz/circuits/pcb/width}}
    \saveddimen{\height}{\pgf@x=\pgfkeysvalueof{/tikz/circuits/pcb/height}}
    \saveddimen{\chipsize}{\pgf@x=\pgfkeysvalueof{/tikz/circuits/pcb/chip size}}
    \savedmacro{\chippins}{\renewcommand{\chippins}[0]{\pgfkeysvalueof{/tikz/circuits/pcb/chip pins}}}
    \saveddimen{\chippinlength}{\pgf@x=\pgfkeysvalueof{/tikz/circuits/pcb/chip pin length}}
    \saveddimen{\chippinpitch}{
        \pgfmathsetlength\pgf@x{\pgfkeysvalueof{/tikz/circuits/pcb/chip size} / (\pgfkeysvalueof{/tikz/circuits/pcb/chip pins} + 1)}
    }
    \savedmacro{\pins}{\renewcommand{\pins}[0]{\pgfkeysvalueof{/tikz/circuits/pcb/out pins}}}
    \saveddimen{\pinpitch}{
        \pgfmathsetlength\pgf@x{\pgfkeysvalueof{/tikz/circuits/pcb/height} / (\pgfkeysvalueof{/tikz/circuits/pcb/out pins} + 2)}
    }
    \saveddimen{\pinwidth}{\pgf@x=\pgfkeysvalueof{/tikz/circuits/pcb/pin width}}
    \saveddimen{\pinheight}{\pgf@x=\pgfkeysvalueof{/tikz/circuits/pcb/pin height}}
    \savedanchor{\centerpoint}{\pgfpointorigin}
    % electrical anchors
    % regular anchors
    \anchor{text}
    {
        \pgfpoint{-0.5\wd\pgfnodeparttextbox}{-0.5\ht\pgfnodeparttextbox}
    }
    \anchor{center}{\centerpoint}
    \anchor{north}{\pgfpointadd{\pgfpointorigin}{\pgfpoint{0.0 * \width}{ 0.5 * \height}}}
    \anchor{south}{\pgfpointadd{\pgfpointorigin}{\pgfpoint{0.0 * \width}{-0.5 * \height}}}
    \anchor{west}{\pgfpointadd{\pgfpointorigin}{\pgfpoint{-0.5 * \width}{ 0.0 * \height}}}
    \anchor{east}{\pgfpointadd{\pgfpointorigin}{\pgfpoint{ 0.5 * \width}{ 0.0 * \height}}}
    \anchor{north east}{\pgfpointadd{\pgfpointorigin}{\pgfpoint{ 0.5 * \width}{ 0.5 * \height}}}
    \anchor{south east}{\pgfpointadd{\pgfpointorigin}{\pgfpoint{ 0.5 * \width}{-0.5 * \height}}}
    \anchor{north west}{\pgfpointadd{\pgfpointorigin}{\pgfpoint{-0.5 * \width}{ 0.5 * \height}}}
    \anchor{south west}{\pgfpointadd{\pgfpointorigin}{\pgfpoint{-0.5 * \width}{-0.5 * \height}}}
    \anchorborder{
        \@tempdima=\pgf@x
        \@tempdimb=\pgf@y
        \pgfpointborderrectangle{\pgfpoint{\@tempdima}{\@tempdimb}}{\pgfpointadd{\pgfpointorigin}{\pgfpoint{ 0.5 * \width}{ 0.5 * \height}}}
    }
    \beforebackgroundpath{
        \pgfsetcornersarced{\pgfpoint{1pt}{1pt}}
        % draw pcb body
        \pgfpathrectanglecorners
            {\pgfpointadd{\pgfpointorigin}{\pgfpoint{-0.5 * \width}{-0.5 * \height}}}
            {\pgfpointadd{\pgfpointorigin}{\pgfpoint{ 0.5 * \width}{ 0.5 * \height}}}
        \pgfusepath{stroke}
        % draw chip
        \pgfsetcornersarced{\pgfpointorigin}
        \pgfpathrectangle
            {\pgfpointadd{\pgfpointorigin}{\pgfpoint{-0.5 * \chipsize}{-0.5 * \chipsize}}}
            {\pgfpointadd{\pgfpointorigin}{\pgfpoint{\chipsize}{\chipsize}}}
        \pgfusepath{fill}
        % draw chip pins
        \foreach \i in {1, 2, ..., \chippins}
        {
            \pgfsetlinewidth{0.3pt}
            \pgfpathmoveto{\pgfpointadd{\pgfpointorigin}{\pgfpoint{(\i - 0.5 * (\chippins + 1)) * \chippinpitch}{0.5 * \chipsize}}}
            \pgfpathlineto{\pgfpointadd{\pgfpointorigin}{\pgfpoint{(\i - 0.5 * (\chippins + 1)) * \chippinpitch}{0.5 * \chipsize + \chippinlength}}}
            \pgfusepath{stroke}
            \pgfpathmoveto{\pgfpointadd{\pgfpointorigin}{\pgfpoint{(\i - 0.5 * (\chippins + 1)) * \chippinpitch}{-0.5 * \chipsize}}}
            \pgfpathlineto{\pgfpointadd{\pgfpointorigin}{\pgfpoint{(\i - 0.5 * (\chippins + 1)) * \chippinpitch}{-0.5 * \chipsize - \chippinlength}}}
            \pgfusepath{stroke}
            \pgfpathmoveto{\pgfpointadd{\pgfpointorigin}{\pgfpoint{0.5 * \chipsize}{(\i - 0.5 * (\chippins + 1)) * \chippinpitch}}}
            \pgfpathlineto{\pgfpointadd{\pgfpointorigin}{\pgfpoint{0.5 * \chipsize + \chippinlength}{(\i - 0.5 * (\chippins + 1)) * \chippinpitch}}}
            \pgfusepath{stroke}
            \pgfpathmoveto{\pgfpointadd{\pgfpointorigin}{\pgfpoint{-0.5 * \chipsize}{(\i - 0.5 * (\chippins + 1)) * \chippinpitch}}}
            \pgfpathlineto{\pgfpointadd{\pgfpointorigin}{\pgfpoint{-0.5 * \chipsize - \chippinlength}{(\i - 0.5 * (\chippins + 1)) * \chippinpitch}}}
            \pgfusepath{stroke}
        }
        % draw mounting holes
        \foreach \x in {-1, 1}
        {
            \foreach \y in {-1, 1}
            {
                \pgfpathcircle{\pgfpointadd{\pgfpointorigin}{\pgfpoint{\x * 0.4 * \width}{\y * 0.4 * \height}}}{0.6pt}
                \pgfusepath{fill}
            }
        }
        \foreach \y in {1, 2, ..., \pins}
        {
            \pgfpathrectangle
                {\pgfpointadd{\pgfpointorigin}{\pgfpoint{-0.5 * \width}{(\y - 0.5 * (\pins + 1)) * \pinpitch - 0.5 * \pinheight}}}
                {\pgfpointadd{\pgfpointorigin}{\pgfpoint{\pinwidth}{\pinheight}}}
            \pgfusepath{fill}
        }
    }
    % \pgf@sh@s@pcbshape contains all the code for the pcb shape
    % and is executed just before a pcb node is drawn.
    \pgfutil@g@addto@macro\pgf@sh@s@pcbshape{%
        % Start with the maximum pin number and go backwards.
        % If the anchor is undefined, create it. Otherwise stop.
        \c@pgf@counta=\pgfkeysvalueof{/tikz/circuits/pcb/out pins}\relax%
        \pgfmathloop%
        \ifnum\c@pgf@counta>0\relax%
            \pgfutil@ifundefined{pgf@anchor@pcbshape@out\space\the\c@pgf@counta}{%
                \expandafter\xdef\csname pgf@anchor@pcbshape@out\space\the\c@pgf@counta\endcsname{%
                    \noexpand\pcbpinanchor{\the\c@pgf@counta}%
                }%
            }{\c@pgf@counta=0\relax}%
            \advance\c@pgf@counta-1\relax%
        \repeatpgfmathloop%  
    }%
}
\def\pcbpinanchor#1{%
    \pgfpointadd{\pgfpointorigin}{\pgfpoint{-0.5 * \width}{(#1 - 0.5 * (\pins + 1)) * \pinpitch}}
}

% vim: ft=plaintex
