\tikzset{
    inverterchain/.style = {inverterchainshape}
    circuits/inverterchain/segments/.initial = 3,
}

\pgfdeclareshape{inverterchainshape}
{
    \saveddimen{\width}{\pgfpointscale{\pgfkeysvalueof{/tikz/circuits/scale}}{\pgfpoint{\pgfkeysvalueof{/tikz/circuits/logic gate/width}}{0pt}}}
    \saveddimen{\height}{\pgfpointscale{\pgfkeysvalueof{/tikz/circuits/scale}}{\pgfpoint{\pgfkeysvalueof{/tikz/circuits/logic gate/height}}{0pt}}}
    \saveddimen{\dotradius}{\pgfpointscale{\pgfkeysvalueof{/tikz/circuits/scale}}{\pgfpoint{\pgfkeysvalueof{/tikz/circuits/logic gate/dot radius}}{0pt}}}
    \savedmacro{\numinv}{\renewcommand{\numinv}[0]{\pgfkeysvalueof{/tikz/circuits/inverterchain/numinv}}}
    \savedanchor{\einput}{%
        \pgfpointadd%
        {\pgfpointorigin}%
        {\pgfpoint%
            {-0.5 * \pgfkeysvalueof{/tikz/circuits/inverter/width}}%
            {0cm}
        }%
    }
    \savedanchor{\output}{%
        \pgfpointadd%
        {\pgfpointorigin}%
        {\pgfpoint%
            {0.5 * \pgfkeysvalueof{/tikz/circuits/inverter/width} + \pgfkeysvalueof{/tikz/circuits/inverter/radius}}%
            {0cm}
        }%
    }
    % electrical terminals (anchors)
    \anchor{in}{\pgfpoint{-0.5 * \width}{0pt}}
    \anchor{out}{\pgfpoint{0.5 * \width + 2 * \dotradius}{0pt}}
    \anchor{vdd}{\pgfpoint{0pt}{0.25 * \height}}
    \anchor{vss}{\pgfpoint{0pt}{-0.25 * \height}}
    % regular anchors
    \anchor{center}{\pgfpointorigin}
    \anchor{north}{\pgfpoint{0pt}{0.5 * \height}}
    \anchor{south}{\pgfpoint{0pt}{-0.5 * \height}}
    \anchor{west}{\pgfpoint{-0.5 * \width}{0pt}}
    \anchor{east}{\pgfpoint{0.5 * \width}{0pt}}
    \anchor{north east}{\pgfpoint{0.5 * \width}{0.5 * \height}}
    \anchor{north west}{\pgfpoint{0.5 * \width}{-0.5 * \height}}
    \anchor{south east}{\pgfpoint{-0.5 * \width}{0.5 * \height}}
    \anchor{south west}{\pgfpoint{-0.5 * \width}{-0.5 * \height}}
    \anchor{center}{\pgfpointorigin}
    \beforebackgroundpath{
        \pgfsetlinewidth{\pgfkeysvalueof{/tikz/circuits/line width}}
        \pgfsetmiterjoin
        \pgfsetmiterlimit{2}
        \foreach \x in {1,...,\segments}
        {
            \pgfpointlineattime{0.1}{\pgfpoint{-0.5 * \width}{-0.5 * \height}}{\pgfpoint{ 0.5 * \width}{0pt}}
            \pgfpathmoveto{\pgfpoint{-0.5 * \width}{-0.5 * \height}}
            \pgfpathlineto{\pgfpoint{-0.5 * \width}{ 0.5 * \height}}
            \pgfpointlineattime{0.1}{\pgfpoint{-0.5 * \width}{ 0.5 * \height}}{\pgfpoint{ 0.5 * \width}{0pt}}
            \pgfpathclose
            \color{white}
            \pgfsetstrokecolor{black}
            \pgfusepath{stroke, fill}
        }

        \pgfpathmoveto{\pgfpoint{-0.5 * \width}{-0.5 * \height}}
        \pgfpathlineto{\pgfpoint{-0.5 * \width}{ 0.5 * \height}}
        \pgfpathlineto{\pgfpoint{ 0.5 * \width}{0pt}}
        \pgfpathclose
        \color{white}
        \pgfsetstrokecolor{black}
        \pgfusepath{stroke, fill}
        % draw output circle
        \pgfpathcircle{\pgfpoint{0.5 * \width + \dotradius}{0pt}}{\dotradius}
        \color{white}
        \pgfsetstrokecolor{black}
        \pgfusepath{stroke, fill}
    }
}

% vim: ft=plaintex nowrap
