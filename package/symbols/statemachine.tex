\tikzset{
    circuits/statemachine/width/.initial = 1.3cm,
    circuits/statemachine/trianglesize/.initial = 3pt,
    circuits/statemachine/ports/.initial = 1,
    circuits/statemachine/port pitch/.initial = 0.30cm,
    statemachine/.style = {statemachineshape}
}

\pgfdeclareshape{statemachineshape}
{
    \saveddimen{\width}{\pgf@x=\pgfkeysvalueof{/tikz/circuits/statemachine/width}}
    \saveddimen{\trianglesize}{\pgf@x=\pgfkeysvalueof{/tikz/circuits/statemachine/trianglesize}}
    \saveddimen\height{
        \pgfmathsetlength\pgf@x{(\pgfkeysvalueof{/tikz/circuits/statemachine/ports}) * \pgfkeysvalueof{/tikz/circuits/statemachine/port pitch}}%
    }
    \savedmacro{\ports}{\renewcommand{\ports}[0]{\pgfkeysvalueof{/tikz/circuits/statemachine/ports}}}
    \saveddimen{\portpitch}{\pgf@x=\pgfkeysvalueof{/tikz/circuits/statemachine/port pitch}}
    \saveddimen{\dotradius}{\pgf@x=2pt}
    % electrical anchors
    \anchor{clk}{\pgfpoint{-0.5 * \width}{0pt + 1 * \portpitch}}
    \anchor{reset}{\pgfpoint{-0.5 * \width - 2 * \dotradius}{0pt + 0 * \portpitch}}
    % regular anchors
    \anchor{text}
    {
        \pgfpoint{-0.5\wd\pgfnodeparttextbox}{-0.5\ht\pgfnodeparttextbox}
    }
    \anchor{center}{\pgfpointorigin}
    \anchor{north}{\pgfpointadd{\pgfpointorigin}{\pgfpoint{0.0 * \width}{ 0.5 * \height}}}
    \anchor{south}{\pgfpointadd{\pgfpointorigin}{\pgfpoint{0.0 * \width}{-0.5 * \height}}}
    \anchor{west}{\pgfpointadd{\pgfpointorigin}{\pgfpoint{-0.5 * \width}{ 0.0 * \height}}}
    \anchor{east}{\pgfpointadd{\pgfpointorigin}{\pgfpoint{ 0.5 * \width}{ 0.0 * \height}}}
    \anchor{north east}{\pgfpointadd{\pgfpointorigin}{\pgfpoint{ 0.5 * \width}{ 0.5 * \height}}}
    \anchor{south east}{\pgfpointadd{\pgfpointorigin}{\pgfpoint{ 0.5 * \width}{-0.5 * \height}}}
    \anchor{north west}{\pgfpointadd{\pgfpointorigin}{\pgfpoint{-0.5 * \width}{ 0.5 * \height}}}
    \anchor{south west}{\pgfpointadd{\pgfpointorigin}{\pgfpoint{-0.5 * \width}{-0.5 * \height}}}
    \beforebackgroundpath{%
        % draw body
        \pgfsetlinewidth{\pgfkeysvalueof{/tikz/circuits/line width}}
        \pgfpathrectangle %
            {\pgfpointadd{\pgfpointorigin}{\pgfpoint{-0.5 * \width}{-0.5 * \height}}}
            {\pgfpointadd{\pgfpointorigin}{\pgfpoint{\width}{\height}}}
        \pgfusepath{stroke}
        % draw clk triangle
        \pgfpathmoveto{%
            \pgfpointadd%
                {\pgfpoint{-0.5 * \width}{0pt + 1 * \portpitch}}%
                {\pgfpoint{0pt}{\trianglesize}}%
        }
        \pgfpathlineto{%
            \pgfpointadd%
                {\pgfpoint{-0.5 * \width}{0pt + 1 * \portpitch}}%
                {\pgfpoint{\trianglesize}{0pt}}%
        }
        \pgfpathlineto{%
            \pgfpointadd%
                {\pgfpoint{-0.5 * \width}{0pt + 1 * \portpitch}}%
                {\pgfpoint{0pt}{-\trianglesize}}%
        }
        \pgfusepath{stroke}
        % draw reset circle
        \pgfpathcircle{\pgfpoint{-0.5 * \width - \dotradius}{0pt}}{\dotradius}
        \pgfusepath{stroke}
    }
    % \pgf@sh@s@statemachine contains all the code for the statemachine shape
    % and is executed just before a statemachine node is drawn.
    \pgfutil@g@addto@macro\pgf@sh@s@statemachineshape{%
        % Start with the maximum pin number and go backwards.
        % If the anchor is undefined, create it. Otherwise stop.
        \c@pgf@counta=\pgfkeysvalueof{/tikz/circuits/statemachine/ports}\relax%
        \pgfmathloop%
        \ifnum\c@pgf@counta>0\relax%
            \pgfutil@ifundefined{pgf@anchor@statemachineshape@port\space\the\c@pgf@counta}{%
                \expandafter\xdef\csname pgf@anchor@statemachineshape@port\space\the\c@pgf@counta\endcsname{%
                    \noexpand\statemachineportanchor{\the\c@pgf@counta}%
                }%
            }{\c@pgf@counta=0\relax}%
            \advance\c@pgf@counta-1\relax%
        \repeatpgfmathloop%  
    }%
}

\def\statemachineportanchor#1{%
    % When this macro is called,
    % \width, \ports and \portpitch will be defined
    \pgfpointadd{\pgfpointorigin}{\pgfpoint{-0.5 * \width}{(#1 - 0.5 * (\ports + 1) - 1) * \portpitch}}
}

% vim: ft=plaintex
