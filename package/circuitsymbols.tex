\newif\ifcircuitsflip

\makeatletter

% drawing styles
\tikzset{
    circuits/general/.style = {thick},
    wire/.style={circuits/general},
    cable/.style={circuits/general, cap = round},
    %rfcable/.style={circuits/general, double, double distance = 0.6pt, cap = round},
    rfcable/.style={circuits/general, double, cap = round},
    arrowwire/.style = {
        circuits/general,
        ->,
        >=stealth,
    },
    lwire/.style={circuits/general, line cap = rect},
    scriptsize/.style={font=\scriptsize},
    ---/.style args={#1 and #2}{to path={-- ++(#1, 0) |- ($(\tikztotarget)+(#2, 0)$)}},
    ---/.default = 0cm and 0cm,
    -|-/.style args={#1 and #2}{to path={-| ([xshift=#1, yshift=#2]$(\tikztostart)!0.5!(\tikztotarget)$) |- (\tikztotarget) \tikztonodes}},
    -|-/.default = 0cm and 0cm,
    |-|/.style args={#1 and #2}{to path={|- ([xshift=#1, yshift=#2]$(\tikztostart)!0.5!(\tikztotarget)$) -| (\tikztotarget) \tikztonodes}},
    |-|/.default = 0cm and 0cm,
    angleconnect/.style args={#1 and #2}{to path={-- ++(#1, 0) -- ($(\tikztotarget)-(#2, 0)$) -- (\tikztotarget)}},
    angleconnect/.value required = true,
    angleconnectx/.style args={#1 and #2}{to path={-- ++(#1, 0) -- ($(\tikztotarget)-(#2, 0)$) -- (\tikztotarget)}},
    angleconnectx/.value required = true,
    angleconnecty/.style args={#1 and #2}{to path={-- ++(0, #1) -- ($(\tikztotarget)-(0, #2)$) -- (\tikztotarget)}},
    angleconnecty/.value required = true,
    tmodule/.style args ={#1 and #2}{draw, rectangle, wire, align = center, minimum width = #1, minimum height = #2},
    tmodule/.default = {0cm and 0cm},
    module width/.store in=\circuits@module@width,
    module height/.store in=\circuits@module@height,
    module width = 1cm,
    module height = 1cm,
    module/.style = {
        wire,
        draw, rectangle,
        align = center,
        minimum width = \circuits@module@width,
        minimum height = \circuits@module@height
    },
    current arrow/.style = {
        postaction = decorate,
        decoration = {
            markings,
            mark = at position #1 with {\arrow{stealth}}
        }
    },
    current arrow/.default = 0.5,
    current arrow reversed/.style = {
        postaction = decorate,
        decoration = {
            markings,
            mark = at position #1 with {\arrowreversed{stealth}}
        }
    },
    current arrow reversed/.default = 0.5,
    bit line/.style = {
        wire,
        postaction = decorate,
        decoration = {
            markings,
            mark = at position 0.5 with { \draw (-#1, -#1) -- (#1, #1); }
        }
    },
    bit line/.default = 2pt,
    % flip components (effect depends on shape)
    flip/.is if = circuitsflip,
    xmirror/.style = {xscale = -1},
    ymirror/.style = {yscale = -1}
}

\pgfkeys{
    % general circuits options
    /tikz/circuits/line width/.initial = 0.8pt,
    /tikz/circuits/dot/radius/.initial = 0.05cm,
    /tikz/dot/.style = {dotshape},
    /tikz/circuits/port/radius/.initial = 0.05cm,
    %/tikz/port/.style = {portshape}
    /tikz/port/.style = {wire, draw=black, fill=white, circle, inner sep = 0cm, minimum size = 0.1cm}
}

\pgfdeclareshape{dotshape}
{
    \saveddimen{\radius}{\pgf@x=\pgfkeysvalueof{/tikz/circuits/dot/radius}}
    \anchor{center}{\pgfpointorigin}
    \anchor{north}{\pgfpointorigin}
    \anchor{south}{\pgfpointorigin}
    \anchor{west}{\pgfpointorigin}
    \anchor{east}{\pgfpointorigin}
    \anchor{north east}{\pgfpointorigin}
    \anchor{north west}{\pgfpointorigin}
    \anchor{south east}{\pgfpointorigin}
    \anchor{south west}{\pgfpointorigin}
    \backgroundpath{
        \pgfpathcircle{\pgfpointorigin}{\radius}
        \pgfusepath{fill}
    }
}
\pgfdeclareshape{portshape}
{
    \saveddimen{\radius}{\pgf@x=\pgfkeysvalueof{/tikz/circuits/port/radius}}
    \anchor{center}{\pgfpointorigin}
    \backgroundpath{
        \color{white}
        \pgfsetstrokecolor{black}
        \pgfsetlinewidth{\pgfkeysvalueof{/tikz/circuits/line width}}
        \pgfpathcircle{\pgfpointorigin}{\radius}
        \pgfusepath{fill, stroke}
    }
}

\def\circuitskeysvalueof#1{\pgfkeysvalueof{/tikz/circuits/.cd, #1}}

\pgfkeys{
    /tikz/circuits/scale/.initial = 1
}

\newif\ifotaflipinputs
\newif\ifotaflipoutputs
\newif\ifotafullydifferential
\newif\ifotanopinlabels

\tikzset{
    circuits/ota/scale/.initial = 1,
    circuits/ota/width/.initial = 2cm,
    circuits/ota/input height/.initial = 2cm,
    circuits/ota/output height/.initial = 1cm,
    circuits/ota/input distribution factor/.initial = 0.6,
    circuits/ota/output distribution factor/.initial = 0.8,
    circuits/ota/pin fontsize/.initial = {\small},
    circuits/ota/flip inputs/.is if=otaflipinputs,
    circuits/ota/flip outputs/.is if=otaflipoutputs,
    circuits/ota/fully differential/.is if=otafullydifferential,
    circuits/ota/no pin labels/.is if=otanopinlabels,
    ota/.style = {fdotashape, circuits/ota/fully differential=false},
    fdota/.style = {fdotashape, circuits/ota/fully differential=true}
}

\pgfdeclareshape{fdotashape}
{
    \saveddimen{\width}{\pgf@x=\pgfkeysvalueof{/tikz/circuits/ota/width}}
    \saveddimen{\inputheight}{\pgf@x=\pgfkeysvalueof{/tikz/circuits/ota/input height}}
    \saveddimen{\outputheight}{\pgf@x=\pgfkeysvalueof{/tikz/circuits/ota/output height}}
    \savedmacro{\inputdistribution}{\renewcommand{\inputdistribution}[0]{\pgfkeysvalueof{/tikz/circuits/ota/input distribution factor}}}
    \savedmacro{\outputdistribution}{\renewcommand{\outputdistribution}[0]{\pgfkeysvalueof{/tikz/circuits/ota/output distribution factor}}}
    \savedmacro{\scale}{\renewcommand{\scale}[0]{\pgfkeysvalueof{/tikz/circuits/ota/scale}}}
    %\savedmacro{\pinfontsize}{\renewcommand{\pinfontsize}[0]{\pgfkeysvalueof{/tikz/circuits/ota/pin fontsize}}}
    \savedanchor{\centerpoint}{\pgfpointorigin}
    \savedanchor{\inputplus}{%
        \pgfpointadd%
        {\pgfpointorigin}%
        {\pgfpoint%
            {-0.5 * \pgfkeysvalueof{/tikz/circuits/ota/width}}%
            {0.5 * \pgfkeysvalueof{/tikz/circuits/ota/input height} * \pgfkeysvalueof{/tikz/circuits/ota/input distribution factor}}%
        }%
    }
    \savedanchor{\inputminus}{%
        \pgfpointadd%
        {\pgfpointorigin}%
        {\pgfpoint%
            {-0.5 * \pgfkeysvalueof{/tikz/circuits/ota/width}}%
            {-0.5 * \pgfkeysvalueof{/tikz/circuits/ota/input height} * \pgfkeysvalueof{/tikz/circuits/ota/input distribution factor}}%
        }%
    }
    \savedanchor{\outputminus}{%
        \pgfpointadd%
        {\pgfpointorigin}%
        {\pgfpoint%
            {0.5 * \pgfkeysvalueof{/tikz/circuits/ota/width}}%
            {0.5 * \pgfkeysvalueof{/tikz/circuits/ota/output height}}
        }%
    }
    \savedanchor{\outputplus}{%
        \pgfpointadd%
        {\pgfpointorigin}%
        {\pgfpoint%
            {0.5 * \pgfkeysvalueof{/tikz/circuits/ota/width}}%
            {-0.5 * \pgfkeysvalueof{/tikz/circuits/ota/output height}}
        }%
    }
    % electrical terminals (anchors)
    \anchor{cmin}{\pgfpointadd{\centerpoint}{\pgfpoint{-\width/2}{0cm}}}
    \anchor{cmout}{\pgfpointadd{\centerpoint}{\pgfpoint{\width/2}{0cm}}}
    \anchor{+in}{\inputplus}
    \anchor{-in}{\inputminus}
    \anchor{in}{\pgfpointadd{\centerpoint}{\pgfpoint{-\width/2}{0cm}}}
    \anchor{out}{
        \pgfpointadd%
        {\pgfpointorigin}%
        {\pgfpoint%
            {0.5 * \width}%
            {0cm}
        }%
    }
    \anchor{+out}{
        \pgfpointlineattime%
        {\outputdistribution}%
        {\pgfpointadd{\inputminus}{\pgfpoint{0cm}{-0.5 * \inputheight * (1 - \inputdistribution)}}}%
        {\outputplus}%
    }
    \anchor{-out}{
        \pgfpointlineattime%
        {\outputdistribution}%
        {\pgfpointadd{\inputplus}{\pgfpoint{0cm}{0.5 * \inputheight * (1 - \inputdistribution)}}}%
        {\outputminus}%
    }
    \anchor{+power}{
        \pgfpointlineattime%
        {0.5}%
        {\pgfpointadd{\inputplus}{\pgfpoint{0cm}{0.5 * \inputheight * (1 - \inputdistribution)}}}%
        {\outputminus}
    }
    \anchor{-power}{
        \pgfpointlineattime%
        {0.5}%
        {\pgfpointadd{\inputminus}{\pgfpoint{0cm}{-0.5 * \inputheight * (1 - \inputdistribution)}}}%
        {\outputplus}
    }
    % regular anchors
    \anchor{center}{\centerpoint}
    \anchor{text} % this is used to center the text in the node
    {
        %\pgfpoint{-.5\wd\pgfnodeparttextbox}{-.5\ht\pgfnodeparttextbox}
        % the text node is shifted more than its width, a way of optical compensation (since the amplifier gets more narrow to the right)
        \pgfpoint{-0.5\wd\pgfnodeparttextbox}{-0.5\ht\pgfnodeparttextbox}
    }
    \anchor{north}{
        \pgfpointadd{\inputplus}{\pgfpoint{\width/2}{0.5 * \inputheight * (1 - \inputdistribution)}}
    }
    \anchor{south}{
        \pgfpointadd{\inputminus}{\pgfpoint{\width/2}{-0.5 * \inputheight * (1 - \inputdistribution)}}
    }
    \anchor{west}{
        \pgfpointadd{\centerpoint}{\pgfpoint{-0.5 * \width}{0pt}}
    }
    \anchor{east}{
        \pgfpointadd{\centerpoint}{\pgfpoint{0.5 * \width}{0pt}}
    }
    \anchor{south west}{
        \pgfpointadd{\inputminus}{\pgfpoint{0cm}{-0.5 * \inputheight * (1 - \inputdistribution)}}
    }
    \anchor{north east}{
        \pgfpointadd{\inputplus}{\pgfpoint{\width}{0.5 * \inputheight * (1 - \inputdistribution)}}
    }
    \anchor{south east}{
        \pgfpointadd{\inputminus}{\pgfpoint{\width}{-0.5 * \inputheight * (1 - \inputdistribution)}}
    }
    \beforebackgroundpath{
        \pgfsetlinewidth{\pgfkeysvalueof{/tikz/circuits/line width}}
        \pgfpathmoveto{\pgfpointadd{\inputplus}{\pgfpoint{0cm}{0.5 * \inputheight * (1 - \inputdistribution)}}}
        \pgfpathlineto{\outputminus}
        \pgfpathlineto{\outputplus}
        \pgfpathlineto{\pgfpointadd{\inputminus}{\pgfpoint{0cm}{-0.5 * \inputheight * (1 - \inputdistribution)}}}
        \pgfpathclose
        \pgfusepath{stroke}
        \ifotanopinlabels
        \else
            \ifotaflipinputs
                \pgftext[left, at={\inputplus}, x=2pt]{\pgfkeysvalueof{/tikz/circuits/ota/pin fontsize}$-$}
                \pgftext[left, at={\inputminus}, x=2pt]{\pgfkeysvalueof{/tikz/circuits/ota/pin fontsize}$+$}
            \else
                \pgftext[left, at={\inputplus}, x=2pt]{\pgfkeysvalueof{/tikz/circuits/ota/pin fontsize}$+$}
                \pgftext[left, at={\inputminus}, x=2pt]{\pgfkeysvalueof{/tikz/circuits/ota/pin fontsize}$-$}
            \fi
            \ifotafullydifferential
                \ifotaflipoutputs
                    \pgftext[at={\pgfpointlineattime{\outputdistribution}{\pgfpointadd{\inputplus}{\pgfpoint{0cm}{0.5 * \inputheight * (1 - \inputdistribution)}}}{\outputminus} }, x = -3pt, y=-6pt]{\pgfkeysvalueof{/tikz/circuits/ota/pin fontsize}$+$}
                    \pgftext[at={\pgfpointlineattime{\outputdistribution}{\pgfpointadd{\inputminus}{\pgfpoint{0cm}{-0.5 * \inputheight * (1 - \inputdistribution)}}}{\outputplus} }, x = -3pt, y=6pt]{\pgfkeysvalueof{/tikz/circuits/ota/pin fontsize}$-$}
                \else
                    \pgftext[at={\pgfpointlineattime{\outputdistribution}{\pgfpointadd{\inputplus}{\pgfpoint{0cm}{0.5 * \inputheight * (1 - \inputdistribution)}}}{\outputminus} }, x = -3pt, y=-6pt]{\pgfkeysvalueof{/tikz/circuits/ota/pin fontsize}$-$}
                    \pgftext[at={\pgfpointlineattime{\outputdistribution}{\pgfpointadd{\inputminus}{\pgfpoint{0cm}{-0.5 * \inputheight * (1 - \inputdistribution)}}}{\outputplus} }, x = -3pt, y=6pt]{\pgfkeysvalueof{/tikz/circuits/ota/pin fontsize}$+$}
                \fi
            \fi
        \fi
    }
}

% vim: ft=plaintex nowrap

\newif\ifopampflipinputs
\newif\ifopampflipoutputs
\newif\ifopampfullydifferential
\newif\if@opamp@draw@input@labels
\newif\if@opamp@draw@output@labels

\tikzset{
    circuits/opamp/width/.initial = 2cm,
    circuits/opamp/height/.initial = 2cm,
    circuits/opamp/input distribution factor/.initial = 0.6,
    circuits/opamp/output distribution factor/.initial = 0.6,
    circuits/opamp/pin fontsize/.initial = {\tiny},
    circuits/opamp/flip inputs/.is if=opampflipinputs,
    circuits/opamp/flip outputs/.is if=opampflipoutputs,
    circuits/opamp/fully differential/.is if=opampfullydifferential,
    circuits/opamp/draw input lables/.is if=@opamp@draw@input@labels,
    circuits/opamp/draw output lables/.is if=@opamp@draw@output@labels,
    opamp/.style = {fdopampshape, circuits/opamp/fully differential=false},
    fdopamp/.style = {fdopampshape, circuits/opamp/fully differential=true}
}

\pgfdeclareshape{fdopampshape}
{
    \saveddimen{\width}{\pgf@x=\pgfkeysvalueof{/tikz/circuits/opamp/width}}
    \saveddimen{\inputheight}{\pgf@x=\pgfkeysvalueof{/tikz/circuits/opamp/height}}
    \savedmacro{\inputdistribution}{\renewcommand{\inputdistribution}[0]{\pgfkeysvalueof{/tikz/circuits/opamp/input distribution factor}}}
    \savedmacro{\outputdistribution}{\renewcommand{\outputdistribution}[0]{\pgfkeysvalueof{/tikz/circuits/opamp/output distribution factor}}}
    \savedanchor{\centerpoint}{\pgfpointorigin}
    \savedanchor{\inputplus}{%
        \pgfpointadd%
        {\pgfpointorigin}%
        {\pgfpoint%
            {-0.5 * \pgfkeysvalueof{/tikz/circuits/opamp/width}}%
            {0.5 * \pgfkeysvalueof{/tikz/circuits/opamp/height} * \pgfkeysvalueof{/tikz/circuits/opamp/input distribution factor}}%
        }%
    }
    \savedanchor{\inputminus}{%
        \pgfpointadd%
        {\pgfpointorigin}%
        {\pgfpoint%
            {-0.5 * \pgfkeysvalueof{/tikz/circuits/opamp/width}}%
            {-0.5 * \pgfkeysvalueof{/tikz/circuits/opamp/height} * \pgfkeysvalueof{/tikz/circuits/opamp/input distribution factor}}%
        }%
    }
    \savedanchor{\output}{%
        \pgfpointadd%
        {\pgfpointorigin}%
        {\pgfpoint%
            {0.5 * \pgfkeysvalueof{/tikz/circuits/opamp/width}}%
            {0cm}
        }%
    }
    % electrical terminals (anchors)
    \anchor{cmin}{\pgfpointadd{\centerpoint}{\pgfpoint{-\width/2}{0cm}}}
    \anchor{cmout}{\output}
    \anchor{out}{\output}
    \anchor{+in}{\inputplus}
    \anchor{-in}{\inputminus}
    \anchor{+out}{
        \pgfpointlineattime%
        {\outputdistribution}%
        {\pgfpointadd{\inputminus}{\pgfpoint{0cm}{-0.5 * \inputheight * (1 - \inputdistribution)}}}%
        {\output}%
    }
    \anchor{-out}{
        \pgfpointlineattime%
        {\outputdistribution}%
        {\pgfpointadd{\inputplus}{\pgfpoint{0cm}{0.5 * \inputheight * (1 - \inputdistribution)}}}%
        {\output}%
    }
    \anchor{+power}{
        \pgfpointlineattime%
        {0.5}%
        {\pgfpointadd{\inputplus}{\pgfpoint{0cm}{0.5 * \inputheight * (1 - \inputdistribution)}}}%
        {\output}%
    }
    \anchor{-power}{
        \pgfpointlineattime%
        {0.5}%
        {\pgfpointadd{\inputminus}{\pgfpoint{0cm}{-0.5 * \inputheight * (1 - \inputdistribution)}}}%
        {\output}%
    }

    % regular anchors
    \anchor{center}{\centerpoint}
    \anchor{west}{\pgfpointadd{\centerpoint}{\pgfpoint{-\width/2}{0cm}}}
    \anchor{east}{\output}
    \anchor{north}{
        \pgfpointadd{\inputplus}{\pgfpoint{\width/2}{0.5 * \inputheight * (1 - \inputdistribution)}}
    }
    \anchor{south}{
        \pgfpointadd{\inputminus}{\pgfpoint{\width/2}{-0.5 * \inputheight * (1 - \inputdistribution)}}
    }
    \anchor{north west}{
        \pgfpointadd{\inputplus}{\pgfpoint{0cm}{0.5 * \inputheight * (1 - \inputdistribution)}}
    }
    \anchor{south west}{
        \pgfpointadd{\inputminus}{\pgfpoint{0cm}{-0.5 * \inputheight * (1 - \inputdistribution)}}
    }
    \anchor{north east}{
        \pgfpointadd{\inputplus}{\pgfpoint{\width}{0.5 * \inputheight * (1 - \inputdistribution)}}
    }
    \anchor{south east}{
        \pgfpointadd{\inputminus}{\pgfpoint{\width}{-0.5 * \inputheight * (1 - \inputdistribution)}}
    }
    \beforebackgroundpath{
        \pgfsetlinewidth{\pgfkeysvalueof{/tikz/circuits/line width}}
        \pgfpathmoveto{\pgfpointadd{\inputplus}{\pgfpoint{0cm}{0.5 * \inputheight * (1 - \inputdistribution)}}}
        \pgfpathlineto{\output}
        \pgfpathlineto{\pgfpointadd{\inputminus}{\pgfpoint{0cm}{-0.5 * \inputheight * (1 - \inputdistribution)}}}
        \pgfpathclose
        \pgfusepath{stroke}
        \if@opamp@draw@input@labels
            \ifopampflipinputs
                \pgftext[left, at={\inputplus}, x=2pt]{\tiny$-$}
                \pgftext[left, at={\inputminus}, x=2pt]{\tiny$+$}
            \else
                \pgftext[left, at={\inputplus}, x=2pt]{\tiny$+$}
                \pgftext[left, at={\inputminus}, x=2pt]{\tiny$-$}
            \fi
        \fi
        \if@opamp@draw@output@labels
            \ifopampfullydifferential
                \ifopampflipoutputs
                    \pgftext[at={\pgfpointlineattime{\outputdistribution}{\pgfpointadd{\inputplus}{\pgfpoint{0cm}{0.5 * \inputheight * (1 - \inputdistribution)}}}{\output} }, x = -3pt, y=-3pt]{\tiny$+$}
                    \pgftext[at={\pgfpointlineattime{\outputdistribution}{\pgfpointadd{\inputminus}{\pgfpoint{0cm}{-0.5 * \inputheight * (1 - \inputdistribution)}}}{\output} }, x = -3pt, y=3pt]{\tiny$-$}
                \else
                    \pgftext[at={\pgfpointlineattime{\outputdistribution}{\pgfpointadd{\inputplus}{\pgfpoint{0cm}{0.5 * \inputheight * (1 - \inputdistribution)}}}{\output} }, x = -3pt, y=-3pt]{\tiny$-$}
                    \pgftext[at={\pgfpointlineattime{\outputdistribution}{\pgfpointadd{\inputminus}{\pgfpoint{0cm}{-0.5 * \inputheight * (1 - \inputdistribution)}}}{\output} }, x = -3pt, y=3pt]{\tiny$+$}
                \fi
            \fi
        \fi
    }
}

% vim: ft=plaintex nowrap

\newif\ifcomparator@drawinputlabels
\newif\ifcomparator@flip@inputs
\newif\ifcomparator@flip@outputs
\newif\ifcomparator@fullydifferential

\tikzset{
    circuits/comparator/width/.initial = 1cm,
    circuits/comparator/height/.initial = 1cm,
    circuits/comparator/annotation width/.initial = 4pt,
    circuits/comparator/annotation height/.initial = 6pt,
    circuits/comparator/input distribution factor/.initial = 0.4,
    circuits/comparator/output distribution factor/.initial = 0.6,
    circuits/comparator/label size/.initial = 5pt,
    circuits/comparator/label shift/.initial = 5pt,
    circuits/comparator/draw input labels/.is if=comparator@drawinputlabels,
    circuits/comparator/flip inputs/.is if=comparator@flip@inputs,
    circuits/comparator/flip outputs/.is if=comparator@flip@outputs,
    circuits/comparator/fully differential/.is if=comparator@fullydifferential,
    comparator/.style = {fdcomparatorshape, circuits/comparator/draw input labels = true, circuits/comparator/fully differential=false},
    fdcomparator/.style = {fdcomparatorshape, circuits/comparator/draw input labels = true, circuits/comparator/fully differential=true}
}

\pgfdeclareshape{fdcomparatorshape}
{
    \saveddimen{\circuits@comparator@width}{\pgfpointscale{\pgfkeysvalueof{/tikz/circuits/scale}}{\pgfpoint{\pgfkeysvalueof{/tikz/circuits/comparator/width}}{0pt}}}
    \saveddimen{\circuits@comparator@anno@width}{\pgf@x=\pgfkeysvalueof{/tikz/circuits/comparator/annotation width}}
    \saveddimen{\circuits@comparator@anno@height}{\pgf@x=\pgfkeysvalueof{/tikz/circuits/comparator/annotation height}}
    \saveddimen{\circuits@comparator@inputheight}{\pgfpointscale{\pgfkeysvalueof{/tikz/circuits/scale}}{\pgfpoint{\pgfkeysvalueof{/tikz/circuits/comparator/height}}{0pt}}}
    \savedmacro{\circuits@comparator@inputdistribution}{\renewcommand{\circuits@comparator@inputdistribution}[0]{\pgfkeysvalueof{/tikz/circuits/comparator/input distribution factor}}}
    \savedmacro{\circuits@comparator@outputdistribution}{\renewcommand{\circuits@comparator@outputdistribution}[0]{\pgfkeysvalueof{/tikz/circuits/comparator/output distribution factor}}}
    \saveddimen{\circuits@comparator@labelsize}{\pgfpointscale{\pgfkeysvalueof{/tikz/circuits/scale}}{\pgfpoint{\pgfkeysvalueof{/tikz/circuits/comparator/label size}}{0pt}}}
    \saveddimen{\circuits@comparator@labelshift}{\pgfpointscale{\pgfkeysvalueof{/tikz/circuits/scale}}{\pgfpoint{\pgfkeysvalueof{/tikz/circuits/comparator/label shift}}{0pt}}}
    \savedanchor{\circuits@comparator@input}{%
        \pgfpointadd%
        {\pgfpointorigin}%
        {\pgfpoint%
            {-0.5 * \pgfkeysvalueof{/tikz/circuits/scale} * \pgfkeysvalueof{/tikz/circuits/comparator/width}}%
            {0cm}%
        }%
    }
    \savedanchor{\circuits@comparator@inputplus}{%
        \pgfpointadd%
        {\pgfpointorigin}%
        {\pgfpoint%
            {-0.5 * \pgfkeysvalueof{/tikz/circuits/scale} * \pgfkeysvalueof{/tikz/circuits/comparator/width}}%
            {0.5 * \pgfkeysvalueof{/tikz/circuits/scale} * \pgfkeysvalueof{/tikz/circuits/comparator/height} * \pgfkeysvalueof{/tikz/circuits/comparator/input distribution factor}}%
        }%
    }
    \savedanchor{\circuits@comparator@inputminus}{%
        \pgfpointadd%
        {\pgfpointorigin}%
        {\pgfpoint%
            {-0.5 * \pgfkeysvalueof{/tikz/circuits/scale} * \pgfkeysvalueof{/tikz/circuits/comparator/width}}%
            {-0.5 * \pgfkeysvalueof{/tikz/circuits/scale} * \pgfkeysvalueof{/tikz/circuits/comparator/height} * \pgfkeysvalueof{/tikz/circuits/comparator/input distribution factor}}%
        }%
    }
    \savedanchor{\circuits@comparator@output}{%
        \pgfpointadd%
        {\pgfpointorigin}%
        {\pgfpoint%
            {0.5 * \pgfkeysvalueof{/tikz/circuits/scale} * \pgfkeysvalueof{/tikz/circuits/comparator/width}}%
            {0cm}
        }%
    }
    % electrical terminals (anchors)
    \anchor{cmin}{\pgfpointadd{\pgfpointorigin}{\pgfpoint{-\circuits@comparator@width/2}{0cm}}}
    \anchor{cmout}{\circuits@comparator@output}
    \anchor{out}{\circuits@comparator@output}
    \anchor{in}{\circuits@comparator@input}
    \anchor{+in}{\circuits@comparator@inputplus}
    \anchor{-in}{\circuits@comparator@inputminus}
    \anchor{+out}{
        \pgfpointlineattime%
        {\circuits@comparator@outputdistribution}%
        {\pgfpointadd{\circuits@comparator@inputplus}{\pgfpoint{0cm}{0.5 * \circuits@comparator@inputheight * (1 - \circuits@comparator@inputdistribution)}}}%
        {\circuits@comparator@output}%
    }
    \anchor{-out}{
        \pgfpointlineattime%
        {\circuits@comparator@outputdistribution}%
        {\pgfpointadd{\circuits@comparator@inputminus}{\pgfpoint{0cm}{-0.5 * \circuits@comparator@inputheight * (1 - \circuits@comparator@inputdistribution)}}}%
        {\circuits@comparator@output}%
    }
    \anchor{+power}{
        \pgfpointlineattime%
        {0.5}%
        {\pgfpointadd{\circuits@comparator@inputplus}{\pgfpoint{0cm}{0.5 * \circuits@comparator@inputheight * (1 - \circuits@comparator@inputdistribution)}}}%
        {\circuits@comparator@output}%
    }
    \anchor{-power}{
        \pgfpointlineattime%
        {0.5}%
        {\pgfpointadd{\circuits@comparator@inputminus}{\pgfpoint{0cm}{-0.5 * \circuits@comparator@inputheight * (1 - \circuits@comparator@inputdistribution)}}}%
        {\circuits@comparator@output}%
    }

    % regular anchors
    \anchor{center}{\pgfpointorigin}
    \anchor{west}{\pgfpointadd{\pgfpointorigin}{\pgfpoint{-\circuits@comparator@width/2}{0cm}}}
    \anchor{east}{\circuits@comparator@output}
    \anchor{north}{
        \pgfpointadd{\circuits@comparator@inputplus}{\pgfpoint{\circuits@comparator@width/2}{0.5 * \circuits@comparator@inputheight * (1 - \circuits@comparator@inputdistribution)}}
    }
    \anchor{south}{
        \pgfpointadd{\circuits@comparator@inputminus}{\pgfpoint{\circuits@comparator@width/2}{-0.5 * \circuits@comparator@inputheight * (1 - \circuits@comparator@inputdistribution)}}
    }
    \anchor{north west}{
        \pgfpointadd{\circuits@comparator@inputplus}{\pgfpoint{0cm}{0.5 * \circuits@comparator@inputheight * (1 - \circuits@comparator@inputdistribution)}}
    }
    \anchor{south west}{
        \pgfpointadd{\circuits@comparator@inputminus}{\pgfpoint{0cm}{-0.5 * \circuits@comparator@inputheight * (1 - \circuits@comparator@inputdistribution)}}
    }
    \anchor{north east}{
        \pgfpointadd{\circuits@comparator@inputplus}{\pgfpoint{\circuits@comparator@width}{0.5 * \circuits@comparator@inputheight * (1 - \circuits@comparator@inputdistribution)}}
    }
    \anchor{south east}{
        \pgfpointadd{\circuits@comparator@inputminus}{\pgfpoint{\circuits@comparator@width}{-0.5 * \circuits@comparator@inputheight * (1 - \circuits@comparator@inputdistribution)}}
    }
    \backgroundpath{
        \pgfsetlinewidth{\pgfkeysvalueof{/tikz/circuits/line width}}
        \pgfpathmoveto{\pgfpointadd{\circuits@comparator@inputplus}{\pgfpoint{0cm}{0.5 * \circuits@comparator@inputheight * (1 - \circuits@comparator@inputdistribution)}}}
        \pgfpathlineto{\circuits@comparator@output}
        \pgfpathlineto{\pgfpointadd{\circuits@comparator@inputminus}{\pgfpoint{0cm}{-0.5 * \circuits@comparator@inputheight * (1 - \circuits@comparator@inputdistribution)}}}
        \pgfpathclose
    }
    \beforebackgroundpath{
        \pgfpathmoveto{\pgfpointadd{\pgfpointorigin}{\pgfpoint{-\circuits@comparator@anno@width}{-0.5 * \circuits@comparator@anno@height}}}
        \pgfpathlineto{\pgfpointadd{\pgfpointorigin}{\pgfpoint{-0.5 * \circuits@comparator@anno@width}{-0.5 * \circuits@comparator@anno@height}}}
        \pgfpathlineto{\pgfpointadd{\pgfpointorigin}{\pgfpoint{-0.5 * \circuits@comparator@anno@width}{0.5 * \circuits@comparator@anno@height}}}
        \pgfpathlineto{\pgfpointadd{\pgfpointorigin}{\pgfpoint{0pt}{0.5 * \circuits@comparator@anno@height}}}
        \pgfusepath{stroke}
        \ifcomparator@drawinputlabels
            \ifcomparator@flip@inputs
                \pgfpathmoveto{\pgfpointadd{\circuits@comparator@inputminus}{\pgfpoint{\circuits@comparator@labelshift - 0.5 * \circuits@comparator@labelsize}{0pt}}}
                \pgfpathlineto{\pgfpointadd{\circuits@comparator@inputminus}{\pgfpoint{\circuits@comparator@labelshift + 0.5 * \circuits@comparator@labelsize}{0pt}}}
                \pgfusepath{stroke}
                \pgfpathmoveto{\pgfpointadd{\circuits@comparator@inputminus}{\pgfpoint{\circuits@comparator@labelshift}{-0.5 * \circuits@comparator@labelsize}}}
                \pgfpathlineto{\pgfpointadd{\circuits@comparator@inputminus}{\pgfpoint{\circuits@comparator@labelshift}{0.5 * \circuits@comparator@labelsize}}}
                \pgfusepath{stroke}
                \pgfpathmoveto{\pgfpointadd{\circuits@comparator@inputplus}{\pgfpoint{\circuits@comparator@labelshift - 0.5 * \circuits@comparator@labelsize}{0pt}}}
                \pgfpathlineto{\pgfpointadd{\circuits@comparator@inputplus}{\pgfpoint{\circuits@comparator@labelshift + 0.5 * \circuits@comparator@labelsize}{0pt}}}
                \pgfusepath{stroke}
            \else
                \pgfpathmoveto{\pgfpointadd{\circuits@comparator@inputplus}{\pgfpoint{\circuits@comparator@labelshift - 0.5 * \circuits@comparator@labelsize}{0pt}}}
                \pgfpathlineto{\pgfpointadd{\circuits@comparator@inputplus}{\pgfpoint{\circuits@comparator@labelshift + 0.5 * \circuits@comparator@labelsize}{0pt}}}
                \pgfusepath{stroke}
                \pgfpathmoveto{\pgfpointadd{\circuits@comparator@inputplus}{\pgfpoint{\circuits@comparator@labelshift}{-0.5 * \circuits@comparator@labelsize}}}
                \pgfpathlineto{\pgfpointadd{\circuits@comparator@inputplus}{\pgfpoint{\circuits@comparator@labelshift}{0.5 * \circuits@comparator@labelsize}}}
                \pgfusepath{stroke}
                \pgfpathmoveto{\pgfpointadd{\circuits@comparator@inputminus}{\pgfpoint{\circuits@comparator@labelshift - 0.5 * \circuits@comparator@labelsize}{0pt}}}
                \pgfpathlineto{\pgfpointadd{\circuits@comparator@inputminus}{\pgfpoint{\circuits@comparator@labelshift + 0.5 * \circuits@comparator@labelsize}{0pt}}}
                \pgfusepath{stroke}
            \fi
        \fi
        \ifcomparator@fullydifferential
            \ifcomparator@flip@outputs
                \pgftext[at={\pgfpointlineattime{\circuits@comparator@outputdistribution}{\pgfpointadd{\circuits@comparator@inputplus}{\pgfpoint{0cm}{0.5 * \circuits@comparator@inputheight * (1 -
                \circuits@comparator@inputdistribution)}}}{\circuits@comparator@output} }, x = -3pt, y=-3pt]{\tiny$+$}
                \pgftext[at={\pgfpointlineattime{\circuits@comparator@outputdistribution}{\pgfpointadd{\circuits@comparator@inputminus}{\pgfpoint{0cm}{-0.5 * \circuits@comparator@inputheight * (1 -
                \circuits@comparator@inputdistribution)}}}{\circuits@comparator@output} }, x = -3pt, y=3pt]{\tiny$-$}
            \else
                \pgftext[at={\pgfpointlineattime{\circuits@comparator@outputdistribution}{\pgfpointadd{\circuits@comparator@inputplus}{\pgfpoint{0cm}{0.5 * \circuits@comparator@inputheight * (1 -
                \circuits@comparator@inputdistribution)}}}{\circuits@comparator@output} }, x = -3pt, y=-3pt]{\tiny$-$}
                \pgftext[at={\pgfpointlineattime{\circuits@comparator@outputdistribution}{\pgfpointadd{\circuits@comparator@inputminus}{\pgfpoint{0cm}{-0.5 * \circuits@comparator@inputheight * (1 -
                \circuits@comparator@inputdistribution)}}}{\circuits@comparator@output} }, x = -3pt, y=3pt]{\tiny$+$}
            \fi
        \fi
    }
}

% vim: ft=plaintex nowrap

\newif\if@circuits@capacitor@variable

\tikzset{
    circuits/capacitor/width/.initial = 0.1cm,
    circuits/capacitor/height/.initial = 0.5cm,
    circuits/capacitor/variable/.is if=@circuits@capacitor@variable,
    capacitor/.style = {capacitorshape, anchor = plus}
}

\pgfdeclareshape{capacitorshape}
{
    \saveddimen{\width}{\pgf@x=\pgfkeysvalueof{/tikz/circuits/capacitor/width}}
    \saveddimen{\height}{\pgf@x=\pgfkeysvalueof{/tikz/circuits/capacitor/height}}

    % electrical terminals
    \anchor{plus}{\pgfpointadd{\pgfpointorigin}{\pgfpoint{0.5 * \width}{0pt}}}
    \anchor{minus}{\pgfpointadd{\pgfpointorigin}{\pgfpoint{-0.5 * \width}{0pt}}}
    % general anchors
    \anchor{center}{\pgfpointorigin}
    \anchor{west}{\pgfpointadd{\pgfpointorigin}{\pgfpoint{-0.5 * \width}{0pt}}}
    \anchor{east}{\pgfpointadd{\pgfpointorigin}{\pgfpoint{0.5 * \width}{0pt}}}
    \anchor{north}{\pgfpointadd{\pgfpointorigin}{\pgfpoint{0cm}{0.5 * \height}}}
    \anchor{south}{\pgfpointadd{\pgfpointorigin}{\pgfpoint{0cm}{-0.5 * \height}}}
    \anchor{south east}{\pgfpointadd{\pgfpointorigin}{\pgfpoint{0.5 * \height}{-0.5 * \width}}}
    \anchor{south west}{\pgfpointadd{\pgfpointorigin}{\pgfpoint{-0.5 * \height}{-0.5 * \width}}}
    \anchor{north east}{\pgfpointadd{\pgfpointorigin}{\pgfpoint{0.5 * \height}{0.5 * \width}}}
    \anchor{north west}{\pgfpointadd{\pgfpointorigin}{\pgfpoint{-0.5 * \height}{0.5 * \width}}}
    \beforebackgroundpath{
        \pgfsetlinewidth{\pgfkeysvalueof{/tikz/circuits/line width}}
        \pgfpathmoveto{\pgfpointadd{\pgfpointorigin}{\pgfpoint{-0.5 * \width}{0.5 * \height}}}
        \pgfpathlineto{\pgfpointadd{\pgfpointorigin}{\pgfpoint{-0.5 * \width}{-0.5 * \height}}}
        %\pgfusepath{stroke}
        \pgfpathmoveto{\pgfpointadd{\pgfpointorigin}{\pgfpoint{0.5 * \width}{0.5 * \height}}}
        \pgfpathlineto{\pgfpointadd{\pgfpointorigin}{\pgfpoint{0.5 * \width}{-0.5 * \height}}}
        \pgfusepath{stroke}
        \if@circuits@capacitor@variable
            \pgfpathmoveto{\pgfpointadd{\pgfpointorigin}{\pgfpoint{-1.75 * \width}{-0.6 * \height}}}
            \pgfpathlineto{\pgfpointadd{\pgfpointorigin}{\pgfpoint{ 1.75 * \width}{ 0.6 * \height}}}
            \pgfsetarrows{-{Stealth[length=3pt, width=3pt]}}
            \pgfusepath{stroke}
        \fi
    }
}

% vim: ft=plaintex

\tikzset{
    circuits/resistor/width/.initial = 0.6cm,
    circuits/resistor/height/.initial = 0.25cm,
    circuits/resistor/segments/.initial = 3,
    circuits/resistor/terminal extension/.initial = 0.05cm,
    resistor/.style = {americanresistorshape, draw, wire}
}

\pgfdeclareshape{americanresistorshape}
{
    \saveddimen{\width}{\pgf@x=\pgfkeysvalueof{/tikz/circuits/resistor/width}}
    \saveddimen{\height}{\pgf@x=\pgfkeysvalueof{/tikz/circuits/resistor/height}}
    \saveddimen{\extension}{\pgf@x=\pgfkeysvalueof{/tikz/circuits/resistor/terminal extension}}
    \savedmacro{\segments}{\renewcommand{\segments}[0]{\pgfkeysvalueof{/tikz/circuits/resistor/segments}}}
    \savedanchor{\leftpoint}{
        \pgfpointadd{\pgfpointorigin}{\pgfpoint{-0.5 * \pgfkeysvalueof{/tikz/circuits/resistor/width}- \pgfkeysvalueof{/tikz/circuits/resistor/terminal extension}}{0cm}}
    }
    \savedanchor{\rightpoint}{
        \pgfpointadd{\pgfpointorigin}{\pgfpoint{0.5 * \pgfkeysvalueof{/tikz/circuits/resistor/width} + \pgfkeysvalueof{/tikz/circuits/resistor/terminal extension}}{0cm}}
    }

    % electrical terminals
    \anchor{plus}{\leftpoint}
    \anchor{minus}{\rightpoint}
    % general anchors
    \anchor{center}{\pgfpointorigin}
    \anchor{west}{\leftpoint}
    \anchor{east}{\rightpoint}
    \anchor{north}{\pgfpointadd{\pgfpointorigin}{\pgfpoint{0cm}{0.5 * \height}}}
    \anchor{south}{\pgfpointadd{\pgfpointorigin}{\pgfpoint{0cm}{-0.5 * \height}}}
    \anchor{south east}{\pgfpointadd{\pgfpointorigin}{\pgfpoint{0.5 * \height}{-0.5 * \width}}}
    \anchor{south west}{\pgfpointadd{\pgfpointorigin}{\pgfpoint{-0.5 * \height}{-0.5 * \width}}}
    \anchor{north east}{\pgfpointadd{\pgfpointorigin}{\pgfpoint{0.5 * \height}{0.5 * \width}}}
    \anchor{north west}{\pgfpointadd{\pgfpointorigin}{\pgfpoint{-0.5 * \height}{0.5 * \width}}}
    \backgroundpath{
        \pgfsetmiterjoin
        \pgfsetmiterlimit{5}
        \pgfpathmoveto{\leftpoint}
        \pgfpathlineto{\pgfpointadd{\leftpoint}{\pgfpoint{\extension}{0cm}}}
        \foreach \x [count=\xi, evaluate=\xi as \y using isodd(\xi)-0.5] in {0.25,0.75,...,\segments}
        {
            \pgfpathlineto{\pgfpointadd{\leftpoint}{\pgfpoint{\extension + 4 * \x * \width / (4 * \segments)}{\y * \height}}}
        }
        \pgfpathlineto{\pgfpointadd{\rightpoint}{\pgfpoint{-\extension}{0cm}}}
        \pgfpathlineto{\rightpoint}
    }
}
\pgfdeclareshape{europeenresistorshape}
{
    \saveddimen{\width}{\pgf@x=\pgfkeysvalueof{/tikz/circuits/resistor/width}}
    \saveddimen{\height}{\pgf@x=\pgfkeysvalueof{/tikz/circuits/resistor/height}}
    \savedanchor{\lowerleft}{
        \pgfpointadd{\pgfpointorigin}{\pgfpoint{-0.5 * \pgfkeysvalueof{/tikz/circuits/resistor/width}}{-0.5 * \pgfkeysvalueof{/tikz/circuits/resistor/height}}}
    }
    \savedanchor{\upperright}{
        \pgfpointadd{\pgfpointorigin}{\pgfpoint{0.5 * \pgfkeysvalueof{/tikz/circuits/resistor/width}}{0.5 * \pgfkeysvalueof{/tikz/circuits/resistor/height}}}
    }

    % electrical terminals
    \anchor{plus}{\pgfpointadd{\pgfpointorigin}{\pgfpoint{-0.5 * \width}{0cm}}}
    \anchor{minus}{\pgfpointadd{\pgfpointorigin}{\pgfpoint{0.5 * \width}{0cm}}}
    % general anchors
    \anchor{lowerleft}{\lowerleft}
    \anchor{upperright}{\upperright}
    \anchor{center}{\pgfpointorigin}
    \anchor{west}{\pgfpointadd{\pgfpointorigin}{\pgfpoint{-0.5 * \width}{0cm}}}
    \anchor{east}{\pgfpointadd{\pgfpointorigin}{\pgfpoint{0.5 * \width}{0cm}}}
    \anchor{north}{\pgfpointadd{\pgfpointorigin}{\pgfpoint{0cm}{0.5 * \height}}}
    \anchor{south}{\pgfpointadd{\pgfpointorigin}{\pgfpoint{0cm}{-0.5 * \height}}}
    \anchor{south east}{\pgfpointadd{\pgfpointorigin}{\pgfpoint{0.5 * \height}{-0.5 * \width}}}
    \anchor{south west}{\pgfpointadd{\pgfpointorigin}{\pgfpoint{-0.5 * \height}{-0.5 * \width}}}
    \anchor{north east}{\pgfpointadd{\pgfpointorigin}{\pgfpoint{0.5 * \height}{0.5 * \width}}}
    \anchor{north west}{\pgfpointadd{\pgfpointorigin}{\pgfpoint{-0.5 * \height}{0.5 * \width}}}
    \backgroundpath{
        \pgfpathrectanglecorners{\lowerleft}{\upperright}
    }
}

% vim: ft=plaintex

\tikzset{
    circuits/inductor/width/.initial = 0.8cm,
    circuits/inductor/height/.initial = 0.266667cm,
    circuits/inductor/segments/.initial = 3,
    circuits/inductor/terminal extension/.initial = 0.05cm,
    inductor/.style = {americaninductorshape}
}

\pgfdeclareshape{americaninductorshape}
{
    \saveddimen{\width}{\pgf@x=\pgfkeysvalueof{/tikz/circuits/inductor/width}}
    \saveddimen{\height}{\pgf@x=\pgfkeysifdefined{/tikz/circuits/inductor/height}{\pgfkeysvalueof{/tikz/circuits/inductor/height}}{\pgfkeysvalueof{/tikz/circuits/inductor/width}}}
    \saveddimen{\extension}{\pgf@x=\pgfkeysvalueof{/tikz/circuits/inductor/terminal extension}}
    \savedmacro{\segments}{\renewcommand{\segments}[0]{\pgfkeysvalueof{/tikz/circuits/inductor/segments}}}
    \savedanchor{\leftpoint}{
        \pgfpointadd{\pgfpointorigin}{\pgfpoint{-0.5 * \pgfkeysvalueof{/tikz/circuits/inductor/width}- \pgfkeysvalueof{/tikz/circuits/inductor/terminal extension}}{0cm}}
    }
    \savedanchor{\rightpoint}{
        \pgfpointadd{\pgfpointorigin}{\pgfpoint{0.5 * \pgfkeysvalueof{/tikz/circuits/inductor/width} + \pgfkeysvalueof{/tikz/circuits/inductor/terminal extension}}{0cm}}
    }

    % electrical terminals
    \anchor{plus}{\leftpoint}
    \anchor{minus}{\rightpoint}
    % general anchors
    \anchor{center}{\pgfpointorigin}
    \anchor{west}{\leftpoint}
    \anchor{inner west}{
        \pgfpointadd{\pgfpointorigin}{
            \pgfpoint
            {-0.5 * \width}
            {0cm}
        }
    }
    \anchor{east}{\rightpoint}
    \anchor{north}{\pgfpointadd{\pgfpointorigin}{\pgfpoint{0cm}{0.5 * \height}}}
    \anchor{south}{\pgfpointadd{\pgfpointorigin}{\pgfpoint{0cm}{-0.5 * \height}}}
    \anchor{south east}{\pgfpointadd{\pgfpointorigin}{\pgfpoint{0.5 * \height}{-0.5 * \width}}}
    \anchor{south west}{\pgfpointadd{\pgfpointorigin}{\pgfpoint{-0.5 * \height}{-0.5 * \width}}}
    \anchor{north east}{\pgfpointadd{\pgfpointorigin}{\pgfpoint{0.5 * \height}{0.5 * \width}}}
    \anchor{north west}{\pgfpointadd{\pgfpointorigin}{\pgfpoint{-0.5 * \height}{0.5 * \width}}}
    \beforebackgroundpath{
        \pgfsetlinewidth{\pgfkeysvalueof{/tikz/circuits/line width}}
        \pgfpathmoveto{\leftpoint}
        \pgfpathlineto{\pgfpointadd{\leftpoint}{\pgfpoint{\extension}{0cm}}}
        \foreach \x in {1,...,\segments}
        {
            \pgfpatharc{180}{0}{\width / \segments / 2 and \height/2}
        }
        \pgfpathlineto{\pgfpointadd{\rightpoint}{\pgfpoint{-\extension}{0cm}}}
        \pgfpathlineto{\rightpoint}
        \pgfusepath{stroke}
    }
}
\pgfdeclareshape{europeeninductorshape}
{
    \saveddimen{\width}{\pgf@x=\pgfkeysvalueof{/tikz/circuits/inductor/width}}
    \saveddimen{\height}{\pgf@x=\pgfkeysvalueof{/tikz/circuits/inductor/height}}
    \savedanchor{\lowerleft}{
        \pgfpointadd{\pgfpointorigin}{\pgfpoint{-0.5 * \pgfkeysvalueof{/tikz/circuits/inductor/width}}{-0.5 * \pgfkeysvalueof{/tikz/circuits/inductor/height}}}
    }
    \savedanchor{\upperright}{
        \pgfpointadd{\pgfpointorigin}{\pgfpoint{0.5 * \pgfkeysvalueof{/tikz/circuits/inductor/width}}{0.5 * \pgfkeysvalueof{/tikz/circuits/inductor/height}}}
    }

    % electrical terminals
    \anchor{plus}{\pgfpointadd{\pgfpointorigin}{\pgfpoint{-0.5 * \width}{0cm}}}
    \anchor{minus}{\pgfpointadd{\pgfpointorigin}{\pgfpoint{0.5 * \width}{0cm}}}
    % general anchors
    \anchor{lowerleft}{\lowerleft}
    \anchor{upperright}{\upperright}
    \anchor{center}{\pgfpointorigin}
    \anchor{west}{\pgfpointadd{\pgfpointorigin}{\pgfpoint{-0.5 * \width}{0cm}}}
    \anchor{east}{\pgfpointadd{\pgfpointorigin}{\pgfpoint{0.5 * \width}{0cm}}}
    \anchor{north}{\pgfpointadd{\pgfpointorigin}{\pgfpoint{0cm}{0.5 * \height}}}
    \anchor{south}{\pgfpointadd{\pgfpointorigin}{\pgfpoint{0cm}{-0.5 * \height}}}
    \anchor{south east}{\pgfpointadd{\pgfpointorigin}{\pgfpoint{0.5 * \width}{-0.5 * \height}}}
    \anchor{south west}{\pgfpointadd{\pgfpointorigin}{\pgfpoint{-0.5 * \width}{-0.5 * \height}}}
    \anchor{north east}{\pgfpointadd{\pgfpointorigin}{\pgfpoint{0.5 * \width}{0.5 * \height}}}
    \anchor{north west}{\pgfpointadd{\pgfpointorigin}{\pgfpoint{-0.5 * \width}{0.5 * \height}}}
    \beforebackgroundpath{
        \pgfsetlinewidth{\pgfkeysvalueof{/tikz/circuits/line width}}
        \pgfpathrectanglecorners{\lowerleft}{\upperright}
        \pgfusepath{fill}
    }
}

% vim: ft=plaintex

\newif\if@circuits@mosfet@draw@bulk
\newif\if@circuits@mosfet@draw@bulk@arrow
\newif\if@circuits@mosfet@draw@source@arrow
\newif\if@circuits@mosfet@draw@gate@circle
\newif\if@circuits@mosfet@fill@gate@circle
\newif\if@circuits@mosfet@nmos

\tikzset{
    circuits/mosfet/scale/.initial=1,
    % mosfet drawing options
    circuits/mosfet/draw source arrow/.is if=@circuits@mosfet@draw@source@arrow,
    circuits/mosfet/draw source arrow/.initial = true,
    circuits/mosfet/draw bulk/.is if=@circuits@mosfet@draw@bulk,
    circuits/mosfet/draw bulk arrow/.is if=@circuits@mosfet@draw@bulk@arrow,
    circuits/pmos/draw gate circle/.is if=@circuits@mosfet@draw@gate@circle,
    circuits/pmos/fill gate circle/.is if=@circuits@mosfet@fill@gate@circle,
    circuits/mosfet/label separation/.initial = {5pt},
    % drawing helper keys
    circuits/mosfet/private/drawnmos/.is if=@circuits@mosfet@nmos,
}

\tikzset{
    % drawing styles
    circuits/mosfet/gate width/.initial          = 0.6cm,
    circuits/mosfet/gate skip/.initial           = 0.1cm,
    circuits/mosfet/drain length/.initial        = 0.2cm,
    circuits/mosfet/source length/.initial       = 0.2cm,
    circuits/mosfet/channel height/.initial      = 0.32cm,
    circuits/mosfet/channel length/.initial      = 0.6cm,
    circuits/mosfet/arrow length/.initial        = 1.75mm,
    circuits/mosfet/arrow width/.initial         = 1.75mm,
    circuits/mosfet/gate line width/.initial     = 1.2pt,
    circuits/mosfet/bulk line width/.initial     = 0.8pt,
    circuits/mosfet/gate dot radius/.initial     = 1.5pt,
    circuits/mosfet/gate dot line width/.initial = 0.8pt,
    % shape interface
    nmos/.style = {nmosshape, circuits/mosfet/private/drawnmos=true},
    pmos/.style = {pmosshape, circuits/mosfet/private/drawnmos=false}
}

\pgfdeclareshape{genericmosshape}
{
    \saveddimen{\gatewidth}{\pgf@x=\pgfkeysvalueof{/tikz/circuits/mosfet/gate width}}
    \saveddimen{\gateskip}{\pgf@x=\pgfkeysvalueof{/tikz/circuits/mosfet/gate skip}}
    \saveddimen{\drainlength}{\pgf@x=\pgfkeysvalueof{/tikz/circuits/mosfet/drain length}}
    \saveddimen{\sourcelength}{\pgf@x=\pgfkeysvalueof{/tikz/circuits/mosfet/source length}}
    \saveddimen{\channellength}{\pgf@x=\pgfkeysvalueof{/tikz/circuits/mosfet/channel length}}
    \saveddimen{\channelheight}{\pgf@x=\pgfkeysvalueof{/tikz/circuits/mosfet/channel height}}
    \saveddimen{\arrowlength}{\pgf@x=\pgfkeysvalueof{/tikz/circuits/mosfet/arrow length}}
    \saveddimen{\arrowwidth}{\pgf@x=\pgfkeysvalueof{/tikz/circuits/mosfet/arrow width}}
    \saveddimen{\circleradius}{\pgf@x=\pgfkeysvalueof{/tikz/circuits/mosfet/gate dot radius}}
    \saveddimen{\circuits@mosfet@label@sep}{\pgf@x=\pgfkeysvalueof{/tikz/circuits/mosfet/label separation}}
    \savedmacro{\scale}{\edef\scale{\pgfkeysvalueof{/tikz/circuits/mosfet/scale}}}

    % electrical terminals
    \anchor{placementgate}{\pgfpointadd{\pgfpointorigin}{\pgfpoint{-\channelheight -\gateskip}{0cm}}}
    \savedanchor{\gate}{%
        \if@circuits@mosfet@draw@gate@circle%
            \if@circuits@mosfet@nmos%
                \pgfpoint{-\pgfkeysvalueof{/tikz/circuits/mosfet/channel height} - \pgfkeysvalueof{/tikz/circuits/mosfet/gate skip} }{0cm}%
            \else%
                \pgfpoint{-\pgfkeysvalueof{/tikz/circuits/mosfet/channel height} - \pgfkeysvalueof{/tikz/circuits/mosfet/gate skip} - 2 * \pgfkeysvalueof{/tikz/circuits/mosfet/gate dot radius}}{0cm}%
            \fi%
        \else%
            \pgfpoint{-\pgfkeysvalueof{/tikz/circuits/mosfet/channel height} - \pgfkeysvalueof{/tikz/circuits/mosfet/gate skip}}{0cm}%
        \fi%
    }
    \deferredanchor{gate}{\gate}
    \anchor{bulk}{\pgfpointorigin}

    % general anchors
    % these are all the same, as this type of placement makes more sense for mosfets
    \anchor{center}{\pgfpointadd{\pgfpointorigin}{\pgfpoint{-\scale * (\channelheight + \gateskip)/2}{0cm}}}
    \anchor{east}{\pgfpointadd{\pgfpointorigin}{\pgfpoint{-\scale * (\channelheight + \gateskip)}{0cm}}}
    \anchor{west}{\pgfpointadd{\pgfpointorigin}{\pgfpoint{-\scale * (\channelheight + \gateskip)}{0cm}}}
    \anchor{north}{\pgfpointadd{\pgfpointorigin}{\pgfpoint{-\scale * (\channelheight + \gateskip)}{0cm}}}
    \anchor{south}{\pgfpointadd{\pgfpointorigin}{\pgfpoint{-\scale * (\channelheight + \gateskip)}{0cm}}}
    \anchor{north east}{\pgfpointadd{\pgfpointorigin}{\pgfpoint{-\scale * (\channelheight + \gateskip)}{0cm}}}
    \anchor{north west}{\pgfpointadd{\pgfpointorigin}{\pgfpoint{-\scale * (\channelheight + \gateskip)}{0cm}}}
    \anchor{south east}{\pgfpointadd{\pgfpointorigin}{\pgfpoint{-\scale * (\channelheight + \gateskip)}{0cm}}}
    \anchor{south west}{\pgfpointadd{\pgfpointorigin}{\pgfpoint{-\scale * (\channelheight + \gateskip)}{0cm}}}

    % text anchor
    \deferredanchor{text}{
        \pgfpointorigin
        \pgf@x=\circuits@mosfet@label@sep
        \pgf@y=-0.5\ht\pgfnodeparttextbox
    }

    \beforebackgroundpath{
        \begin{pgfscope}
            \pgfsetrectcap
            \pgfsetlinewidth{\pgfkeysvalueof{/tikz/circuits/line width}}
            %% draw channel, source and drain extensions
            % move to source
            \pgfpathmoveto{\pgfpointadd{\pgfpointorigin}{\pgfpoint{0cm}{\scale * (0.5 * \channellength + \drainlength)}}}
            % draw drain extension
            \pgfpathlineto{\pgfpointadd{\pgfpointorigin}{\pgfpoint{0cm}{\scale * 0.5 * \channellength}}}
            % draw to channel
            \pgfpathlineto{\pgfpointadd{\pgfpointorigin}{\pgfpoint{\scale * -\channelheight}{\scale * 0.5 * \channellength}}}
            % draw half of channel
            \pgfpathlineto{\pgfpointadd{\pgfpointorigin}{\pgfpoint{\scale * -\channelheight}{0cm}}}
            % draw other half of channel (starting from drain
            \pgfpathlineto{\pgfpointadd{\pgfpointorigin}{\pgfpoint{\scale * -\channelheight}{\scale * -0.5 * \channellength}}}
            % move to 'inner' source (without extension)
            \pgfpathlineto{\pgfpointadd{\pgfpointorigin}{\pgfpoint{0cm}{\scale * -0.5 * \channellength}}}
            % draw drain extension
            \pgfpathlineto{\pgfpointadd{\pgfpointorigin}{\pgfpoint{0cm}{\scale * (-0.5 * \channellength - \sourcelength)}}}
            \pgfusepath{stroke}
            %% draw gate
            \pgfsetbuttcap
            \pgfsetlinewidth{\pgfkeysvalueof{/tikz/circuits/mosfet/gate line width}}
            \pgfpathmoveto{\pgfpointadd{\pgfpointorigin}{\pgfpoint{\scale * (-\channelheight - \gateskip)}{\scale * 0.5 * \gatewidth + 0.5 * \pgfkeysvalueof{/tikz/circuits/line width}}}}
            \pgfpathlineto{\pgfpointadd{\pgfpointorigin}{\pgfpoint{\scale * (-\channelheight - \gateskip)}{\scale * -0.5 * \gatewidth - 0.5 * \pgfkeysvalueof{/tikz/circuits/line width}}}}
            \pgfusepath{stroke}
        \end{pgfscope}

        % draw bulk, source arrow and gate circle, depending on the display options
        % FIXME: this code is old and probably wrong
        \if@circuits@mosfet@draw@bulk
            \if@circuits@mosfet@draw@bulk@arrow % draw bulk (with arrow)
                \begin{pgfscope}
                    \pgfsetlinewidth{\pgfkeysvalueof{/tikz/circuits/mosfet/bulk line width}}
                    \if@circuits@mosfet@nmos
                        \pgfpathmoveto{\pgfpointorigin}
                        \pgfpathlineto{\pgfpointadd{\pgfpointorigin}{\pgfpoint{\scale * -\channelheight}{0cm}}}
                    \else
                        \pgfpathmoveto{\pgfpointadd{\pgfpointorigin}{\pgfpoint{\scale * -\channelheight}{0cm}}}
                        \pgfpathlineto{\pgfpointorigin}
                    \fi
                    \pgfsetarrows{-{Stealth[length=\pgfkeysvalueof{/tikz/circuits/mosfet/arrow length}, width=\pgfkeysvalueof{/tikz/circuits/mosfet/arrow width}]}}
                    \pgfusepath{stroke}
                \end{pgfscope}
            \else % draw bulk (without arrow)
                \begin{pgfscope}
                    \pgfsetlinewidth{\pgfkeysvalueof{/tikz/circuits/mosfet/bulk line width}}
                    \pgfpathmoveto{\pgfpointorigin}
                    \pgfpathlineto{\pgfpointadd{\pgfpointorigin}{\pgfpoint{\scale * -\channelheight}{0cm}}}
                    \pgfusepath{stroke}
                    % draw source arrow
                    \pgfsetlinewidth{0.8*\pgfkeysvalueof{/tikz/circuits/line width}}
                    \if@circuits@mosfet@nmos
                        \pgfpathmoveto{\pgfpointadd{\pgfpointorigin}{\pgfpoint{\scale * -\channelheight}{\scale * -0.5 * \channellength}}}
                        \pgfpathlineto{\pgfpointadd{\pgfpointorigin}{\pgfpoint{0cm}{\scale * -0.5 * \channellength}}}
                    \else
                        \pgfpathmoveto{\pgfpointadd{\pgfpointorigin}{\pgfpoint{0cm}{\scale * 0.5 * \channellength}}}
                        \pgfpathlineto{\pgfpointadd{\pgfpointorigin}{\pgfpoint{\scale * -\channelheight}{\scale * 0.5 * \channellength}}}
                    \fi
                    \pgfsetarrows{-{Stealth[length=\pgfkeysvalueof{/tikz/circuits/mosfet/arrow length}, width=\pgfkeysvalueof{/tikz/circuits/mosfet/arrow width}]}}
                    \pgfusepath{stroke}
                    \pgfsetarrowsend{}
                \end{pgfscope}
            \fi
        \fi

        % draw source arrow
        \if@circuits@mosfet@draw@source@arrow
            \begin{pgfscope}
                \pgfsetlinewidth{0.8*\pgfkeysvalueof{/tikz/circuits/line width}}
                \if@circuits@mosfet@nmos
                    \pgfpathmoveto{\pgfpointadd{\pgfpointorigin}{\pgfpoint{\scale * -\channelheight}{\scale * -0.5 * \channellength}}}
                    \pgfpathlineto{\pgfpointadd{\pgfpointorigin}{\pgfpoint{0cm}{\scale * -0.5 * \channellength}}}
                \else
                    \pgfpathmoveto{\pgfpointadd{\pgfpointorigin}{\pgfpoint{0cm}{\scale * 0.5 * \channellength}}}
                    \pgfpathlineto{\pgfpointadd{\pgfpointorigin}{\pgfpoint{\scale * -\channelheight}{\scale * 0.5 * \channellength}}}
                \fi
                \pgfsetarrows{-{Stealth[length=\pgfkeysvalueof{/tikz/circuits/mosfet/arrow length}, width=\pgfkeysvalueof{/tikz/circuits/mosfet/arrow width}]}}
                \pgfusepath{stroke}
                \pgfsetarrowsend{}
            \end{pgfscope}
        \fi

        % draw gate circle (pmos)
        \if@circuits@mosfet@draw@gate@circle
            \if@circuits@mosfet@nmos
                \relax % no circle in nmos
            \else
                \begin{pgfscope}
                    \pgfsetlinewidth{\pgfkeysvalueof{/tikz/circuits/mosfet/bulk line width}}
                    \pgfpathcircle{\pgfpointadd{\pgfpointorigin}{\pgfpoint{\scale * (-\channelheight - \gateskip) - \circleradius}{0cm}}}{\circleradius}
                    \if@circuits@mosfet@fill@gate@circle
                        \pgfusepath{stroke, fill}
                    \else
                        \pgfusepath{stroke}
                    \fi
                \end{pgfscope}
            \fi
        \fi
    }
}

\pgfdeclareshape{pmosshape}
{
    % saved anchors and macros
    \inheritsavedanchors[from=genericmosshape]

    % anchors
    \inheritanchor[from=genericmosshape]{text}
    \inheritanchor[from=genericmosshape]{center}
    \inheritanchor[from=genericmosshape]{north}
    \inheritanchor[from=genericmosshape]{south}
    \inheritanchor[from=genericmosshape]{west}
    \inheritanchor[from=genericmosshape]{east}
    \inheritanchor[from=genericmosshape]{north west}
    \inheritanchor[from=genericmosshape]{north east}
    \inheritanchor[from=genericmosshape]{south west}
    \inheritanchor[from=genericmosshape]{south east}

    % terminals
    \inheritanchor[from=genericmosshape]{gate}
    \inheritanchor[from=genericmosshape]{bulk}
    \anchor{drain}{\pgfpointadd{\pgfpointorigin}{\pgfpoint{0cm}{-\scale * (0.5 * \channellength + \drainlength)}}}
    \anchor{source}{\pgfpointadd{\pgfpointorigin}{\pgfpoint{0cm}{\scale * (0.5 * \channellength + \sourcelength)}}}

    % path
    \inheritbeforebackgroundpath[from=genericmosshape]
}

\pgfdeclareshape{nmosshape}
{
    % saved anchors and macros
    \inheritsavedanchors[from=genericmosshape]

    % anchors
    \inheritanchor[from=genericmosshape]{text}
    \inheritanchor[from=genericmosshape]{center}
    \inheritanchor[from=genericmosshape]{north}
    \inheritanchor[from=genericmosshape]{south}
    \inheritanchor[from=genericmosshape]{west}
    \inheritanchor[from=genericmosshape]{east}
    \inheritanchor[from=genericmosshape]{north west}
    \inheritanchor[from=genericmosshape]{north east}
    \inheritanchor[from=genericmosshape]{south west}
    \inheritanchor[from=genericmosshape]{south east}

    % terminals
    \inheritanchor[from=genericmosshape]{gate}
    \inheritanchor[from=genericmosshape]{bulk}
    \inheritanchor[from=genericmosshape]{source}
    \inheritanchor[from=genericmosshape]{source}
    \anchor{drain}{\pgfpointadd{\pgfpointorigin}{\pgfpoint{0cm}{\scale * (0.5 * \channellength + \drainlength)}}}
    \anchor{source}{\pgfpointadd{\pgfpointorigin}{\pgfpoint{0cm}{-\scale * (0.5 * \channellength + \sourcelength)}}}

    % path
    \inheritbeforebackgroundpath[from=genericmosshape]
}

% vim: ft=plaintex nowrap

\tikzset{
    circuits/currentsource/radius/.initial = 0.25cm,
    circuits/currentsource/extension factor/.initial = 1.0,
    circuits/currentsource/terminal extension/.initial = 0cm,
    circuits/currentsource/arrow length/.initial={1.75mm},
    circuits/currentsource/arrow width/.initial={1.75mm},
    currentsource/.style = {currentsourceshape}
}

\pgfdeclareshape{currentsourceshape}
{
    \saveddimen{\radius}{\pgf@x=\pgfkeysvalueof{/tikz/circuits/currentsource/radius}}
    \saveddimen{\extension}{\pgf@x=\pgfkeysvalueof{/tikz/circuits/currentsource/terminal extension}}
    \savedmacro{\extensionfactor}{\edef\extensionfactor{\pgfkeysvalueof{/tikz/circuits/currentsource/extension factor}}}

    % general anchors
    \anchor{center}{\pgfpointorigin}
    \anchor{west}{\pgfpointadd{\pgfpointorigin}{\pgfpoint{-\radius}{0cm}}}
    \anchor{east}{\pgfpointadd{\pgfpointorigin}{\pgfpoint{\radius}{0cm}}}
    \anchor{outer north}{\pgfpointadd{\pgfpointorigin}{\pgfpoint{0cm}{ \radius * (1 + \extensionfactor)}}}
    \anchor{outer south}{\pgfpointadd{\pgfpointorigin}{\pgfpoint{0cm}{-\radius * (1 + \extensionfactor)}}}
    \anchor{north}{\pgfpointadd{\pgfpointorigin}{\pgfpoint{0cm}{ \radius + \extension}}}
    \anchor{south}{\pgfpointadd{\pgfpointorigin}{\pgfpoint{0cm}{-\radius - \extension}}}
    \anchor{south east}{\pgfpointadd{\pgfpointorigin}{\pgfpoint{ \radius + \extension}{-\radius}}}
    \anchor{south west}{\pgfpointadd{\pgfpointorigin}{\pgfpoint{-\radius - \extension}{-\radius}}}
    \anchor{north east}{\pgfpointadd{\pgfpointorigin}{\pgfpoint{ \radius + \extension}{ \radius}}}
    \anchor{north west}{\pgfpointadd{\pgfpointorigin}{\pgfpoint{-\radius - \extension}{ \radius}}}
    \beforebackgroundpath{
        \pgfsetlinewidth{\pgfkeysvalueof{/tikz/circuits/line width}}
        \pgfpathcircle{\pgfpointorigin}{\radius}
        \pgfusepath{stroke}
        % draw terminals
        \pgfpathmoveto{\pgfpointadd{\pgfpointorigin}{\pgfpoint{0cm}{-\radius - \extension}}}
        \pgfpathlineto{\pgfpointadd{\pgfpointorigin}{\pgfpoint{0cm}{-\radius}}}
        \pgfpathmoveto{\pgfpointadd{\pgfpointorigin}{\pgfpoint{0cm}{ \radius + \extension}}}
        \pgfpathlineto{\pgfpointadd{\pgfpointorigin}{\pgfpoint{0cm}{ \radius}}}
        \pgfusepath{stroke}
        % draw arrow
        \pgfpathmoveto{\pgfpointadd{\pgfpointorigin}{\pgfpoint{0cm}{0.7 * \radius}}}
        \pgfpathlineto{\pgfpointadd{\pgfpointorigin}{\pgfpoint{0cm}{-0.7 * \radius}}}
        \pgfsetarrows{-{Stealth[length=\pgfkeysvalueof{/tikz/circuits/currentsource/arrow length}, width=\pgfkeysvalueof{/tikz/circuits/currentsource/arrow width}]}}
        \pgfusepath{stroke}
    }
}

% vim: ft=plaintex

\tikzset{
    circuits/ground/lines/.initial = 3,
    circuits/ground/width/.initial = 0.15cm,
    circuits/ground/height/.initial = 0.075cm,
    circuits/ground/last width factor/.initial = 0.4,
    ground/.style = {groundshape}
}

\pgfdeclareshape{groundshape}
{
    \savedmacro{\lines}{\edef\lines{\pgfkeysvalueof{/tikz/circuits/ground/lines}}}
    \savedmacro{\lastwidthfactor}{\edef\lastwidthfactor{\pgfkeysvalueof{/tikz/circuits/ground/last width factor}}}
    \saveddimen{\width}{\pgf@x=\pgfkeysvalueof{/tikz/circuits/ground/width}}
    \saveddimen{\height}{\pgf@x=\pgfkeysvalueof{/tikz/circuits/ground/height}}
    % general anchors
    \anchor{center}{\pgfpointadd{\pgfpointorigin}{\pgfpoint{0cm}{ 0.5 * \height}}}
    \anchor{north}{\pgfpointadd{\pgfpointorigin}{\pgfpoint{0cm}{ 0.5 * \height}}}
    \anchor{south}{\pgfpointadd{\pgfpointorigin}{\pgfpoint{0cm}{ 0.5 * \height}}}
    \anchor{west}{\pgfpointadd{\pgfpointorigin}{\pgfpoint{0cm}{ 0.5 * \height}}}
    \anchor{east}{\pgfpointadd{\pgfpointorigin}{\pgfpoint{0cm}{ 0.5 * \height}}}
    \anchor{north west}{\pgfpointadd{\pgfpointorigin}{\pgfpoint{0cm}{ 0.5 * \height}}}
    \anchor{north east}{\pgfpointadd{\pgfpointorigin}{\pgfpoint{0cm}{ 0.5 * \height}}}
    \anchor{south west}{\pgfpointadd{\pgfpointorigin}{\pgfpoint{0cm}{ 0.5 * \height}}}
    \anchor{south east}{\pgfpointadd{\pgfpointorigin}{\pgfpoint{0cm}{ 0.5 * \height}}}
    \beforebackgroundpath{
        \pgfsetlinewidth{\pgfkeysvalueof{/tikz/circuits/line width}}
        \foreach \y in {1,2,...,\lines}
        {
            \pgfpathmoveto{\pgfpointadd{\pgfpointorigin}{\pgfpoint{ (1 - (1 - \lastwidthfactor) / (\lines - 1) * (\y - 1)) * \width}{0.5 * \height - \height / (\lines - 1) * (\y - 1)}}}
            \pgfpathlineto{\pgfpointadd{\pgfpointorigin}{\pgfpoint{-(1 - (1 - \lastwidthfactor) / (\lines - 1) * (\y - 1)) * \width}{0.5 * \height - \height / (\lines - 1) * (\y - 1)}}}
        }
        \pgfusepath{stroke}
    }
}

% vim: ft=plaintex nowrap

\tikzset{
    circuits/vdd/width/.initial = 0.25cm,
    vdd/.style = {vddshape}
}

\pgfdeclareshape{vddshape}
{
    \saveddimen{\width}{\pgf@x=\pgfkeysvalueof{/tikz/circuits/vdd/width}}
    % general anchors
    \anchor{center}{\pgfpointorigin}
    \anchor{west}{\pgfpointadd{\pgfpointorigin}{\pgfpoint{-0.5 * \width}{0cm}}}
    \anchor{east}{\pgfpointadd{\pgfpointorigin}{\pgfpoint{ 0.5 * \width}{0cm}}}
    \anchor{north}{\pgfpointorigin}
    \anchor{south}{\pgfpointorigin}
    \beforebackgroundpath{
        \pgfsetlinewidth{\pgfkeysvalueof{/tikz/circuits/line width}}
            \pgfpathmoveto{\pgfpointadd{\pgfpointorigin}{\pgfpoint{-0.5 * \width}{0cm}}}
            \pgfpathlineto{\pgfpointadd{\pgfpointorigin}{\pgfpoint{0.5 * \width}{0cm}}}
        \pgfusepath{stroke}
    }
}

% vim: ft=plaintex nowrap

\tikzset{
    circuits/impedance/width/.initial = 0.6cm,
    circuits/impedance/height/.initial = 0.25cm,
    circuits/impedance/segments/.initial = 3,
    circuits/impedance/terminal extension/.initial = 0.05cm,
    impedance/.style = {impedanceshape}
}

\pgfdeclareshape{impedanceshape}
{
    \saveddimen{\width}{\pgf@x=\pgfkeysvalueof{/tikz/circuits/impedance/width}}
    \saveddimen{\height}{\pgf@x=\pgfkeysvalueof{/tikz/circuits/impedance/height}}
    \savedanchor{\lowerleft}{
        \pgfpointadd{\pgfpointorigin}{\pgfpoint{-0.5 * \pgfkeysvalueof{/tikz/circuits/impedance/width}}{-0.5 * \pgfkeysvalueof{/tikz/circuits/impedance/height}}}
    }
    \savedanchor{\upperright}{
        \pgfpointadd{\pgfpointorigin}{\pgfpoint{0.5 * \pgfkeysvalueof{/tikz/circuits/impedance/width}}{0.5 * \pgfkeysvalueof{/tikz/circuits/impedance/height}}}
    }

    % electrical terminals
    \anchor{plus}{\pgfpointadd{\pgfpointorigin}{\pgfpoint{-0.5 * \width}{0cm}}}
    \anchor{minus}{\pgfpointadd{\pgfpointorigin}{\pgfpoint{0.5 * \width}{0cm}}}
    % general anchors
    \anchor{lowerleft}{\lowerleft}
    \anchor{upperright}{\upperright}
    \anchor{center}{\pgfpointorigin}
    \anchor{west}{\pgfpointadd{\pgfpointorigin}{\pgfpoint{-0.5 * \width}{0cm}}}
    \anchor{east}{\pgfpointadd{\pgfpointorigin}{\pgfpoint{0.5 * \width}{0cm}}}
    \anchor{north}{\pgfpointadd{\pgfpointorigin}{\pgfpoint{0cm}{0.5 * \height}}}
    \anchor{south}{\pgfpointadd{\pgfpointorigin}{\pgfpoint{0cm}{-0.5 * \height}}}
    \anchor{south east}{\pgfpointadd{\pgfpointorigin}{\pgfpoint{0.5 * \height}{-0.5 * \width}}}
    \anchor{south west}{\pgfpointadd{\pgfpointorigin}{\pgfpoint{-0.5 * \height}{-0.5 * \width}}}
    \anchor{north east}{\pgfpointadd{\pgfpointorigin}{\pgfpoint{0.5 * \height}{0.5 * \width}}}
    \anchor{north west}{\pgfpointadd{\pgfpointorigin}{\pgfpoint{-0.5 * \height}{0.5 * \width}}}
    \beforebackgroundpath{
        \pgfsetlinewidth{\pgfkeysvalueof{/tikz/circuits/line width}}
        \pgfpathrectanglecorners{\lowerleft}{\upperright}
        \pgfusepath{fill}
    }
}

% vim: ft=plaintex

\tikzset{
    circuits/amplifier/width/.initial = 1.20cm,
    circuits/amplifier/height/.initial = 1.20cm,
    amplifier/.style = {amplifiershape, draw, wire}
}

\pgfdeclareshape{amplifiershape}
{
    \saveddimen{\width}{\pgf@x=\pgfkeysvalueof{/tikz/circuits/amplifier/width}}
    \saveddimen{\height}{\pgf@x=\pgfkeysvalueof{/tikz/circuits/amplifier/height}}
    \savedanchor{\centerpoint}{\pgfpointorigin}
    \savedanchor{\inputport}{%
        \pgfpointadd%
        {\pgfpointorigin}%
        {\pgfpoint%
            {-0.5 * \pgfkeysvalueof{/tikz/circuits/amplifier/width}}%
            {0cm}
        }%
    }
    \savedanchor{\output}{%
        \pgfpointadd%
        {\pgfpointorigin}%
        {\pgfpoint%
            {0.5 * \pgfkeysvalueof{/tikz/circuits/amplifier/width}}%
            {0cm}
        }%
    }
    % electrical terminals (anchors)
    \anchor{in}{\pgfpointadd{\centerpoint}{\pgfpoint{-0.5 * \width}{0cm}}}
    \anchor{out}{\output}
    \anchor{+power}{
        \pgfpointlineattime%
        {0.5}%
        {\pgfpointadd{\pgfpointorigin}{\pgfpoint{-0.5 * \width}{0.5 * \height}}}
        {\output}%
    }
    \anchor{-power}{
        \pgfpointlineattime%
        {0.5}%
        {\pgfpointadd{\pgfpointorigin}{\pgfpoint{-0.5 * \width}{-0.5 * \height}}}
        {\output}%
    }
    % regular anchors
    \anchor{center}{\centerpoint}
    \anchor{text} % this is used to center the text in the node
    {
        %\pgfpoint{-.5\wd\pgfnodeparttextbox}{-.5\ht\pgfnodeparttextbox}
        % the text node is shifted more than its width, a way of optical compensation (since the amplifier gets more narrow to the right)
        \pgfpoint{-0.75\wd\pgfnodeparttextbox}{-0.5\ht\pgfnodeparttextbox}
    }
    \anchor{north}{
        \pgfpointadd{\pgfpointorigin}{\pgfpoint{0cm}{0.5 * \height}}
    }
    \anchor{south}{
        \pgfpointadd{\pgfpointorigin}{\pgfpoint{0cm}{-0.5 * \height}}
    }
    \anchor{west}{
        \pgfpointadd{\pgfpointorigin}{\pgfpoint{-0.5 * \width}{0cm}}
    }
    \anchor{east}{
        \pgfpointadd{\pgfpointorigin}{\pgfpoint{0.5 * \width}{0cm}}
    }
    \anchor{north east}{
        \pgfpointadd{\pgfpointorigin}{\pgfpoint{0.5 * \width}{0.5 * \height}}
    }
    \anchor{north west}{
        \pgfpointadd{\pgfpointorigin}{\pgfpoint{-0.5 * \width}{0.5 * \height}}
    }
    \anchor{south east}{
        \pgfpointadd{\pgfpointorigin}{\pgfpoint{0.5 * \width}{-0.5 * \height}}
    }
    \anchor{south west}{
        \pgfpointadd{\pgfpointorigin}{\pgfpoint{-0.5 * \width}{-0.5 * \height}}
    }
    \backgroundpath{
        \pgfpathmoveto{\output}
        \pgfpathlineto{\pgfpointadd{\output}{\pgfpoint{-\width}{-0.5 * \height}}}
        \pgfpathlineto{\pgfpointadd{\output}{\pgfpoint{-\width}{0.5 * \height}}}
        \pgfpathclose
    }
}

% vim: ft=plaintex nowrap

\tikzset{
    circuits/transmission gate/width/.initial = 1cm,
    circuits/transmission gate/height/.initial = 1cm,
    circuits/transmission gate/circle size/.initial = 3pt,
    transmission gate/.style = {tgateshape},
    tgate/.style = {tgateshape}
}

\pgfdeclareshape{tgateshape}
{
    \saveddimen{\width}{\pgf@x=\pgfkeysvalueof{/tikz/circuits/transmission gate/width}}
    \saveddimen{\height}{\pgf@x=\pgfkeysvalueof{/tikz/circuits/transmission gate/height}}
    \saveddimen{\circlesize}{\pgf@x=\pgfkeysvalueof{/tikz/circuits/transmission gate/circle size}}
    \savedanchor{\centerpoint}{\pgfpointorigin}
    \savedanchor{\inputport}{%
        \pgfpointadd%
        {\pgfpointorigin}%
        {\pgfpoint%
            {-0.5 * \pgfkeysvalueof{/tikz/circuits/transmission gate/width}}%
            {0cm}
        }%
    }
    \savedanchor{\output}{%
        \pgfpointadd%
        {\pgfpointorigin}%
        {\pgfpoint%
            {0.5 * \pgfkeysvalueof{/tikz/circuits/transmission gate/width}}%
            {0cm}
        }%
    }
    % electrical terminals (anchors)
    \anchor{in}{\pgfpointadd{\centerpoint}{\pgfpoint{-0.5 * \width}{0cm}}}
    \anchor{out}{\output}
    \anchor{clkp}{
        \pgfpointadd{%
            \pgfpointlineattime%
            {0.5}%
            {\pgfpointadd{\pgfpointorigin}{\pgfpoint{-0.5 * \width}{0.5 * \height}}}
            {\output}%
        }{\pgfpoint{0pt}{2 * \circlesize + 0.5pt}}%
    }
    \anchor{clkn}{
        \pgfpointadd{%
            \pgfpointlineattime%
            {0.5}%
            {\pgfpointadd{\pgfpointorigin}{\pgfpoint{-0.5 * \width}{-0.5 * \height}}}
            {\output}%
        }{\pgfpoint{0pt}{0pt}}%
    }
    % regular anchors
    \anchor{center}{\centerpoint}
    \anchor{text} % this is used to center the text in the node
    {
        %\pgfpoint{-.5\wd\pgfnodeparttextbox}{-.5\ht\pgfnodeparttextbox}
        % the text node is shifted more than its width, a way of optical compensation (since the transmission gate gets more narrow to the right)
        \pgfpoint{-0.75\wd\pgfnodeparttextbox}{-0.5\ht\pgfnodeparttextbox}
    }
    \anchor{north}{
        \pgfpointadd{\pgfpointorigin}{\pgfpoint{0cm}{0.5 * \height}}
    }
    \anchor{south}{
        \pgfpointadd{\pgfpointorigin}{\pgfpoint{0cm}{-0.5 * \height}}
    }
    \anchor{west}{
        \pgfpointadd{\pgfpointorigin}{\pgfpoint{-0.5 * \width}{0cm}}
    }
    \anchor{east}{
        \pgfpointadd{\pgfpointorigin}{\pgfpoint{0.5 * \width}{0cm}}
    }
    \anchor{north east}{
        \pgfpointadd{\pgfpointorigin}{\pgfpoint{0.5 * \width}{0.5 * \height}}
    }
    \anchor{north west}{
        \pgfpointadd{\pgfpointorigin}{\pgfpoint{-0.5 * \width}{0.5 * \height}}
    }
    \anchor{south east}{
        \pgfpointadd{\pgfpointorigin}{\pgfpoint{0.5 * \width}{-0.5 * \height}}
    }
    \anchor{south west}{
        \pgfpointadd{\pgfpointorigin}{\pgfpoint{-0.5 * \width}{-0.5 * \height}}
    }
    \beforebackgroundpath{
        \pgfsetlinewidth{\pgfkeysvalueof{/tikz/circuits/line width}}
        \pgfpathmoveto{\output}
        \pgfpathlineto{\pgfpointadd{\output}{\pgfpoint{-\width}{-0.5 * \height}}}
        \pgfpathlineto{\pgfpointadd{\output}{\pgfpoint{-\width}{0.5 * \height}}}
        \pgfpathclose
        \pgfusepath{stroke}
        \pgfpathmoveto{\pgfpointadd{\output}{\pgfpoint{0pt}{0.5 * \height}}}
        \pgfpathlineto{\pgfpointadd{\output}{\pgfpoint{0pt}{-0.5 * \height}}}
        \pgfpathlineto{\pgfpointadd{\output}{\pgfpoint{-\width}{0pt}}}
        \pgfpathclose
        \pgfusepath{stroke}
        % draw input inversion
        \pgfpathcircle%
        {
            \pgfpointadd{%
                \pgfpointlineattime%
                {0.5}%
                {\pgfpointadd{\pgfpointorigin}{\pgfpoint{-0.5 * \width}{0.5 * \height}}}
                {\output}%
            }{\pgfpoint{0pt}{\circlesize + 0.5pt}}%
        }
        {\circlesize}
        \pgfusepath{stroke}
    }
}

% vim: ft=plaintex nowrap

\tikzset{
    circuits/mixer/radius/.initial = 0.35cm,
    mixer/.style = {mixershape}
}

\pgfdeclareshape{mixershape}
{
    \saveddimen{\radius}{\pgf@x=\pgfkeysvalueof{/tikz/circuits/mixer/radius}}

    % electrical anchors
    \anchor{left}{\pgfpointadd{\pgfpointorigin}{\pgfpoint{-\radius}{0cm}}}
    \anchor{right}{\pgfpointadd{\pgfpointorigin}{\pgfpoint{\radius}{0cm}}}
    \anchor{top}{\pgfpointadd{\pgfpointorigin}{\pgfpoint{0cm}{ \radius}}}
    \anchor{bottom}{\pgfpointadd{\pgfpointorigin}{\pgfpoint{0cm}{-\radius}}}
    % general anchors
    \anchor{center}{\pgfpointorigin}
    \anchor{west}{\pgfpointadd{\pgfpointorigin}{\pgfpoint{-\radius}{0cm}}}
    \anchor{east}{\pgfpointadd{\pgfpointorigin}{\pgfpoint{\radius}{0cm}}}
    \anchor{north}{\pgfpointadd{\pgfpointorigin}{\pgfpoint{0cm}{ \radius}}}
    \anchor{south}{\pgfpointadd{\pgfpointorigin}{\pgfpoint{0cm}{-\radius}}}
    \anchor{south east}{\pgfpointadd{\pgfpointorigin}{\pgfpoint{ \radius}{-\radius}}}
    \anchor{south west}{\pgfpointadd{\pgfpointorigin}{\pgfpoint{-\radius}{-\radius}}}
    \anchor{north east}{\pgfpointadd{\pgfpointorigin}{\pgfpoint{ \radius}{ \radius}}}
    \anchor{north west}{\pgfpointadd{\pgfpointorigin}{\pgfpoint{-\radius}{ \radius}}}
    \beforebackgroundpath{
        \pgfsetlinewidth{\pgfkeysvalueof{/tikz/circuits/line width}}
        \pgfpathcircle{\pgfpointorigin}{\radius}
        \pgfusepath{stroke}
        % draw cross
        \pgfpathmoveto{\pgfpointadd{\pgfpointorigin}{\pgfpoint{-0.707107 * \radius}{-0.707107 * \radius}}}
        \pgfpathlineto{\pgfpointadd{\pgfpointorigin}{\pgfpoint{ 0.707107 * \radius}{ 0.707107 * \radius}}}
        \pgfpathmoveto{\pgfpointadd{\pgfpointorigin}{\pgfpoint{-0.707107 * \radius}{ 0.707107 * \radius}}}
        \pgfpathlineto{\pgfpointadd{\pgfpointorigin}{\pgfpoint{ 0.707107 * \radius}{-0.707107 * \radius}}}
        \pgfusepath{stroke}
    }
}

% vim: ft=plaintex

\tikzset{
    circuits/oscillator/radius/.initial = 0.4cm,
    circuits/oscillator/sine width factor/.initial = 0.7,
    circuits/oscillator/sine height factor/.initial = 0.5,
    oscillator/.style = {oscillatorshape}
}

\pgfdeclareshape{oscillatorshape}
{
    \saveddimen{\radius}{\pgf@x=\pgfkeysvalueof{/tikz/circuits/oscillator/radius}}
    \savedmacro{\sinewidthfactor}{\edef\sinewidthfactor{\pgfkeysvalueof{/tikz/circuits/oscillator/sine width factor}}}
    \savedmacro{\sineheightfactor}{\edef\sineheightfactor{\pgfkeysvalueof{/tikz/circuits/oscillator/sine height factor}}}

    % general anchors
    \anchor{center}{\pgfpointorigin}
    \anchor{west}{\pgfpointadd{\pgfpointorigin}{\pgfpoint{-\radius}{0cm}}}
    \anchor{east}{\pgfpointadd{\pgfpointorigin}{\pgfpoint{\radius}{0cm}}}
    \anchor{north}{\pgfpointadd{\pgfpointorigin}{\pgfpoint{0cm}{ \radius}}}
    \anchor{south}{\pgfpointadd{\pgfpointorigin}{\pgfpoint{0cm}{-\radius}}}
    \anchor{south east}{\pgfpointadd{\pgfpointorigin}{\pgfpoint{ \radius}{-\radius}}}
    \anchor{south west}{\pgfpointadd{\pgfpointorigin}{\pgfpoint{-\radius}{-\radius}}}
    \anchor{north east}{\pgfpointadd{\pgfpointorigin}{\pgfpoint{ \radius}{ \radius}}}
    \anchor{north west}{\pgfpointadd{\pgfpointorigin}{\pgfpoint{-\radius}{ \radius}}}
    \beforebackgroundpath{
        \pgfsetlinewidth{\pgfkeysvalueof{/tikz/circuits/line width}}
        \pgfpathcircle{\pgfpointorigin}{\radius}
        \pgfusepath{stroke}
        % draw sine
        \pgfpathmoveto{\pgfpointadd{\pgfpointorigin}{\pgfpoint{-1 * \sinewidthfactor * \radius}{0.0 * \radius}}}
        \pgfpathsine{\pgfpointadd{\pgfpointorigin}{\pgfpoint{0.5 * \sinewidthfactor * \radius}{\sineheightfactor * \radius}}}
        \pgfpathcosine{\pgfpointadd{\pgfpointorigin}{\pgfpoint{0.5 * \sinewidthfactor * \radius}{-\sineheightfactor * \radius}}}
        \pgfpathsine{\pgfpointadd{\pgfpointorigin}{\pgfpoint{0.5 * \sinewidthfactor * \radius}{-\sineheightfactor * \radius}}}
        \pgfpathcosine{\pgfpointadd{\pgfpointorigin}{\pgfpoint{0.5 * \sinewidthfactor * \radius}{\sineheightfactor * \radius}}}
        \pgfusepath{stroke}
    }
}

% vim: ft=plaintex

\newif\ifantennadrawthrough
\tikzset{
    circuits/antenna/width/.initial = 0.35cm,
    circuits/antenna/head height/.initial = 0.3cm,
    circuits/antenna/foot height/.initial = 0.3cm,
    circuits/antenna/draw through/.is if=antennadrawthrough,
    antenna/.style = {antennashape, circuits/antenna/draw through = false}
}

\pgfdeclareshape{antennashape}
{
    \saveddimen{\width}{\pgf@x=\pgfkeysvalueof{/tikz/circuits/antenna/width}}
    \saveddimen{\headheight}{\pgf@x=\pgfkeysvalueof{/tikz/circuits/antenna/head height}}
    \saveddimen{\footheight}{\pgf@x=\pgfkeysvalueof{/tikz/circuits/antenna/foot height}}
    \anchor{port}{
        \pgfpointadd{\pgfpointorigin}{\pgfpoint{0cm}{-\footheight}}
    }
    \anchor{center}{\pgfpointorigin}
    \anchor{head center}{
        \pgfpointadd{\pgfpointorigin}{\pgfpoint{0cm}{0.5 * \headheight}}
    }
    \anchor{north}{
        \pgfpointadd{\pgfpointorigin}{\pgfpoint{0cm}{\headheight}}
    }
    \anchor{south}{
        \pgfpointadd{\pgfpointorigin}{\pgfpoint{0cm}{-\footheight}}
    }
    \anchor{south east}{
        \pgfpointadd{\pgfpointorigin}{\pgfpoint{0.5 * \width}{-\footheight}}
    }
    \anchor{south west}{
        \pgfpointadd{\pgfpointorigin}{\pgfpoint{-0.5 * \width}{-\footheight}}
    }
    \anchor{north east}{
        \pgfpointadd{\pgfpointorigin}{\pgfpoint{0.5 * \width}{\headheight}}
    }
    \anchor{north west}{
        \pgfpointadd{\pgfpointorigin}{\pgfpoint{-0.5 * \width}{\headheight}}
    }
    \anchor{west}{
        \pgfpointadd{\pgfpointorigin}{\pgfpoint{-0.5 * \width}{0cm}}
    }
    \anchor{east}{
        \pgfpointadd{\pgfpointorigin}{\pgfpoint{0.5 * \width}{0cm}}
    }
    \beforebackgroundpath{
        \pgfsetlinewidth{\pgfkeysvalueof{/tikz/circuits/line width}}
        \pgfpathmoveto{\pgfpointadd{\pgfpointorigin}{\pgfpoint{0cm}{-\footheight}}}
        \ifantennadrawthrough
            \pgfpathmoveto{\pgfpointadd{\pgfpointorigin}{\pgfpoint{0cm}{\headheight}}}
        \else
            \pgfpathlineto{\pgfpointorigin}
        \fi
        \pgfusepath{stroke}
        \pgfpathmoveto{\pgfpointorigin}
        \pgfpathlineto{\pgfpointadd{\pgfpointorigin}{\pgfpoint{0.5 * \width}{\headheight}}}
        \pgfpathlineto{\pgfpointadd{\pgfpointorigin}{\pgfpoint{-0.5 * \width}{\headheight}}}
        \pgfpathclose
        \pgfusepath{stroke}
    }
}

% vim: ft=plaintex nowrap

\newif\if@circuits@diode@lightemitting
\newif\if@circuits@diode@lightreceiving
\tikzset{
    circuits/diode/width/.initial = 0.35cm,
    circuits/diode/head height/.initial = 0.3cm,
    circuits/diode/terminal extension/.initial = 5pt,
    circuits/diode/light emitting/.is if=@circuits@diode@lightemitting,
    circuits/diode/light receiving/.is if=@circuits@diode@lightreceiving,
    diode/.style = {diodeshape}
}

\pgfdeclareshape{diodeshape}
{
    \saveddimen{\width}{\pgf@x=\pgfkeysvalueof{/tikz/circuits/diode/width}}
    \saveddimen{\headheight}{\pgf@x=\pgfkeysvalueof{/tikz/circuits/diode/head height}}
    \saveddimen{\extension}{\pgf@x=\pgfkeysvalueof{/tikz/circuits/diode/terminal extension}}
    \anchor{port}{
        \pgfpointadd{\pgfpointorigin}{\pgfpoint{0cm}{-\footheight}}
    }
    \anchor{center}{\pgfpointorigin}
    \anchor{head center}{
        \pgfpointadd{\pgfpointorigin}{\pgfpoint{0cm}{0.5 * \headheight}}
    }
    \anchor{north}{
        \pgfpointadd{\pgfpointorigin}{\pgfpoint{0cm}{\headheight}}
    }
    \anchor{south}{
        \pgfpointadd{\pgfpointorigin}{\pgfpoint{0cm}{-\footheight}}
    }
    \anchor{south east}{
        \pgfpointadd{\pgfpointorigin}{\pgfpoint{0.5 * \width}{-\footheight}}
    }
    \anchor{south west}{
        \pgfpointadd{\pgfpointorigin}{\pgfpoint{-0.5 * \width}{-\footheight}}
    }
    \anchor{north east}{
        \pgfpointadd{\pgfpointorigin}{\pgfpoint{0.5 * \width}{\headheight}}
    }
    \anchor{north west}{
        \pgfpointadd{\pgfpointorigin}{\pgfpoint{-0.5 * \width}{\headheight}}
    }
    \anchor{west}{
        \pgfpointadd{\pgfpointorigin}{\pgfpoint{-0.5 * \width}{0cm}}
    }
    \anchor{east}{
        \pgfpointadd{\pgfpointorigin}{\pgfpoint{0.5 * \width}{0cm}}
    }
    \beforebackgroundpath{
        % head triangle
        \pgfsetlinewidth{\pgfkeysvalueof{/tikz/circuits/line width}}
        \pgfpathmoveto{\pgfpointadd{\pgfpointorigin}{\pgfpoint{0pt}{-0.5 * \headheight}}}
        \pgfpathlineto{\pgfpointadd{\pgfpointorigin}{\pgfpoint{ 0.5 * \width}{0.5 * \headheight}}}
        \pgfpathlineto{\pgfpointadd{\pgfpointorigin}{\pgfpoint{-0.5 * \width}{0.5 * \headheight}}}
        \pgfpathclose
        \pgfusepath{stroke}
        % head line
        \pgfpathmoveto{\pgfpointadd{\pgfpointorigin}{\pgfpoint{-0.5 * \width}{-0.5 * \headheight -0.5pt}}}
        \pgfpathlineto{\pgfpointadd{\pgfpointorigin}{\pgfpoint{ 0.5 * \width}{-0.5 * \headheight -0.5pt}}}
        \pgfpathclose
        \pgfusepath{stroke}
        % draw extension
        \pgfpathmoveto{\pgfpointadd{\pgfpointorigin}{\pgfpoint{0pt}{-0.5 * \headheight}}}
        \pgfpathlineto{\pgfpointadd{\pgfpointorigin}{\pgfpoint{0pt}{-0.5 * \headheight - \extension}}}
        \pgfpathmoveto{\pgfpointadd{\pgfpointorigin}{\pgfpoint{0pt}{ 0.5 * \headheight}}}
        \pgfpathlineto{\pgfpointadd{\pgfpointorigin}{\pgfpoint{0pt}{ 0.5 * \headheight + \extension}}}
        \pgfpathclose
        \pgfusepath{stroke}
        \if@circuits@diode@lightemitting
            \pgfsetarrows{-{Stealth[length=2pt, width=2pt]}}
            \pgfpathmoveto{\pgfpointadd{\pgfpointorigin}{\pgfpoint{0.6 * \width}{0.15 * \headheight}}}
            \pgfpathlineto{\pgfpointadd{\pgfpointorigin}{\pgfpoint{0.6 * \width + 5pt}{0.15 * \headheight + 3pt}}}
            \pgfusepath{stroke}
            \pgfpathmoveto{\pgfpointadd{\pgfpointorigin}{\pgfpoint{0.6 * \width}{0.15 * \headheight + 2pt}}}
            \pgfpathlineto{\pgfpointadd{\pgfpointorigin}{\pgfpoint{0.6 * \width + 5pt}{0.15 * \headheight + 2pt + 3pt}}}
            \pgfusepath{stroke}
            \pgfsetarrowsend{}
        \fi
        \if@circuits@diode@lightreceiving
            \pgfsetarrows{-{Stealth[length=2pt, width=2pt]}}
            \pgfpathmoveto{\pgfpointadd{\pgfpointorigin}{\pgfpoint{0.6 * \width + 5pt}{0.15 * \headheight + 3pt}}}
            \pgfpathlineto{\pgfpointadd{\pgfpointorigin}{\pgfpoint{0.6 * \width}{0.15 * \headheight}}}
            \pgfusepath{stroke}
            \pgfpathmoveto{\pgfpointadd{\pgfpointorigin}{\pgfpoint{0.6 * \width + 5pt}{0.15 * \headheight + 2pt + 3pt}}}
            \pgfpathlineto{\pgfpointadd{\pgfpointorigin}{\pgfpoint{0.6 * \width}{0.15 * \headheight + 2pt}}}
            \pgfusepath{stroke}
            \pgfsetarrowsend{}
        \fi
    }
}

% vim: ft=plaintex nowrap

\newif\ifgyratordrawarrow

\tikzset{
    circuits/gyrator/inner width/.initial = 1.25cm,
    circuits/gyrator/outer width/.initial = 0.75cm,
    circuits/gyrator/height/.initial = 2cm,
    circuits/gyrator/radius/.initial = 1.00cm,
    circuits/gyrator/draw arrow/.is if=gyratordrawarrow,
    gyrator/.style = {gyratorshape}
}

\pgfdeclareshape{gyratorshape}
{
    \saveddimen{\innerwidth}{\pgf@x=\pgfkeysvalueof{/tikz/circuits/gyrator/inner width}}
    \saveddimen{\outerwidth}{\pgf@x=\pgfkeysvalueof{/tikz/circuits/gyrator/outer width}}
    \saveddimen{\height}{\pgf@x=\pgfkeysvalueof{/tikz/circuits/gyrator/height}}
    \saveddimen{\radius}{\pgf@x=\pgfkeysvalueof{/tikz/circuits/gyrator/radius}}
    \anchor{center}{\pgfpointorigin}
    \anchor{north}{\pgfpointadd{\pgfpointorigin}{\pgfpoint{0cm}{ 0.5 * \height}}}
    \anchor{south}{\pgfpointadd{\pgfpointorigin}{\pgfpoint{0cm}{-0.5 * \height}}}
    \anchor{east}{\pgfpointadd{\pgfpointorigin}{\pgfpoint{ 0.5 * \innerwidth + \outerwidth}{0cm}}}
    \anchor{west}{\pgfpointadd{\pgfpointorigin}{\pgfpoint{-0.5 * \innerwidth - \outerwidth}{0cm}}}
    \anchor{left in upper}{\pgfpointadd{\pgfpointorigin}{\pgfpoint{-0.5 * \innerwidth - \outerwidth}{ 0.5 * \height}}}
    \anchor{left in lower}{\pgfpointadd{\pgfpointorigin}{\pgfpoint{-0.5 * \innerwidth - \outerwidth}{-0.5 * \height}}}
    \anchor{right in upper}{\pgfpointadd{\pgfpointorigin}{\pgfpoint{ 0.5 * \innerwidth + \outerwidth}{ 0.5 * \height}}}
    \anchor{right in lower}{\pgfpointadd{\pgfpointorigin}{\pgfpoint{ 0.5 * \innerwidth + \outerwidth}{-0.5 * \height}}}
    \beforebackgroundpath{
        \pgfsetlinewidth{\pgfkeysvalueof{/tikz/circuits/line width}}
        \pgfpathmoveto{\pgfpointadd{\pgfpointorigin}{\pgfpoint{-0.5 * \innerwidth - 1 * \outerwidth}{ 0.5 * \height}}}
        \pgfpathlineto{\pgfpointadd{\pgfpointorigin}{\pgfpoint{-0.5 * \innerwidth - 0 * \outerwidth}{ 0.5 * \height}}}
        \pgfpathlineto{\pgfpointadd{\pgfpointorigin}{\pgfpoint{-0.5 * \innerwidth - 0 * \outerwidth}{-0.5 * \height}}}
        \pgfpathlineto{\pgfpointadd{\pgfpointorigin}{\pgfpoint{-0.5 * \innerwidth - 1 * \outerwidth}{-0.5 * \height}}}
        \pgfusepath{stroke}
        \pgfpathmoveto{\pgfpointadd{\pgfpointorigin}{\pgfpoint{ 0.5 * \innerwidth + 1 * \outerwidth}{ 0.5 * \height}}}
        \pgfpathlineto{\pgfpointadd{\pgfpointorigin}{\pgfpoint{ 0.5 * \innerwidth + 0 * \outerwidth}{ 0.5 * \height}}}
        \pgfpathlineto{\pgfpointadd{\pgfpointorigin}{\pgfpoint{ 0.5 * \innerwidth + 0 * \outerwidth}{-0.5 * \height}}}
        \pgfpathlineto{\pgfpointadd{\pgfpointorigin}{\pgfpoint{ 0.5 * \innerwidth + 1 * \outerwidth}{-0.5 * \height}}}
        \pgfusepath{stroke}
        % draw circle
        \pgfpathmoveto{\pgfpointadd{\pgfpointorigin}{\pgfpoint{ 0.5 * \innerwidth}{0.5 * \radius}}}
        \pgfpatharc{90}{270}{0.5 * \radius}
        \pgfusepath{stroke}
        \pgfpathmoveto{\pgfpointadd{\pgfpointorigin}{\pgfpoint{-0.5 * \innerwidth}{0.5 * \radius}}}
        \pgfpatharc{90}{-90}{0.5 * \radius}
        \pgfusepath{stroke}
        % draw arrow
        \ifgyratordrawarrow
            \pgfpathmoveto{\pgfpointadd{\pgfpointorigin}{\pgfpoint{-0.4 * \innerwidth}{0.5 * \height}}}
            \pgfpathlineto{\pgfpointadd{\pgfpointorigin}{\pgfpoint{ 0.4 * \innerwidth}{0.5 * \height}}}
            \pgfsetarrows{-{Stealth}}
            \pgfusepath{stroke}
        \fi
    }
}

% vim: ft=plaintex nowrap

\tikzset{
    circuits/inverter/width/.initial = 1.25cm,
    circuits/inverter/height/.initial = 1.25cm,
    circuits/inverter/radius/.initial = 0.1cm,
    inverter/.style = {invertershape}
}

\pgfdeclareshape{invertershape}
{
    \saveddimen{\width}{\pgf@x=\pgfkeysvalueof{/tikz/circuits/inverter/width}}
    \saveddimen{\height}{\pgf@x=\pgfkeysvalueof{/tikz/circuits/inverter/height}}
    \saveddimen{\radius}{\pgf@x=\pgfkeysvalueof{/tikz/circuits/inverter/radius}}
    \savedanchor{\centerpoint}{\pgfpointorigin}
    \savedanchor{\einput}{%
        \pgfpointadd%
        {\pgfpointorigin}%
        {\pgfpoint%
            {-0.5 * \pgfkeysvalueof{/tikz/circuits/inverter/width}}%
            {0cm}
        }%
    }
    \savedanchor{\output}{%
        \pgfpointadd%
        {\pgfpointorigin}%
        {\pgfpoint%
            {0.5 * \pgfkeysvalueof{/tikz/circuits/inverter/width} + \pgfkeysvalueof{/tikz/circuits/inverter/radius}}%
            {0cm}
        }%
    }
    % electrical terminals (anchors)
    \anchor{in}{\pgfpointadd{\centerpoint}{\pgfpoint{-\width/2}{0cm}}}
    \anchor{out}{\pgfpointadd{\output}{\pgfpoint{\radius}{0cm}}}
    \anchor{+power}{
        \pgfpointlineattime%
        {0.5}%
        {\pgfpointadd{\einput}{\pgfpoint{0cm}{0.5 * \height}}}
        {\pgfpointadd{\output}{\pgfpoint{-\radius}{0cm}}}
    }
    \anchor{-power}{
        \pgfpointlineattime%
        {0.5}%
        {\pgfpointadd{\einput}{\pgfpoint{0cm}{-0.5 * \height}}}
        {\pgfpointadd{\output}{\pgfpoint{-\radius}{0cm}}}
    }
    % regular anchors
    \anchor{center}{\centerpoint}
    \anchor{north}{
        \pgfpointadd{\pgfpointorigin}{\pgfpoint{0cm}{0.5 * \height}}
    }
    \anchor{south}{
        \pgfpointadd{\pgfpointorigin}{\pgfpoint{0cm}{-0.5 * \height}}
    }
    \anchor{east}{
        \pgfpointadd{\pgfpointorigin}{\pgfpoint{\width/2 + 2 * \radius}{0cm}}
    }
    \anchor{west}{
        \pgfpointadd{\pgfpointorigin}{\pgfpoint{-\width/2}{0cm}}
    }
    \anchor{north west}{
        \pgfpointadd{\einput}{\pgfpoint{0cm}{0.5 * \height}}
    }
    \anchor{south west}{
        \pgfpointadd{\pgfpointorigin}{\pgfpoint{-0.5 * \width}{-0.5 * \height}}
    }
    \anchor{north east}{
        \pgfpointadd{\einput}{\pgfpoint{\width + 2 * \radius}{0.5 * \height}}
    }
    \anchor{south east}{
        \pgfpointadd{\pgfpointorigin}{\pgfpoint{0.5 * \width + 2 * \radius}{-0.5 * \height}}
    }
    \beforebackgroundpath{
        \pgfsetlinewidth{\pgfkeysvalueof{/tikz/circuits/line width}}
        \pgfsetmiterjoin
        \pgfsetmiterlimit{2}
        \pgfpathmoveto{\pgfpointadd{\pgfpointorigin}{\pgfpoint{-0.5 * \width}{-0.5 * \height}}}
        \pgfpathlineto{\pgfpointadd{\output}{\pgfpoint{-\radius}{0cm}}}
        \pgfpathlineto{\pgfpointadd{\pgfpointorigin}{\pgfpoint{-0.5 * \width}{ 0.5 * \height}}}
        \pgfpathclose
        \pgfusepath{stroke}
        \pgfpathcircle{
            \pgfpointadd%
            {\pgfpointorigin}%
            {\pgfpoint%
                {0.5 * \pgfkeysvalueof{/tikz/circuits/inverter/width} + \pgfkeysvalueof{/tikz/circuits/inverter/radius}}%
                {0cm}
            }%
        }{\radius}
        \pgfusepath{stroke}
    }
}

% vim: ft=plaintex nowrap

\newif\ifswitchopen

\tikzset{
    % switch
    circuits/switch/open/.is if=switchopen,
    circuits/switch/extension/.initial=0.15cm,
    circuits/switch/length/.initial=0.3cm,
    circuits/switch/circle radius/.initial=0.04cm,
    circuits/switch/angle/.initial=30
}

\pgfdeclareshape{eswitch}
{
    \savedanchor{\center}{
        \pgfpointadd{\pgfpointorigin}{\pgfpoint{sin(\pgfkeysvalueof{/tikz/circuits/switch/angle}) * \pgfkeysvalueof{/tikz/circuits/switch/length}}{0cm}}
    }
    \savedanchor{\plus}{
        \pgfpointadd{\pgfpointorigin}{\pgfpoint{\pgfkeysvalueof{/tikz/circuits/switch/length} + \pgfkeysvalueof{/tikz/circuits/switch/extension}}{0cm}}
    }
    \savedanchor{\minus}{
        \pgfpointadd{\pgfpointorigin}{\pgfpoint{-\pgfkeysvalueof{/tikz/circuits/switch/extension}}{0cm}}
    }
    \saveddimen{\length}{\pgf@x=\pgfkeysvalueof{/tikz/circuits/switch/length}}
    \saveddimen{\xfactor}{\pgfmathsetlength\pgf@x{cos(\pgfkeysvalueof{/tikz/circuits/switch/angle})}}
    \saveddimen{\yfactor}{\pgfmathsetlength\pgf@x{sin(\pgfkeysvalueof{/tikz/circuits/switch/angle})}}
    \saveddimen{\extension}{\pgf@x=\pgfkeysvalueof{/tikz/circuits/switch/extension}}
    \saveddimen{\radius}{\pgf@x=\pgfkeysvalueof{/tikz/circuits/switch/circle radius}}

    \anchor{center}{\center}
    \anchor{plus}{\plus}
    \anchor{minus}{\minus}
    \anchor{left}{\center}
    \anchor{right}{\center}
    \anchor{east}{\plus}
    \anchor{west}{\minus}
    \anchor{north}{\center}
    \anchor{south}{\center}
    \anchor{north east}{\center}
    \anchor{north west}{\center}
    \anchor{south east}{\center}
    \anchor{south west}{\center}

    \beforebackgroundpath{
        \pgfsetlinewidth{\pgfkeysvalueof{/tikz/circuits/line width}}

        % draw switch body
        \pgfpathmoveto{\pgfpointadd{\pgfpointorigin}{\pgfpoint{-\extension}{0cm}}}
        \pgfpathlineto{\pgfpointorigin}
        \pgfpathlineto{\pgfpointadd{\pgfpointorigin}{\pgfpoint{\xfactor * \length}{\yfactor * \length}}}
        \pgfpathmoveto{\pgfpointadd{\pgfpointorigin}{\pgfpoint{\length + \pgfkeysvalueof{/tikz/circuits/line width}}{0cm}}}
        \pgfpathlineto{\pgfpointadd{\pgfpointorigin}{\pgfpoint{\length + \extension}{0cm}}}
        \pgfusepath{stroke}

        % draw switch circles
        \pgfpathcircle{\pgfpointorigin}{\radius}
        \pgfusepath{fill}
        \pgfpathcircle{\pgfpointadd{\pgfpointorigin}{\pgfpoint{\length}{0cm}}}{\radius}
        \pgfusepath{stroke}
    }
}

% vim: ft=plaintex

\pgfkeys{/pgf/.cd,
  magnifying glass handle angle/.initial=-45,
  magnifying glass handle aspect/.initial=1.5
}
\tikzset{
    magglass/size/.initial = 35pt,
    magglass/handle length/.initial = 10pt,
    magglass/.style = {magglassshape, draw}
}

\pgfdeclareshape{magglassshape}
{
    \saveddimen{\size}{\pgf@x=\pgfkeysvalueof{/tikz/magglass/size}}
    \saveddimen{\handlelength}{\pgf@x=\pgfkeysvalueof{/tikz/magglass/handle length}}
    \savedanchor{\center}{\pgfpointorigin}
    \anchor{center}{\center}
    \beforebackgroundpath{
        \pgfpathcircle{\pgfpointorigin}{0.5 * \size}
    }
    %\foregroundpath{
    %    \pgfpathmoveto{\pgfpointadd{\pgfpointorigin}{\pgfpointpolar{-45}{0.5 * \size}}}
    %    \pgfpathlineto{\pgfpointadd{\pgfpointorigin}{\pgfpointpolar{-45}{0.5 * \size + \handlelength}}}
    %    \pgfusepath{stroke}
    %}
}

% vim: ft=plaintex

\tikzset{
    circuits/balun/width/.initial = 1.2cm,
    circuits/balun/height/.initial = 0.6cm,
    circuits/balun/port width/.initial = 2.5pt,
    circuits/balun/port height/.initial = 2.5pt,
    balun/.style = {balunshape, draw, circuits/general}
}

\pgfdeclareshape{balunshape}
{
    \saveddimen{\width}{\pgf@x=\pgfkeysvalueof{/tikz/circuits/balun/width}}
    \saveddimen{\height}{\pgf@x=\pgfkeysvalueof{/tikz/circuits/balun/height}}
    \saveddimen{\portwidth}{\pgf@x=\pgfkeysvalueof{/tikz/circuits/balun/port width}}
    \saveddimen{\portheight}{\pgf@x=\pgfkeysvalueof{/tikz/circuits/balun/port height}}
    \savedanchor{\centerpoint}{\pgfpointorigin}
    \savedanchor{\diffplus}{%
        \pgfpointadd%
        {\pgfpointorigin}%
        {\pgfpoint%
            {-0.5 * \pgfkeysvalueof{/tikz/circuits/balun/width}}%
            {0.25 * \pgfkeysvalueof{/tikz/circuits/balun/height}}
        }%
    }
    \savedanchor{\diffminus}{%
        \pgfpointadd%
        {\pgfpointorigin}%
        {\pgfpoint%
            {-0.5 * \pgfkeysvalueof{/tikz/circuits/balun/width}}%
            {-0.25 * \pgfkeysvalueof{/tikz/circuits/balun/height}}
        }%
    }
    \savedanchor{\singleended}{%
        \pgfpointadd%
        {\pgfpointorigin}%
        {\pgfpoint%
            {0.5 * \pgfkeysvalueof{/tikz/circuits/balun/width}}%
            {0pt}
        }%
    }
    % electrical terminals (anchors)
    \anchor{diffp}{\diffplus}
    \anchor{diffn}{\diffminus}
    \anchor{singleended}{\singleended}
    % regular anchors
    \anchor{center}{\centerpoint}
    \anchor{north}{\pgfpointadd{\pgfpointorigin}{\pgfpoint{0.0 * \width}{ 0.5 * \height}}}
    \anchor{south}{\pgfpointadd{\pgfpointorigin}{\pgfpoint{0.0 * \width}{-0.5 * \height}}}
    \anchor{west}{\pgfpointadd{\pgfpointorigin}{\pgfpoint{-0.5 * \width}{ 0.0 * \height}}}
    \anchor{east}{\pgfpointadd{\pgfpointorigin}{\pgfpoint{ 0.5 * \width}{ 0.0 * \height}}}
    \anchor{north east}{\pgfpointadd{\pgfpointorigin}{\pgfpoint{ 0.5 * \width}{ 0.5 * \height}}}
    \anchor{south east}{\pgfpointadd{\pgfpointorigin}{\pgfpoint{ 0.5 * \width}{-0.5 * \height}}}
    \anchor{north west}{\pgfpointadd{\pgfpointorigin}{\pgfpoint{-0.5 * \width}{ 0.0 * \height}}}
    \anchor{south west}{\pgfpointadd{\pgfpointorigin}{\pgfpoint{-0.5 * \width}{-0.0 * \height}}}
    \anchorborder{
        \@tempdima=\pgf@x
        \@tempdimb=\pgf@y
        \pgfpointborderrectangle{\pgfpoint{\@tempdima}{\@tempdimb}}{\pgfpointadd{\pgfpointorigin}{\pgfpoint{ 0.5 * \width}{ 0.5 * \height}}}
    }
    \beforebackgroundpath{
        \pgfpathrectangle %
            {\pgfpointadd{\pgfpointorigin}{\pgfpoint{-0.5 * \width}{-0.5 * \height}}}
            {\pgfpointadd{\pgfpointorigin}{\pgfpoint{\width}{\height}}}
        \pgfusepath{stroke}
        \pgfpathrectangle %
            {\pgfpointadd{\diffplus}{\pgfpoint{-0.5 * \portwidth}{-0.5 * \portheight}}}
            {\pgfpointadd{\pgfpointorigin}{\pgfpoint{\portwidth}{\portheight}}}
        \pgfusepath{fill}
        \pgfpathrectangle %
            {\pgfpointadd{\diffminus}{\pgfpoint{-0.5 * \portwidth}{-0.5 * \portheight}}}
            {\pgfpointadd{\pgfpointorigin}{\pgfpoint{\portwidth}{\portheight}}}
        \pgfusepath{fill}
        \pgfpathrectangle %
            {\pgfpointadd{\singleended}{\pgfpoint{-0.5 * \portwidth}{-0.5 * \portheight}}}
            {\pgfpointadd{\pgfpointorigin}{\pgfpoint{\portwidth}{\portheight}}}
        \pgfusepath{fill}
        %\pgftext[left, at={\diffplus}, x = 2pt]{\tiny DATA}
        %\pgftext[left, at={\diffminus}, x = 2pt]{\tiny $\overline{\text{DATA}}$}
        %\pgftext[right, at={\singleended}, x = -2pt]{\tiny OUT}
    }
}

% vim: ft=plaintex nowrap

\tikzset{
    circuits/chip/width/.initial = 25pt,
    circuits/chip/pad width/.initial = 0.125cm,
    circuits/chip/pad height/.initial = 0.125cm,
    circuits/chip/pad offset/.initial = 3.5pt,
    circuits/chip/pad pitch/.initial = 0.2cm,
    circuits/chip/left pins/.initial = 5,
    circuits/chip/right pins/.initial = 7,
    chip/.style = {chipshape, draw, circuits/general}
}

\pgfdeclareshape{chipshape}
{
    \saveddimen{\width}{\pgf@x=\pgfkeysvalueof{/tikz/circuits/chip/width}}
    \saveddimen\height{
        \pgfmathsetlength\pgf@x{(max(\pgfkeysvalueof{/tikz/circuits/chip/right pins}, \pgfkeysvalueof{/tikz/circuits/chip/left pins})+1)*\pgfkeysvalueof{/tikz/circuits/chip/pad pitch}}%
    }
    \saveddimen{\padpitch}{\pgf@x=\pgfkeysvalueof{/tikz/circuits/chip/pad pitch}}
    \saveddimen{\padwidth}{\pgf@x=\pgfkeysvalueof{/tikz/circuits/chip/pad width}}
    \saveddimen{\padheight}{\pgf@x=\pgfkeysvalueof{/tikz/circuits/chip/pad height}}
    \savedmacro{\leftpins}{\renewcommand{\leftpins}[0]{\pgfkeysvalueof{/tikz/circuits/chip/left pins}}}
    \savedmacro{\rightpins}{\renewcommand{\rightpins}[0]{\pgfkeysvalueof{/tikz/circuits/chip/right pins}}}
    \saveddimen{\padoffset}{\pgf@x=\pgfkeysvalueof{/tikz/circuits/chip/pad offset}}
    \savedanchor{\centerpoint}{\pgfpointorigin}
    % electrical anchors
    \anchor{rightmidpad}{\pgfpointadd{\pgfpointorigin}{\pgfpoint{0.5 * \width - \padoffset - 0.5 * \padwidth}{0cm}}}
    \anchor{leftmidpad}{\pgfpointadd{\pgfpointorigin}{\pgfpoint{-0.5 * \width + \padoffset + 0.5 * \padwidth}{0cm}}}
    % regular anchors
    \anchor{text}
    {
        \pgfpoint{-0.5\wd\pgfnodeparttextbox}{-0.5\ht\pgfnodeparttextbox}
    }
    \anchor{center}{\centerpoint}
    \anchor{north}{\pgfpointadd{\pgfpointorigin}{\pgfpoint{0.0 * \width}{ 0.5 * \height}}}
    \anchor{south}{\pgfpointadd{\pgfpointorigin}{\pgfpoint{0.0 * \width}{-0.5 * \height}}}
    \anchor{west}{\pgfpointadd{\pgfpointorigin}{\pgfpoint{-0.5 * \width}{ 0.0 * \height}}}
    \anchor{east}{\pgfpointadd{\pgfpointorigin}{\pgfpoint{ 0.5 * \width}{ 0.0 * \height}}}
    \anchor{north east}{\pgfpointadd{\pgfpointorigin}{\pgfpoint{ 0.5 * \width}{ 0.5 * \height}}}
    \anchor{south east}{\pgfpointadd{\pgfpointorigin}{\pgfpoint{ 0.5 * \width}{-0.5 * \height}}}
    \anchor{north west}{\pgfpointadd{\pgfpointorigin}{\pgfpoint{-0.5 * \width}{ 0.0 * \height}}}
    \anchor{south west}{\pgfpointadd{\pgfpointorigin}{\pgfpoint{-0.5 * \width}{-0.0 * \height}}}
    \anchorborder{
        \@tempdima=\pgf@x
        \@tempdimb=\pgf@y
        \pgfpointborderrectangle{\pgfpoint{\@tempdima}{\@tempdimb}}{\pgfpointadd{\pgfpointorigin}{\pgfpoint{0.5 * \width}{0.5 * \height}}}
    }
    \backgroundpath{%
        \pgfpathrectanglecorners %
            {\pgfpointadd{\pgfpointorigin}{\pgfpoint{-0.5 * \width}{-0.5 * \height}}}
            {\pgfpointadd{\pgfpointorigin}{\pgfpoint{ 0.5 * \width}{ 0.5 * \height}}}
    }
    \beforebackgroundpath{%
        \foreach \y in {1, 2, ..., \leftpins}
        {
            \pgfpathrectangle%
                {\pgfpointadd{\pgfpointorigin}{\pgfpoint{-0.5 * \width + \padwidth + \padoffset}{(\y - 0.5 * (\leftpins + 1)) * \padpitch - 0.5 * \padheight}}}
                {\pgfpointadd{\pgfpointorigin}{\pgfpoint{-\padwidth}{\padheight}}}
        }
        \foreach \y in {1, 2, ..., \rightpins}
        {
            \pgfpathrectangle%
                {\pgfpointadd{\pgfpointorigin}{\pgfpoint{0.5 * \width - \padwidth - \padoffset}{(\y - 0.5 * (\rightpins + 1)) * \padpitch - 0.5 * \padheight}}}
                {\pgfpointadd{\pgfpointorigin}{\pgfpoint{\padwidth}{\padheight}}}
        }
    }
    % \pgf@sh@s@chip contains all the code for the chip shape
    % and is executed just before a chip node is drawn.
    \pgfutil@g@addto@macro\pgf@sh@s@chipshape{%
        % Start with the maximum pin number and go backwards.
        % If the anchor is undefined, create it. Otherwise stop.
        \c@pgf@counta=\pgfkeysvalueof{/tikz/circuits/chip/right pins}\relax%
        \pgfmathloop%
        \ifnum\c@pgf@counta>0\relax%
            \pgfutil@ifundefined{pgf@anchor@chipshape@pin\space\the\c@pgf@counta}{%
                \expandafter\xdef\csname pgf@anchor@chipshape@rightpin\space\the\c@pgf@counta\endcsname{%
                    \noexpand\chiprightpinanchor{\the\c@pgf@counta}%
                }%
            }{\c@pgf@counta=0\relax}%
            \advance\c@pgf@counta-1\relax%
        \repeatpgfmathloop%  
    }%
    % \pgf@sh@s@chip contains all the code for the chip shape
    % and is executed just before a chip node is drawn.
    \pgfutil@g@addto@macro\pgf@sh@s@chipshape{%
        % Start with the maximum pin number and go backwards.
        % If the anchor is undefined, create it. Otherwise stop.
        \c@pgf@counta=\pgfkeysvalueof{/tikz/circuits/chip/left pins}\relax%
        \pgfmathloop%
        \ifnum\c@pgf@counta>0\relax%
            \pgfutil@ifundefined{pgf@anchor@chipshape@pin\space\the\c@pgf@counta}{%
                \expandafter\xdef\csname pgf@anchor@chipshape@leftpin\space\the\c@pgf@counta\endcsname{%
                    \noexpand\chipleftpinanchor{\the\c@pgf@counta}%
                }%
            }{\c@pgf@counta=0\relax}%
            \advance\c@pgf@counta-1\relax%
        \repeatpgfmathloop%  
    }%
}

\def\chiprightpinanchor#1{%
    % When this macro is called,
    % \width, \needlelength, \pins and \padpitch will be defined
    \pgfpointadd{\pgfpointorigin}{\pgfpoint{0.5 * \width - 0.5 * \padwidth - \padoffset}{(#1 - 0.5 * (\rightpins + 1)) * \padpitch}}
}
\def\chipleftpinanchor#1{%
    % When this macro is called,
    % \width, \needlelength, \pins and \padpitch will be defined
    \pgfpointadd{\pgfpointorigin}{\pgfpoint{-0.5 * \width + 0.5 * \padwidth + \padoffset}{(#1 - 0.5 * (\leftpins + 1)) * \padpitch}}
}


% vim: ft=plaintex

\tikzset{
    circuits/probe/width/.initial = 25pt,
    circuits/probe/offset/.initial = 0.15cm,
    circuits/probe/pad pitch/.initial = 0.2cm,
    circuits/probe/needles/.initial = 3,
    circuits/probe/needle width/.initial = 3pt,
    circuits/probe/needle length/.initial = 10pt,
    circuits/probe/ports/.initial = 1,
    circuits/probe/port width/.initial = 2.5pt,
    circuits/probe/port height/.initial = 2.5pt,
    circuits/probe/port pitch/.initial = 0.2cm,
    probe/.style = {probeshape, draw, circuits/general}
}

\pgfdeclareshape{probeshape}
{
    \saveddimen{\width}{\pgf@x=\pgfkeysvalueof{/tikz/circuits/probe/width}}
    \saveddimen\height{
        \pgfmathsetlength\pgf@x{(\pgfkeysvalueof{/tikz/circuits/probe/needles} + 1) * \pgfkeysvalueof{/tikz/circuits/probe/pad pitch} + \pgfkeysvalueof{/tikz/circuits/probe/offset}}%
    }
    \saveddimen{\offset}{\pgf@x=\pgfkeysvalueof{/tikz/circuits/probe/offset}}
    \saveddimen{\padpitch}{\pgf@x=\pgfkeysvalueof{/tikz/circuits/probe/pad pitch}}
    \saveddimen{\portpitch}{\pgf@x=\pgfkeysvalueof{/tikz/circuits/probe/port pitch}}
    \savedmacro{\needles}{\renewcommand{\needles}[0]{\pgfkeysvalueof{/tikz/circuits/probe/needles}}}
    \saveddimen{\needlewidth}{\pgf@x=\pgfkeysvalueof{/tikz/circuits/probe/needle width}}
    \saveddimen{\needlelength}{\pgf@x=\pgfkeysvalueof{/tikz/circuits/probe/needle length}}
    \savedmacro{\ports}{\renewcommand{\ports}[0]{\pgfkeysvalueof{/tikz/circuits/probe/ports}}}
    \saveddimen{\portwidth}{\pgf@x=\pgfkeysvalueof{/tikz/circuits/probe/port width}}
    \saveddimen{\portheight}{\pgf@x=\pgfkeysvalueof{/tikz/circuits/probe/port height}}
    \savedanchor{\centerpoint}{\pgfpointorigin}
    % electrical anchors
    \anchor{midneedle}{\pgfpointadd{\pgfpointorigin}{\pgfpoint{-0.5 * \width - \needlelength}{0cm}}}
    % regular anchors
    \anchor{text}
    {
        \pgfpoint{-0.5\wd\pgfnodeparttextbox}{-0.5\ht\pgfnodeparttextbox}
    }
    \anchor{center}{\centerpoint}
    \anchor{north}{\pgfpointadd{\pgfpointorigin}{\pgfpoint{0.0 * \width}{ 0.5 * \height}}}
    \anchor{south}{\pgfpointadd{\pgfpointorigin}{\pgfpoint{0.0 * \width}{-0.5 * \height}}}
    \anchor{west}{\pgfpointadd{\pgfpointorigin}{\pgfpoint{-0.5 * \width}{ 0.0 * \height}}}
    \anchor{east}{\pgfpointadd{\pgfpointorigin}{\pgfpoint{ 0.5 * \width}{ 0.0 * \height}}}
    \anchor{north east}{\pgfpointadd{\pgfpointorigin}{\pgfpoint{ 0.5 * \width}{ 0.5 * \height}}}
    \anchor{south east}{\pgfpointadd{\pgfpointorigin}{\pgfpoint{ 0.5 * \width}{-0.5 * \height}}}
    \anchor{north west}{\pgfpointadd{\pgfpointorigin}{\pgfpoint{-0.5 * \width}{ 0.5 * \height}}}
    \anchor{south west}{\pgfpointadd{\pgfpointorigin}{\pgfpoint{-0.5 * \width}{-0.5 * \height}}}
    \anchorborder{
        \@tempdima=\pgf@x
        \@tempdimb=\pgf@y
        \pgfpointborderrectangle{\pgfpoint{\@tempdima}{\@tempdimb}}{\pgfpointadd{\pgfpointorigin}{\pgfpoint{0.5 * \width}{0.5 * \height}}}
    }
    \beforebackgroundpath{%
        % draw body
        \pgfpathmoveto{\pgfpointadd{\pgfpointorigin}{\pgfpoint{ 0.5 * \width}{0cm}}}
        \pgfpathlineto{\pgfpointadd{\pgfpointorigin}{\pgfpoint{ 0.5 * \width}{-0.5 * \height}}}
        \pgfpathlineto{\pgfpointadd{\pgfpointorigin}{\pgfpoint{-0.5 * \width + \offset}{-0.5 * \height}}}
        \pgfpathlineto{\pgfpointadd{\pgfpointorigin}{\pgfpoint{-0.5 * \width}{-0.5 * \height + \offset}}}
        \pgfpathlineto{\pgfpointadd{\pgfpointorigin}{\pgfpoint{-0.5 * \width}{ 0.5 * \height - \offset}}}
        \pgfpathlineto{\pgfpointadd{\pgfpointorigin}{\pgfpoint{-0.5 * \width + \offset}{ 0.5 * \height}}}
        \pgfpathlineto{\pgfpointadd{\pgfpointorigin}{\pgfpoint{ 0.5 * \width}{0.5 * \height}}}
        \pgfpathlineto{\pgfpointadd{\pgfpointorigin}{\pgfpoint{ 0.5 * \width}{0cm}}}
%
        %\pgfpathmoveto{\pgfpointadd{\pgfpointorigin}{\pgfpoint{-0.5 * \width + \offset + \width}{0cm}}}
        %\pgfpathlineto{\pgfpointadd{\pgfpointorigin}{\pgfpoint{-0.5 * \width + \offset + \width}{-0.5 * \height - \offset}}}
        %\pgfpathlineto{\pgfpointadd{\pgfpointorigin}{\pgfpoint{-0.5 * \width + \offset}{-0.5 * \height - \offset}}}
        %\pgfpathlineto{\pgfpointadd{\pgfpointorigin}{\pgfpoint{-0.5 * \width}{-0.5 * \height}}}
        %\pgfpathlineto{\pgfpointadd{\pgfpointorigin}{\pgfpoint{-0.5 * \width}{0.5 * \height}}}
        %\pgfpathlineto{\pgfpointadd{\pgfpointorigin}{\pgfpoint{-0.5 * \width + \offset}{0.5 * \height + \offset}}}
        %\pgfpathlineto{\pgfpointadd{\pgfpointorigin}{\pgfpoint{-0.5 * \width + \offset + \width}{0.5 * \height + \offset}}}
        %\pgfpathlineto{\pgfpointadd{\pgfpointorigin}{\pgfpoint{-0.5 * \width + \offset + \width}{0cm}}}
        \pgfusepath{stroke}
        % draw needles
        \pgfsetbeveljoin
        \foreach \y in {1, ..., \needles}
        {
            \pgfpathmoveto{\pgfpointadd{\pgfpointorigin}{\pgfpoint{-0.5 * \width}{(\y - 0.5 * (\needles + 1)) * \padpitch + 0.5 * \needlewidth}}}
            \pgfpathlineto{\pgfpointadd{\pgfpointorigin}{\pgfpoint{-0.5 * \width - \needlelength}{(\y - 0.5 * (\needles + 1)) * \padpitch}}}
            \pgfpathlineto{\pgfpointadd{\pgfpointorigin}{\pgfpoint{-0.5 * \width}{(\y - 0.5 * (\needles + 1)) * \padpitch - 0.5 * \needlewidth}}}
            \pgfusepath{stroke, fill}
        }
        % draw ports
        \foreach \y in {1, ..., \ports}
        {
            \pgfpathrectangle
                {\pgfpointadd{\pgfpointorigin}{\pgfpoint{0.5 * \width - 0.5 * \portwidth}{(\y - 0.5 * (\ports + 1)) * \portpitch - 0.5 * \portheight}}}
                {\pgfpointadd{\pgfpointorigin}{\pgfpoint{\portwidth}{\portheight}}}
            \pgfusepath{fill}
        }
    }
    % \pgf@sh@s@probe contains all the code for the probe shape
    % and is executed just before a probe node is drawn.
    \pgfutil@g@addto@macro\pgf@sh@s@probeshape{%
        % Start with the maximum pin number and go backwards.
        % If the anchor is undefined, create it. Otherwise stop.
        \c@pgf@counta=\pgfkeysvalueof{/tikz/circuits/probe/ports}\relax%
        \pgfmathloop%
        \ifnum\c@pgf@counta>0\relax%
            \pgfutil@ifundefined{pgf@anchor@probeshape@port\space\the\c@pgf@counta}{%
                \expandafter\xdef\csname pgf@anchor@probeshape@port\space\the\c@pgf@counta\endcsname{%
                    \noexpand\probeportanchor{\the\c@pgf@counta}%
                }%
            }{\c@pgf@counta=0\relax}%
            \advance\c@pgf@counta-1\relax%
        \repeatpgfmathloop%  
    }%
    % \pgf@sh@s@probe contains all the code for the probe shape
    % and is executed just before a probe node is drawn.
    \pgfutil@g@addto@macro\pgf@sh@s@probeshape{%
        % Start with the maximum pin number and go backwards.
        % If the anchor is undefined, create it. Otherwise stop.
        \c@pgf@counta=\pgfkeysvalueof{/tikz/circuits/probe/needles}\relax%
        \pgfmathloop%
        \ifnum\c@pgf@counta>0\relax%
            \pgfutil@ifundefined{pgf@anchor@probeshape@needle\space\the\c@pgf@counta}{%
                \expandafter\xdef\csname pgf@anchor@probeshape@needle\space\the\c@pgf@counta\endcsname{%
                    \noexpand\probeneedleanchor{\the\c@pgf@counta}%
                }%
            }{\c@pgf@counta=0\relax}%
            \advance\c@pgf@counta-1\relax%
        \repeatpgfmathloop%  
    }%
    % \pgf@sh@s@probe contains all the code for the probe shape
    % and is executed just before a probe node is drawn.
    \pgfutil@g@addto@macro\pgf@sh@s@probeshape{%
        % Start with the maximum pin number and go backwards.
        % If the anchor is undefined, create it. Otherwise stop.
        \c@pgf@counta=\pgfkeysvalueof{/tikz/circuits/probe/needles}\relax%
        \pgfmathloop%
        \ifnum\c@pgf@counta>0\relax%
            \pgfutil@ifundefined{pgf@anchor@probeshape@innerport\space\the\c@pgf@counta}{%
                \expandafter\xdef\csname pgf@anchor@probeshape@innerport\space\the\c@pgf@counta\endcsname{%
                    \noexpand\probeinnerportanchor{\the\c@pgf@counta}%
                }%
            }{\c@pgf@counta=0\relax}%
            \advance\c@pgf@counta-1\relax%
        \repeatpgfmathloop%  
    }%
}

\def\probeportanchor#1{%
    % When this macro is called,
    % \width, \needlelength, \pins and \portpitch will be defined
    \pgfpointadd{\pgfpointorigin}{\pgfpoint{0.5 * \width}{(#1 - 0.5 * (\ports + 1)) * \portpitch}}
}
\def\probeinnerportanchor#1{%
    % When this macro is called,
    % \width, \needlelength, \pins and \portpitch will be defined
    \pgfpointadd{\pgfpointorigin}{\pgfpoint{-0.5 * \width}{(#1 - 0.5 * (\needles + 1)) * \padpitch}}
}
\def\probeneedleanchor#1{%
    % When this macro is called,
    % \width, \needlelength, \pins and \padpitch will be defined
    \pgfpointadd{\pgfpointorigin}{\pgfpoint{-0.5 * \width - \needlelength}{(#1 - 0.5 * (\needles + 1)) * \padpitch}}
}

% vim: ft=plaintex

\tikzset{
    circuits/instrument/width/.initial = 2cm,
    circuits/instrument/height/.initial = 1.25cm,
    circuits/instrument/screen width/.initial = 1.4cm,
    circuits/instrument/screen height/.initial = 0.9cm,
    circuits/instrument/screen xoffset/.initial = 2pt,
    circuits/instrument/screen yoffset/.initial = 2pt,
    circuits/instrument/port radius/.initial = 2pt,
    circuits/instrument/grid x segments/.initial = 9,
    circuits/instrument/grid y segments/.initial = 6,
    instrument/.style = {instrumentshape, draw, circuits/general}
}

\pgfdeclareshape{instrumentshape}
{
    \saveddimen{\width}{\pgf@x=\pgfkeysvalueof{/tikz/circuits/instrument/width}}
    \saveddimen{\height}{\pgf@x=\pgfkeysvalueof{/tikz/circuits/instrument/height}}
    \saveddimen{\screenwidth}{\pgf@x=\pgfkeysvalueof{/tikz/circuits/instrument/screen width}}
    \saveddimen{\screenheight}{\pgf@x=\pgfkeysvalueof{/tikz/circuits/instrument/screen height}}
    \saveddimen{\screenxoffset}{\pgf@x=\pgfkeysvalueof{/tikz/circuits/instrument/screen xoffset}}
    \saveddimen{\screenyoffset}{\pgf@x=\pgfkeysvalueof{/tikz/circuits/instrument/screen yoffset}}
    \saveddimen{\portradius}{\pgf@x=\pgfkeysvalueof{/tikz/circuits/instrument/port radius}}
    \saveddimen\gridxstep{
        \pgfmathsetlength\pgf@x{\pgfkeysvalueof{/tikz/circuits/instrument/screen width} / \pgfkeysvalueof{/tikz/circuits/instrument/grid x segments}}
    }
    \saveddimen\gridystep{
        \pgfmathsetlength\pgf@x{\pgfkeysvalueof{/tikz/circuits/instrument/screen height} / \pgfkeysvalueof{/tikz/circuits/instrument/grid y segments}}
    }
    \savedanchor{\centerpoint}{\pgfpointorigin}
    % electrical anchors
    \anchor{port}{\pgfpointadd{\pgfpointorigin}{\pgfpoint{0.5 * \width - 2 * \portradius}{-0.5 * \height + 2 * \portradius}}}
    % regular anchors
    \anchor{text}
    {
        \pgfpoint{-0.5\wd\pgfnodeparttextbox}{-0.5\ht\pgfnodeparttextbox}
    }
    \anchor{center}{\centerpoint}
    \anchor{north}{\pgfpointadd{\pgfpointorigin}{\pgfpoint{0.0 * \width}{ 0.5 * \height}}}
    \anchor{south}{\pgfpointadd{\pgfpointorigin}{\pgfpoint{0.0 * \width}{-0.5 * \height}}}
    \anchor{west}{\pgfpointadd{\pgfpointorigin}{\pgfpoint{-0.5 * \width}{ 0.0 * \height}}}
    \anchor{east}{\pgfpointadd{\pgfpointorigin}{\pgfpoint{ 0.5 * \width}{ 0.0 * \height}}}
    \anchor{north east}{\pgfpointadd{\pgfpointorigin}{\pgfpoint{ 0.5 * \width}{ 0.5 * \height}}}
    \anchor{south east}{\pgfpointadd{\pgfpointorigin}{\pgfpoint{ 0.5 * \width}{-0.5 * \height}}}
    \anchor{north west}{\pgfpointadd{\pgfpointorigin}{\pgfpoint{-0.5 * \width}{ 0.5 * \height}}}
    \anchor{south west}{\pgfpointadd{\pgfpointorigin}{\pgfpoint{-0.5 * \width}{-0.5 * \height}}}
    \anchorborder{
        \@tempdima=\pgf@x
        \@tempdimb=\pgf@y
        \pgfpointborderrectangle{\pgfpoint{\@tempdima}{\@tempdimb}}{\pgfpointadd{\pgfpointorigin}{\pgfpoint{ 0.5 * \width}{ 0.5 * \height}}}
    }
    \beforebackgroundpath{
        % draw outer body
        \pgfpathrectanglecorners
            {\pgfpointadd{\pgfpointorigin}{\pgfpoint{-0.5 * \width}{-0.5 * \height}}}
            {\pgfpointadd{\pgfpointorigin}{\pgfpoint{ 0.5 * \width}{ 0.5 * \height}}}
        % draw screen
        \pgfpathrectangle
            {\pgfpointadd{\pgfpointorigin}{\pgfpoint{-0.5 * \width + \screenxoffset}{0.5 * \height - \screenyoffset}}}
            {\pgfpointadd{\pgfpointorigin}{\pgfpoint{\screenwidth}{-\screenheight}}}
        \pgfusepath{stroke}
        \pgfsetlinewidth{0.15pt}
        \pgfpathgrid[step = \pgfpoint{\gridxstep}{\gridystep}]
            {\pgfpointadd{\pgfpointorigin}{\pgfpoint{-0.5 * \width + \screenxoffset}{0.5 * \height - \screenyoffset}}}
            {\pgfpointadd{\pgfpointorigin}{\pgfpoint{-0.5 * \width + \screenxoffset + \screenwidth}{0.5 * \height - \screenyoffset - \screenheight}}}
        \pgfusepath{stroke}
        % draw signal
        \pgfsetlinewidth{0.8pt}
        \pgfpathmoveto{\pgfpointadd{\pgfpointorigin}{\pgfpoint{-0.5 * \width + \screenxoffset}{0.5 * \height - \screenyoffset - 0.8 * \screenheight}}}
        \pgfpathcurveto
            {\pgfpoint{-0.5 * \width + \screenxoffset + 0.5 * \screenwidth}{0.5 * \height - \screenyoffset - 0.8 * \screenheight}}
            {\pgfpoint{-0.5 * \width + \screenxoffset + 0.5 * \screenwidth}{0.5 * \height - \screenyoffset - 0.7 * \screenheight}}
            {\pgfpoint{-0.5 * \width + \screenxoffset + 0.5 * \screenwidth}{0.5 * \height - \screenyoffset - 0.2 * \screenheight}}
        \pgfpathcurveto
            {\pgfpoint{-0.5 * \width + \screenxoffset + 0.5 * \screenwidth}{0.5 * \height - \screenyoffset - 0.7 * \screenheight}}
            {\pgfpoint{-0.5 * \width + \screenxoffset + 0.5 * \screenwidth}{0.5 * \height - \screenyoffset - 0.8 * \screenheight}}
            {\pgfpoint{-0.5 * \width + \screenxoffset + 1.0 * \screenwidth}{0.5 * \height - \screenyoffset - 0.8 * \screenheight}}
        \pgfusepath{stroke}
        % draw port
        \pgfpathcircle
            {\pgfpointadd{\pgfpointorigin}{\pgfpoint{0.5 * \width - 2 * \portradius}{-0.5 * \height + 2 * \portradius}}}
            {\portradius}
    }
}

% vim: ft=plaintex

\tikzset{
    circuits/supply/width/.initial = 1.2cm,
    circuits/supply/height/.initial = 0.8cm,
    circuits/supply/ports/.initial = 3,
    circuits/supply/port radius/.initial = 2pt,
    supply/.style = {supplyshape, draw, circuits/general}
}

\pgfdeclareshape{supplyshape}
{
    \saveddimen{\width}{\pgf@x=\pgfkeysvalueof{/tikz/circuits/supply/width}}
    \saveddimen{\height}{\pgf@x=\pgfkeysvalueof{/tikz/circuits/supply/height}}
    \savedmacro{\ports}{\renewcommand{\ports}[0]{\pgfkeysvalueof{/tikz/circuits/supply/ports}}}
    \saveddimen{\portradius}{\pgf@x=\pgfkeysvalueof{/tikz/circuits/supply/port radius}}
    \saveddimen{\portpitch}{
        \pgfmathsetlength\pgf@x{\pgfkeysvalueof{/tikz/circuits/supply/width} / (\pgfkeysvalueof{/tikz/circuits/supply/ports})}
    }
    \savedanchor{\centerpoint}{\pgfpointorigin}
    % regular anchors
    \anchor{text}
    {
        \pgfpoint{-0.5\wd\pgfnodeparttextbox}{-0.5\ht\pgfnodeparttextbox}
    }
    \anchor{center}{\centerpoint}
    \anchor{north}{\pgfpointadd{\pgfpointorigin}{\pgfpoint{0.0 * \width}{ 0.5 * \height}}}
    \anchor{south}{\pgfpointadd{\pgfpointorigin}{\pgfpoint{0.0 * \width}{-0.5 * \height}}}
    \anchor{west}{\pgfpointadd{\pgfpointorigin}{\pgfpoint{-0.5 * \width}{ 0.0 * \height}}}
    \anchor{east}{\pgfpointadd{\pgfpointorigin}{\pgfpoint{ 0.5 * \width}{ 0.0 * \height}}}
    \anchor{north east}{\pgfpointadd{\pgfpointorigin}{\pgfpoint{ 0.5 * \width}{ 0.5 * \height}}}
    \anchor{south east}{\pgfpointadd{\pgfpointorigin}{\pgfpoint{ 0.5 * \width}{-0.5 * \height}}}
    \anchor{north west}{\pgfpointadd{\pgfpointorigin}{\pgfpoint{-0.5 * \width}{ 0.5 * \height}}}
    \anchor{south west}{\pgfpointadd{\pgfpointorigin}{\pgfpoint{-0.5 * \width}{-0.5 * \height}}}
    \anchorborder{
        \@tempdima=\pgf@x
        \@tempdimb=\pgf@y
        \pgfpointborderrectangle{\pgfpoint{\@tempdima}{\@tempdimb}}{\pgfpointadd{\pgfpointorigin}{\pgfpoint{ 0.5 * \width}{ 0.5 * \height}}}
    }
    \beforebackgroundpath{
        \pgfpathrectanglecorners
            {\pgfpointadd{\pgfpointorigin}{\pgfpoint{-0.5 * \width}{-0.5 * \height}}}
            {\pgfpointadd{\pgfpointorigin}{\pgfpoint{ 0.5 * \width}{ 0.5 * \height}}}
        \foreach \x in {1, 2, ..., \ports}
        {
            \pgfpathcircle
                {\pgfpointadd{\pgfpointorigin}{\pgfpoint{(\x - 0.5 * (\ports + 1)) * \portpitch}{-0.5 * \height + 2 * \portradius}}}
                {\portradius}
            \pgfusepath{stroke}
        }
    }
    % \pgf@sh@s@probe contains all the code for the probe shape
    % and is executed just before a probe node is drawn.
    \pgfutil@g@addto@macro\pgf@sh@s@supplyshape{%
        % Start with the maximum pin number and go backwards.
        % If the anchor is undefined, create it. Otherwise stop.
        \c@pgf@counta=\pgfkeysvalueof{/tikz/circuits/supply/ports}\relax%
        \pgfmathloop%
        \ifnum\c@pgf@counta>0\relax%
            \pgfutil@ifundefined{pgf@anchor@supplyshape@port\space\the\c@pgf@counta}{%
                \expandafter\xdef\csname pgf@anchor@supplyshape@port\space\the\c@pgf@counta\endcsname{%
                    \noexpand\supplyportanchor{\the\c@pgf@counta}%
                }%
            }{\c@pgf@counta=0\relax}%
            \advance\c@pgf@counta-1\relax%
        \repeatpgfmathloop%  
    }%
}

\def\supplyportanchor#1{%
    \pgfpointadd{\pgfpointorigin}{\pgfpoint{(#1 - 0.5 * (\ports + 1)) * \portpitch}{-0.5 * \height + 2 * \portradius}}
}

% vim: ft=plaintex

\tikzset{
    circuits/computer/width/.initial = 0.5cm,
    circuits/computer/height/.initial = 1.2cm,
    computer/.style = {computershape, draw, circuits/general}
}

\pgfdeclareshape{computershape}
{
    \saveddimen{\width}{\pgf@x=\pgfkeysvalueof{/tikz/circuits/computer/width}}
    \saveddimen{\height}{\pgf@x=\pgfkeysvalueof{/tikz/circuits/computer/height}}
    \savedanchor{\centerpoint}{\pgfpointorigin}
    % electrical anchors
    % regular anchors
    \anchor{text}
    {
        \pgfpoint{-0.5\wd\pgfnodeparttextbox}{-0.5\ht\pgfnodeparttextbox}
    }
    \anchor{center}{\centerpoint}
    \anchor{north}{\pgfpointadd{\pgfpointorigin}{\pgfpoint{0.0 * \width}{ 0.5 * \height}}}
    \anchor{south}{\pgfpointadd{\pgfpointorigin}{\pgfpoint{0.0 * \width}{-0.5 * \height}}}
    \anchor{west}{\pgfpointadd{\pgfpointorigin}{\pgfpoint{-0.5 * \width}{ 0.0 * \height}}}
    \anchor{east}{\pgfpointadd{\pgfpointorigin}{\pgfpoint{ 0.5 * \width}{ 0.0 * \height}}}
    \anchor{north east}{\pgfpointadd{\pgfpointorigin}{\pgfpoint{ 0.5 * \width}{ 0.5 * \height}}}
    \anchor{south east}{\pgfpointadd{\pgfpointorigin}{\pgfpoint{ 0.5 * \width}{-0.5 * \height}}}
    \anchor{north west}{\pgfpointadd{\pgfpointorigin}{\pgfpoint{-0.5 * \width}{ 0.5 * \height}}}
    \anchor{south west}{\pgfpointadd{\pgfpointorigin}{\pgfpoint{-0.5 * \width}{-0.5 * \height}}}
    \anchorborder{
        \@tempdima=\pgf@x
        \@tempdimb=\pgf@y
        \pgfpointborderrectangle{\pgfpoint{\@tempdima}{\@tempdimb}}{\pgfpointadd{\pgfpointorigin}{\pgfpoint{ 0.5 * \width}{ 0.5 * \height}}}
    }
    \beforebackgroundpath{
        \pgfpathrectanglecorners
            {\pgfpointadd{\pgfpointorigin}{\pgfpoint{-0.5 * \width}{-0.5 * \height}}}
            {\pgfpointadd{\pgfpointorigin}{\pgfpoint{ 0.5 * \width}{ 0.5 * \height}}}
        \pgfusepath{stroke}
        \pgfpathrectanglecorners
            {\pgfpointadd{\pgfpointorigin}{\pgfpoint{-0.4 * \width}{ 0.35 * \height}}}
            {\pgfpointadd{\pgfpointorigin}{\pgfpoint{ 0.4 * \width}{ 0.30 * \height}}}
        \pgfusepath{fill}
        \pgfpathrectanglecorners
            {\pgfpointadd{\pgfpointorigin}{\pgfpoint{-0.4 * \width}{ 0.28 * \height}}}
            {\pgfpointadd{\pgfpointorigin}{\pgfpoint{ 0.4 * \width}{ 0.23 * \height}}}
        \pgfusepath{fill}
        \pgfpathcircle
            {\pgfpointadd{\pgfpointorigin}{\pgfpoint{0cm}{-0.1 * \height}}}
            {1.25pt}
        \pgfusepath{fill}
        \pgfpathcircle
            {\pgfpointadd{\pgfpointorigin}{\pgfpoint{0cm}{-0.2 * \height}}}
            {0.9pt}
        \pgfusepath{fill}
    }
}

% vim: ft=plaintex

\tikzset{
    circuits/pcb/width/.initial = 0.7cm,
    circuits/pcb/height/.initial = 0.6cm,
    circuits/pcb/chip size/.initial = 0.15cm,
    circuits/pcb/chip pins/.initial = 4,
    circuits/pcb/chip pin length/.initial = 1pt,
    circuits/pcb/out pins/.initial = 3,
    circuits/pcb/pin width/.initial = 2pt,
    circuits/pcb/pin height/.initial = 2pt,
    pcb/.style = {pcbshape, draw, circuits/general}
}

\pgfdeclareshape{pcbshape}
{
    \saveddimen{\width}{\pgf@x=\pgfkeysvalueof{/tikz/circuits/pcb/width}}
    \saveddimen{\height}{\pgf@x=\pgfkeysvalueof{/tikz/circuits/pcb/height}}
    \saveddimen{\chipsize}{\pgf@x=\pgfkeysvalueof{/tikz/circuits/pcb/chip size}}
    \savedmacro{\chippins}{\renewcommand{\chippins}[0]{\pgfkeysvalueof{/tikz/circuits/pcb/chip pins}}}
    \saveddimen{\chippinlength}{\pgf@x=\pgfkeysvalueof{/tikz/circuits/pcb/chip pin length}}
    \saveddimen{\chippinpitch}{
        \pgfmathsetlength\pgf@x{\pgfkeysvalueof{/tikz/circuits/pcb/chip size} / (\pgfkeysvalueof{/tikz/circuits/pcb/chip pins} + 1)}
    }
    \savedmacro{\pins}{\renewcommand{\pins}[0]{\pgfkeysvalueof{/tikz/circuits/pcb/out pins}}}
    \saveddimen{\pinpitch}{
        \pgfmathsetlength\pgf@x{\pgfkeysvalueof{/tikz/circuits/pcb/height} / (\pgfkeysvalueof{/tikz/circuits/pcb/out pins} + 2)}
    }
    \saveddimen{\pinwidth}{\pgf@x=\pgfkeysvalueof{/tikz/circuits/pcb/pin width}}
    \saveddimen{\pinheight}{\pgf@x=\pgfkeysvalueof{/tikz/circuits/pcb/pin height}}
    \savedanchor{\centerpoint}{\pgfpointorigin}
    % electrical anchors
    % regular anchors
    \anchor{text}
    {
        \pgfpoint{-0.5\wd\pgfnodeparttextbox}{-0.5\ht\pgfnodeparttextbox}
    }
    \anchor{center}{\centerpoint}
    \anchor{north}{\pgfpointadd{\pgfpointorigin}{\pgfpoint{0.0 * \width}{ 0.5 * \height}}}
    \anchor{south}{\pgfpointadd{\pgfpointorigin}{\pgfpoint{0.0 * \width}{-0.5 * \height}}}
    \anchor{west}{\pgfpointadd{\pgfpointorigin}{\pgfpoint{-0.5 * \width}{ 0.0 * \height}}}
    \anchor{east}{\pgfpointadd{\pgfpointorigin}{\pgfpoint{ 0.5 * \width}{ 0.0 * \height}}}
    \anchor{north east}{\pgfpointadd{\pgfpointorigin}{\pgfpoint{ 0.5 * \width}{ 0.5 * \height}}}
    \anchor{south east}{\pgfpointadd{\pgfpointorigin}{\pgfpoint{ 0.5 * \width}{-0.5 * \height}}}
    \anchor{north west}{\pgfpointadd{\pgfpointorigin}{\pgfpoint{-0.5 * \width}{ 0.5 * \height}}}
    \anchor{south west}{\pgfpointadd{\pgfpointorigin}{\pgfpoint{-0.5 * \width}{-0.5 * \height}}}
    \anchorborder{
        \@tempdima=\pgf@x
        \@tempdimb=\pgf@y
        \pgfpointborderrectangle{\pgfpoint{\@tempdima}{\@tempdimb}}{\pgfpointadd{\pgfpointorigin}{\pgfpoint{ 0.5 * \width}{ 0.5 * \height}}}
    }
    \beforebackgroundpath{
        \pgfsetcornersarced{\pgfpoint{1pt}{1pt}}
        % draw pcb body
        \pgfpathrectanglecorners
            {\pgfpointadd{\pgfpointorigin}{\pgfpoint{-0.5 * \width}{-0.5 * \height}}}
            {\pgfpointadd{\pgfpointorigin}{\pgfpoint{ 0.5 * \width}{ 0.5 * \height}}}
        \pgfusepath{stroke}
        % draw chip
        \pgfsetcornersarced{\pgfpointorigin}
        \pgfpathrectangle
            {\pgfpointadd{\pgfpointorigin}{\pgfpoint{-0.5 * \chipsize}{-0.5 * \chipsize}}}
            {\pgfpointadd{\pgfpointorigin}{\pgfpoint{\chipsize}{\chipsize}}}
        \pgfusepath{fill}
        % draw chip pins
        \foreach \i in {1, 2, ..., \chippins}
        {
            \pgfsetlinewidth{0.3pt}
            \pgfpathmoveto{\pgfpointadd{\pgfpointorigin}{\pgfpoint{(\i - 0.5 * (\chippins + 1)) * \chippinpitch}{0.5 * \chipsize}}}
            \pgfpathlineto{\pgfpointadd{\pgfpointorigin}{\pgfpoint{(\i - 0.5 * (\chippins + 1)) * \chippinpitch}{0.5 * \chipsize + \chippinlength}}}
            \pgfusepath{stroke}
            \pgfpathmoveto{\pgfpointadd{\pgfpointorigin}{\pgfpoint{(\i - 0.5 * (\chippins + 1)) * \chippinpitch}{-0.5 * \chipsize}}}
            \pgfpathlineto{\pgfpointadd{\pgfpointorigin}{\pgfpoint{(\i - 0.5 * (\chippins + 1)) * \chippinpitch}{-0.5 * \chipsize - \chippinlength}}}
            \pgfusepath{stroke}
            \pgfpathmoveto{\pgfpointadd{\pgfpointorigin}{\pgfpoint{0.5 * \chipsize}{(\i - 0.5 * (\chippins + 1)) * \chippinpitch}}}
            \pgfpathlineto{\pgfpointadd{\pgfpointorigin}{\pgfpoint{0.5 * \chipsize + \chippinlength}{(\i - 0.5 * (\chippins + 1)) * \chippinpitch}}}
            \pgfusepath{stroke}
            \pgfpathmoveto{\pgfpointadd{\pgfpointorigin}{\pgfpoint{-0.5 * \chipsize}{(\i - 0.5 * (\chippins + 1)) * \chippinpitch}}}
            \pgfpathlineto{\pgfpointadd{\pgfpointorigin}{\pgfpoint{-0.5 * \chipsize - \chippinlength}{(\i - 0.5 * (\chippins + 1)) * \chippinpitch}}}
            \pgfusepath{stroke}
        }
        % draw mounting holes
        \foreach \x in {-1, 1}
        {
            \foreach \y in {-1, 1}
            {
                \pgfpathcircle{\pgfpointadd{\pgfpointorigin}{\pgfpoint{\x * 0.4 * \width}{\y * 0.4 * \height}}}{0.6pt}
                \pgfusepath{fill}
            }
        }
        \foreach \y in {1, 2, ..., \pins}
        {
            \pgfpathrectangle
                {\pgfpointadd{\pgfpointorigin}{\pgfpoint{-0.5 * \width}{(\y - 0.5 * (\pins + 1)) * \pinpitch - 0.5 * \pinheight}}}
                {\pgfpointadd{\pgfpointorigin}{\pgfpoint{\pinwidth}{\pinheight}}}
            \pgfusepath{fill}
        }
    }
    % \pgf@sh@s@pcbshape contains all the code for the pcb shape
    % and is executed just before a pcb node is drawn.
    \pgfutil@g@addto@macro\pgf@sh@s@pcbshape{%
        % Start with the maximum pin number and go backwards.
        % If the anchor is undefined, create it. Otherwise stop.
        \c@pgf@counta=\pgfkeysvalueof{/tikz/circuits/pcb/out pins}\relax%
        \pgfmathloop%
        \ifnum\c@pgf@counta>0\relax%
            \pgfutil@ifundefined{pgf@anchor@pcbshape@out\space\the\c@pgf@counta}{%
                \expandafter\xdef\csname pgf@anchor@pcbshape@out\space\the\c@pgf@counta\endcsname{%
                    \noexpand\pcbpinanchor{\the\c@pgf@counta}%
                }%
            }{\c@pgf@counta=0\relax}%
            \advance\c@pgf@counta-1\relax%
        \repeatpgfmathloop%  
    }%
}
\def\pcbpinanchor#1{%
    \pgfpointadd{\pgfpointorigin}{\pgfpoint{-0.5 * \width}{(#1 - 0.5 * (\pins + 1)) * \pinpitch}}
}

% vim: ft=plaintex

\tikzset{
    circuits/adc/width/.initial = 0.75cm,
    circuits/adc/height/.initial = 0.75cm,
    circuits/adc/trianglesize/.initial = 3pt,
    circuits/adc/angle/.initial = 45, % currently not used
    adc/.style = {adcshape, draw, wire}
}

\pgfdeclareshape{adcshape}
{
    \saveddimen{\width}{\pgf@x=\pgfkeysvalueof{/tikz/circuits/adc/width}}
    \saveddimen{\height}{\pgf@x=\pgfkeysvalueof{/tikz/circuits/adc/height}}
    \saveddimen{\trianglesize}{\pgf@x=\pgfkeysvalueof{/tikz/circuits/adc/trianglesize}}
    \savedanchor{\centerpoint}{\pgfpointorigin}
    \savedanchor{\outputport}{%
        \pgfpointadd%
        {\pgfpointorigin}%
        {\pgfpoint%
            {0.5 * \pgfkeysvalueof{/tikz/circuits/adc/width} + 0.25 * \pgfkeysvalueof{/tikz/circuits/adc/height}}%
            {0cm}
        }%
    }
    \savedanchor{\inputport}{%
        \pgfpointadd%
        {\pgfpointorigin}%
        {\pgfpoint%
            {-0.5 * \pgfkeysvalueof{/tikz/circuits/adc/width} - 0.25 * \pgfkeysvalueof{/tikz/circuits/adc/height}}%
            {0cm}
        }%
    }
    % electrical terminals (anchors)
    \anchor{in}{\pgfpointadd{\centerpoint}{\pgfpoint{-0.5 * \width - 0.25 * \height}{0cm}}}
    \anchor{out}{\outputport}
    \anchor{clk}{\pgfpointadd{\inputport}{\pgfpoint{0.5 * \height + 0.5 * \width}{-0.5 * \height}}}
    % regular anchors
    \anchor{center}{\centerpoint}
    \anchor{text} % this is used to center the text in the node
    {
        % the text node is shifted more than its width, a way of optical compensation (since the adc gets more narrow to the left)
        \pgfpoint{-0.4\wd\pgfnodeparttextbox}{-0.5\ht\pgfnodeparttextbox}
    }
    \anchor{north}{
        \pgfpointadd{\centerpoint}{\pgfpoint{0cm}{0.5 * \height}}
    }
    \anchor{south}{
        \pgfpointadd{\centerpoint}{\pgfpoint{0cm}{-0.5 * \height}}
    }
    \anchor{east}{
        \pgfpointadd{\centerpoint}{\pgfpoint{0.5 * \width + 0.25 * \height}{0cm}}
    }
    \anchor{west}{
        \pgfpointadd{\centerpoint}{\pgfpoint{-0.5 * \width - 0.25 * \height}{0cm}}
    }
    \anchor{north west}{
        \pgfpointadd{\centerpoint}{\pgfpoint{-0.5 * \width - 0.25 * \height}{0.5 * \height}}
    }
    \anchor{north east}{
        \pgfpointadd{\centerpoint}{\pgfpoint{0.5 * \width + 0.25 * \height}{0.5 * \height}}
    }
    \anchor{south west}{
        \pgfpointadd{\centerpoint}{\pgfpoint{-0.5 * \width - 0.25 * \height}{-0.5 * \height}}
    }
    \anchor{south east}{
        \pgfpointadd{\centerpoint}{\pgfpoint{0.5 * \width - 0.25 * \height}{-0.5 * \height}}
    }
    \backgroundpath{
        \pgfpathmoveto{\inputport}
        \pgfpathlineto{\pgfpointadd{\inputport}{\pgfpoint{0.5 * \height}{0.5 * \height}}}
        \pgfpathlineto{\pgfpointadd{\inputport}{\pgfpoint{0.5 * \height + \width}{0.5 * \height}}}
        \pgfpathlineto{\pgfpointadd{\inputport}{\pgfpoint{0.5 * \height + \width}{-0.5 * \height}}}
        \pgfpathlineto{\pgfpointadd{\inputport}{\pgfpoint{0.5 * \height}{-0.5 * \height}}}
        \pgfpathclose
    }
    \beforebackgroundpath{
        % draw clk triangle
        \pgfsetlinewidth{\pgfkeysvalueof{/tikz/circuits/line width}}
        \pgfpathmoveto{\pgfpointadd{\inputport}{\pgfpoint{0.5 * \height + 0.5 * \width - \trianglesize}{-0.5 * \height}}}
        \pgfpathlineto{\pgfpointadd{\inputport}{\pgfpoint{0.5 * \height + 0.5 * \width}{-0.5 * \height + \trianglesize}}}
        \pgfpathlineto{\pgfpointadd{\inputport}{\pgfpoint{0.5 * \height + 0.5 * \width + \trianglesize}{-0.5 * \height}}}
        \pgfusepath{stroke}
    }
}

% vim: ft=plaintex nowrap

\tikzset{
    circuits/dac/width/.initial = 0.75cm,
    circuits/dac/height/.initial = 0.75cm,
    circuits/dac/angle/.initial = 45, % currently not used
    dac/.style = {draw, wire, dacshape}
}

\pgfdeclareshape{dacshape}
{
    \saveddimen{\width}{\pgf@x=\pgfkeysvalueof{/tikz/circuits/dac/width}}
    \saveddimen{\height}{\pgf@x=\pgfkeysvalueof{/tikz/circuits/dac/height}}
    \savedanchor{\centerpoint}{\pgfpointorigin}
    \savedanchor{\outputport}{%
        \pgfpointadd%
        {\pgfpointorigin}%
        {\pgfpoint%
            {0.5 * \pgfkeysvalueof{/tikz/circuits/dac/width} + 0.25 * \pgfkeysvalueof{/tikz/circuits/dac/height}}%
            {0cm}
        }%
    }
    \savedanchor{\inputport}{%
        \pgfpointadd%
        {\pgfpointorigin}%
        {\pgfpoint%
            {-0.5 * \pgfkeysvalueof{/tikz/circuits/dac/width} - 0.25 * \pgfkeysvalueof{/tikz/circuits/dac/height}}%
            {0cm}
        }%
    }
    % electrical terminals (anchors)
    \anchor{in}{\pgfpointadd{\centerpoint}{\pgfpoint{-0.5 * \width - 0.25 * \height}{0cm}}}
    \anchor{out}{\outputport}
    % regular anchors
    \anchor{center}{\centerpoint}
    \anchor{text} % this is used to center the text in the node
    {
        % the text node is shifted more than its width, a way of optical compensation (since the dac gets more narrow to the right)
        \pgfpoint{-0.6\wd\pgfnodeparttextbox}{-0.5\ht\pgfnodeparttextbox}
    }
    \anchor{north}{
        \pgfpointadd{\centerpoint}{\pgfpoint{0cm}{0.5 * \height}}
    }
    \anchor{south}{
        \pgfpointadd{\centerpoint}{\pgfpoint{0cm}{-0.5 * \height}}
    }
    \anchor{east}{
        \pgfpointadd{\centerpoint}{\pgfpoint{0.5 * \width + 0.25 * \height}{0cm}}
    }
    \anchor{west}{
        \pgfpointadd{\centerpoint}{\pgfpoint{-0.5 * \width - 0.25 * \height}{0cm}}
    }
    \anchor{north west}{
        \pgfpointadd{\centerpoint}{\pgfpoint{-0.5 * \width - 0.25 * \height}{0.5 * \height}}
    }
    \anchor{north east}{
        \pgfpointadd{\centerpoint}{\pgfpoint{0.5 * \width + 0.25 * \height}{0.5 * \height}}
    }
    \anchor{south west}{
        \pgfpointadd{\centerpoint}{\pgfpoint{-0.5 * \width - 0.25 * \height}{-0.5 * \height}}
    }
    \anchor{south east}{
        \pgfpointadd{\centerpoint}{\pgfpoint{0.5 * \width + 0.25 * \height}{-0.5 * \height}}
    }
    \beforebackgroundpath{
        %\pgfsetlinewidth{\pgfkeysvalueof{/tikz/circuits/line width}}
        \pgfpathmoveto{\outputport}
        \pgfpathlineto{\pgfpointadd{\outputport}{\pgfpoint{-0.5 * \height}{0.5 * \height}}}
        \pgfpathlineto{\pgfpointadd{\outputport}{\pgfpoint{-0.5 * \height - \width}{0.5 * \height}}}
        \pgfpathlineto{\pgfpointadd{\outputport}{\pgfpoint{-0.5 * \height - \width}{-0.5 * \height}}}
        \pgfpathlineto{\pgfpointadd{\outputport}{\pgfpoint{-0.5 * \height}{-0.5 * \height}}}
        \pgfpathclose
        %\pgfusepath{stroke}
    }
}

% vim: ft=plaintex nowrap

\tikzset{
    circuits/sumpoint/radius/.initial = 0.25cm,
    sumpoint/.style = {sumpointshape}
}

\pgfdeclareshape{sumpointshape}
{
    \saveddimen{\radius}{\pgf@x=\pgfkeysvalueof{/tikz/circuits/sumpoint/radius}}

    % electrical anchors
    \anchor{left}{\pgfpointadd{\pgfpointorigin}{\pgfpoint{-\radius}{0cm}}}
    \anchor{right}{\pgfpointadd{\pgfpointorigin}{\pgfpoint{\radius}{0cm}}}
    \anchor{top}{\pgfpointadd{\pgfpointorigin}{\pgfpoint{0cm}{ \radius}}}
    \anchor{bottom}{\pgfpointadd{\pgfpointorigin}{\pgfpoint{0cm}{-\radius}}}
    % general anchors
    \anchor{center}{\pgfpointorigin}
    \anchor{west}{\pgfpointadd{\pgfpointorigin}{\pgfpoint{-\radius}{0cm}}}
    \anchor{east}{\pgfpointadd{\pgfpointorigin}{\pgfpoint{\radius}{0cm}}}
    \anchor{north}{\pgfpointadd{\pgfpointorigin}{\pgfpoint{0cm}{ \radius}}}
    \anchor{south}{\pgfpointadd{\pgfpointorigin}{\pgfpoint{0cm}{-\radius}}}
    \anchor{south east}{\pgfpointadd{\pgfpointorigin}{\pgfpoint{ \radius}{-\radius}}}
    \anchor{south west}{\pgfpointadd{\pgfpointorigin}{\pgfpoint{-\radius}{-\radius}}}
    \anchor{north east}{\pgfpointadd{\pgfpointorigin}{\pgfpoint{ \radius}{ \radius}}}
    \anchor{north west}{\pgfpointadd{\pgfpointorigin}{\pgfpoint{-\radius}{ \radius}}}
    \anchorborder{
        \pgf@xa=\pgf@x%
        \pgf@ya=\pgf@y%
        \edef\pgf@marshal{%
            \noexpand\pgfpointborderellipse
            {\noexpand\pgfqpoint{\the\pgf@xa}{\the\pgf@ya}}
            {\noexpand\pgfqpoint{\radius}{\radius}}
        }%
        \pgf@marshal%
        \pgf@xa=\pgf@x%
        \pgf@ya=\pgf@y%
        \pgfpointorigin%
        \advance\pgf@x by\pgf@xa%
        \advance\pgf@y by\pgf@ya%
    }
    \beforebackgroundpath{
        \pgfsetlinewidth{\pgfkeysvalueof{/tikz/circuits/line width}}
        \pgfpathcircle{\pgfpointorigin}{\radius}
        \pgfusepath{stroke}
        % draw plus sign
        \pgfpathmoveto{\pgfpointadd{\pgfpointorigin}{\pgfpoint{-0.5 * \radius}{0cm}}}
        \pgfpathlineto{\pgfpointadd{\pgfpointorigin}{\pgfpoint{ 0.5 * \radius}{0cm}}}
        \pgfpathmoveto{\pgfpointadd{\pgfpointorigin}{\pgfpoint{0cm}{ 0.5 * \radius}}}
        \pgfpathlineto{\pgfpointadd{\pgfpointorigin}{\pgfpoint{0cm}{-0.5 * \radius}}}
        \pgfusepath{stroke}
    }
}

% vim: ft=plaintex

\newif\if@circuits@logicgate@isinverting
\tikzset{
    circuits/logic gate/width/.initial = 0.8cm,
    circuits/logic gate/height/.initial = 0.8cm,
    circuits/logic gate/dot radius/.initial = 3pt,
    circuits/logic gate/input distribution factor/.initial = 0.6,
    circuits/logic gate/output distribution factor/.initial = 0.8,
    circuits/logic gate/is inverting/.is if=@circuits@logicgate@isinverting,
}
\newif\if@circuits@orgate@isnor
\tikzset{
    circuits/orgate/width/.initial = 0.8cm,
    circuits/orgate/height/.initial = 0.8cm,
    circuits/orgate/dotradius/.initial = 2pt,
    circuits/orgate/input distribution factor/.initial = 0.6,
    circuits/orgate/output distribution factor/.initial = 0.8,
    circuits/orgate/is nor/.is if=@circuits@orgate@isnor,
    orgate/.style = {norgateshape, circuits/orgate/is nor = false, circuits/orgate/dotradius = 0pt},
    norgate/.style = {norgateshape, circuits/orgate/is nor = true}
}

\pgfdeclareshape{norgateshape}
{
    \saveddimen{\width}{\pgf@x=\pgfkeysvalueof{/tikz/circuits/orgate/width}}
    \saveddimen{\height}{\pgf@x=\pgfkeysvalueof{/tikz/circuits/orgate/height}}
    \saveddimen{\ordotradius}{\pgf@x=\pgfkeysvalueof{/tikz/circuits/orgate/dotradius}}
    \savedmacro{\inputdistribution}{\renewcommand{\inputdistribution}[0]{\pgfkeysvalueof{/tikz/circuits/orgate/input distribution factor}}}
    \savedanchor{\inputplus}{%
        \pgfpointadd%
        {\pgfpointorigin}%
        {\pgfpoint%
            {-0.5 * \pgfkeysvalueof{/tikz/circuits/orgate/width}}%
            {0.5 * \pgfkeysvalueof{/tikz/circuits/orgate/height} * \pgfkeysvalueof{/tikz/circuits/orgate/input distribution factor}}%
        }%
    }
    \savedanchor{\inputminus}{%
        \pgfpointadd%
        {\pgfpointorigin}%
        {\pgfpoint%
            {-0.5 * \pgfkeysvalueof{/tikz/circuits/orgate/width}}%
            {-0.5 * \pgfkeysvalueof{/tikz/circuits/orgate/height} * \pgfkeysvalueof{/tikz/circuits/orgate/input distribution factor}}%
        }%
    }
    \savedanchor{\output}{%
        \pgfpointadd%
        {\pgfpointorigin}%
        {\pgfpoint%
            {0.5 * \pgfkeysvalueof{/tikz/circuits/orgate/width}}%
            {0pt}
        }%
    }
    % electrical terminals (anchors)
    \anchor{in}{\pgfpointadd{\pgfpointorigin}{\pgfpoint{-\width/2}{0cm}}}
    \anchor{out}{\pgfpointadd{\output}{\pgfpoint{\ordotradius + 2.5 * \pgfkeysvalueof{/tikz/circuits/line width}}{0pt}}\if@circuits@orgate@isnor\pgf@x = \pgf@x + 5pt\fi}
    \anchor{in1}{\inputplus}
    \anchor{in2}{\inputminus}
    % regular anchors
    \anchor{center}{\pgfpointorigin}
    \anchor{north}{\pgfpointadd{\inputplus}{\pgfpoint{\width/2}{0.5 * \height * (1 - \inputdistribution)}}}
    \anchor{south}{\pgfpointadd{\inputminus}{\pgfpoint{\width/2}{-0.5 * \height * (1 - \inputdistribution)}}}
    \anchor{west}{\pgfpoint{-0.5 * \width}{0pt}}
    \anchor{east}{\pgfpoint{0.5 * \width}{0pt}}
    \anchor{north west}{\pgfpointadd{\inputplus}{\pgfpoint{0cm}{0.5 * \height * (1 - \inputdistribution)}}}
    \anchor{south west}{\pgfpointadd{\inputminus}{\pgfpoint{0cm}{-0.5 * \height * (1 - \inputdistribution)}}}
    \anchor{north east}{\pgfpointadd{\inputplus}{\pgfpoint{\width}{0.5 * \height * (1 - \inputdistribution)}}}
    \anchor{south east}{\pgfpointadd{\inputminus}{\pgfpoint{\width}{-0.5 * \height * (1 - \inputdistribution)}}}
    %\beforebackgroundpath{
    %    \pgfsetlinewidth{\pgfkeysvalueof{/tikz/circuits/line width}}
    %    \pgfpathmoveto{\pgfpointadd{\pgfpointorigin}{\pgfpoint{ 0.5 * \width - 0.5 * \height}{ 0.5 * \height}}}
    %    \pgfpathlineto{\pgfpointadd{\pgfpointorigin}{\pgfpoint{-0.5 * \width}{ 0.5 * \height}}}
    %    \pgfpathlineto{\pgfpointadd{\pgfpointorigin}{\pgfpoint{-0.5 * \width}{-0.5 * \height}}}
    %    \pgfpathlineto{\pgfpointadd{\pgfpointorigin}{\pgfpoint{ 0.5 * \width - 0.5 * \height}{-0.5 * \height}}}
    %    \pgfpatharc{-90}{90}{0.5 * \height}
    %    \pgfpathclose
    %    \pgfusepath{stroke}
    %    \if@circuits@orgate@isnor
    %        \pgfpathcircle{\pgfpointadd{\output}{\pgfpoint{0.5 * \ordotradius + 1.5 * \pgfkeysvalueof{/tikz/circuits/line width}}{0pt}}}{\ordotradius}
    %        \pgfusepath{stroke}
    %    \fi
    %}
    \beforebackgroundpath{
        \pgfsetlinewidth{\pgfkeysvalueof{/tikz/circuits/line width}}
        \pgfpathmoveto{\pgfpointadd{\pgfpointorigin}{\pgfpoint{-0.5 * \width}{-0.5 * \height}}}
        \pgfpathquadraticcurveto{\pgfpointadd{\pgfpointorigin}{\pgfpoint{0pt}{0pt}}}{\pgfpointadd{\pgfpointorigin}{\pgfpoint{-0.5 * \width}{ 0.5 * \height}}}
        \pgfpathquadraticcurveto{\pgfpointadd{\output}{\pgfpoint{-0.3cm}{0.5cm}}}{\output}
        \pgfpathquadraticcurveto{\pgfpointadd{\output}{\pgfpoint{-0.3cm}{-0.5cm}}}{\pgfpointadd{\pgfpointorigin}{\pgfpoint{-0.5 * \width}{-0.5 * \height}}}
        \pgfpathclose
        \pgfusepath{stroke}
        \if@circuits@orgate@isnor
            \pgfpathcircle{\pgfpointadd{\output}{\pgfpoint{0.5 * \ordotradius + 1.5 * \pgfkeysvalueof{/tikz/circuits/line width}}{0pt}}}{\ordotradius}
            \pgfusepath{stroke}
        \fi
    }
}

% vim: ft=plaintex nowrap

\tikzset{
    circuits/xor/scale/.initial = 1,
    circuits/xor/width/.initial = 0.8cm,
    circuits/xor/input height/.initial = 0.8cm,
    circuits/xor/output height/.initial = 0.32cm,
    circuits/xor/xorlineoffset/.initial = 3pt,
    circuits/xor/input distribution factor/.initial = 0.6,
    circuits/xor/output distribution factor/.initial = 0.8,
    xor/.style = {xorshape}
}

\pgfdeclareshape{xorshape}
{
    \saveddimen{\width}{\pgf@x=\pgfkeysvalueof{/tikz/circuits/xor/width}}
    \saveddimen{\inputheight}{\pgf@x=\pgfkeysvalueof{/tikz/circuits/xor/input height}}
    \saveddimen{\outputheight}{\pgf@x=\pgfkeysvalueof{/tikz/circuits/xor/output height}}
    \saveddimen{\xorlineoffset}{\pgf@x=\pgfkeysvalueof{/tikz/circuits/xor/xorlineoffset}}
    \savedmacro{\inputdistribution}{\renewcommand{\inputdistribution}[0]{\pgfkeysvalueof{/tikz/circuits/xor/input distribution factor}}}
    \savedmacro{\outputdistribution}{\renewcommand{\outputdistribution}[0]{\pgfkeysvalueof{/tikz/circuits/xor/output distribution factor}}}
    \savedmacro{\scale}{\renewcommand{\scale}[0]{\pgfkeysvalueof{/tikz/circuits/xor/scale}}}
    \savedanchor{\centerpoint}{\pgfpointorigin}
    \savedanchor{\inputanchor}{%
        \pgfpointadd%
        {\pgfpointorigin}%
        {\pgfpoint%
            {-0.5 * \pgfkeysvalueof{/tikz/circuits/xor/width}}%
            {0pt}%
        }%
    }
    \savedanchor{\inputplus}{%
        \pgfpointadd%
        {\pgfpointorigin}%
        {\pgfpoint%
            {-0.5 * \pgfkeysvalueof{/tikz/circuits/xor/width}}%
            {0.5 * \pgfkeysvalueof{/tikz/circuits/xor/input height} * \pgfkeysvalueof{/tikz/circuits/xor/input distribution factor}}%
        }%
    }
    \savedanchor{\inputminus}{%
        \pgfpointadd%
        {\pgfpointorigin}%
        {\pgfpoint%
            {-0.5 * \pgfkeysvalueof{/tikz/circuits/xor/width}}%
            {-0.5 * \pgfkeysvalueof{/tikz/circuits/xor/input height} * \pgfkeysvalueof{/tikz/circuits/xor/input distribution factor}}%
        }%
    }
    \savedanchor{\outputminus}{%
        \pgfpointadd%
        {\pgfpointorigin}%
        {\pgfpoint%
            {0.5 * \pgfkeysvalueof{/tikz/circuits/xor/width}}%
            {0.5 * \pgfkeysvalueof{/tikz/circuits/xor/output height}}
        }%
    }
    \savedanchor{\outputplus}{%
        \pgfpointadd%
        {\pgfpointorigin}%
        {\pgfpoint%
            {0.5 * \pgfkeysvalueof{/tikz/circuits/xor/width}}%
            {0.5 * -\pgfkeysvalueof{/tikz/circuits/xor/output height}}
        }%
    }
    \savedanchor{\output}{%
        \pgfpointadd%
        {\pgfpointorigin}%
        {\pgfpoint%
            {0.5 * \pgfkeysvalueof{/tikz/circuits/xor/width}}%
            {0pt}
        }%
    }
    % electrical terminals (anchors)
    \anchor{in}{\pgfpointadd{\centerpoint}{\pgfpoint{-\width/2}{0cm}}}
    \anchor{out}{\pgfpointadd{\centerpoint}{\pgfpoint{\width/2}{0cm}}}
    \anchor{in1}{\inputplus}
    \anchor{in2}{\inputminus}
    \anchor{topcontrol}{
        \pgfpointlineattime%
        {0.5}%
        {\pgfpointadd{\inputplus}{\pgfpoint{0cm}{0.5 * \inputheight * (1 - \inputdistribution)}}}%
        {\outputminus}
    }
    \anchor{botcontrol}{
        \pgfpointlineattime%
        {0.5}%
        {\pgfpointadd{\inputminus}{\pgfpoint{0cm}{-0.5 * \inputheight * (1 - \inputdistribution)}}}%
        {\outputplus}
    }
    % regular anchors
    \anchor{center}{\centerpoint}
    \anchor{north}{
        \pgfpointadd{\inputplus}{\pgfpoint{\width/2}{0.5 * \inputheight * (1 - \inputdistribution)}}
    }
    \anchor{south}{
        \pgfpointadd{\inputminus}{\pgfpoint{\width/2}{-0.5 * \inputheight * (1 - \inputdistribution)}}
    }
    \anchor{west}{
        \pgfpoint{-0.5 * \width}{0pt}
    }
    \anchor{east}{
        \pgfpoint{0.5 * \width}{0pt}
    }
    \anchor{north west}{
        \pgfpointadd{\inputplus}{\pgfpoint{0cm}{0.5 * \inputheight * (1 - \inputdistribution)}}
    }
    \anchor{south west}{
        \pgfpointadd{\inputminus}{\pgfpoint{0cm}{-0.5 * \inputheight * (1 - \inputdistribution)}}
    }
    \anchor{north east}{
        \pgfpointadd{\inputplus}{\pgfpoint{\width}{0.5 * \inputheight * (1 - \inputdistribution)}}
    }
    \anchor{south east}{
        \pgfpointadd{\inputminus}{\pgfpoint{\width}{-0.5 * \inputheight * (1 - \inputdistribution)}}
    }
    \beforebackgroundpath{
        \pgfsetlinewidth{\pgfkeysvalueof{/tikz/circuits/line width}}
        \pgfpathmoveto{\pgfpointadd{\inputplus}{\pgfpoint{0cm}{0.5 * \inputheight * (1 - \inputdistribution)}}}
        \pgfpathquadraticcurveto{\pgfpointadd{\output}{\pgfpoint{-0.3cm}{0.5cm}}}{\output}
        \pgfpathquadraticcurveto{\pgfpointadd{\output}{\pgfpoint{-0.3cm}{-0.5cm}}}{\pgfpointadd{\inputminus}{\pgfpoint{0cm}{-0.5 * \inputheight * (1 - \inputdistribution)}}}
        \pgfpathquadraticcurveto{\pgfpointadd{\inputanchor}{\pgfpoint{0.3cm}{0pt}}}{\pgfpointadd{\inputplus}{\pgfpoint{0cm}{0.5 * \inputheight * (1 - \inputdistribution)}}}
        \pgfusepath{stroke}
        \pgfpathmoveto{\pgfpointadd{\inputminus}{\pgfpoint{-\xorlineoffset}{-0.5 * \inputheight * (1 - \inputdistribution)}}}
        \pgfpathquadraticcurveto{\pgfpointadd{\inputanchor}{\pgfpoint{0.3cm - \xorlineoffset}{0pt}}}{\pgfpointadd{\inputplus}{\pgfpoint{-\xorlineoffset}{0.5 * \inputheight * (1 - \inputdistribution)}}}
        \pgfusepath{stroke}
    }
}

% vim: ft=plaintex nowrap

\tikzset{
    andgate/.style = {nandgateshape, circuits/logic gate/is inverting = false, circuits/logic gate/dot radius = 0pt},
    nandgate/.style = {nandgateshape, circuits/logic gate/is inverting = true}
}

\pgfdeclareshape{nandgateshape}
{
    \saveddimen{\width}{\pgfpointscale{\pgfkeysvalueof{/tikz/circuits/scale}}{\pgfpoint{\pgfkeysvalueof{/tikz/circuits/logic gate/width}}{0pt}}}
    \saveddimen{\height}{\pgfpointscale{\pgfkeysvalueof{/tikz/circuits/scale}}{\pgfpoint{\pgfkeysvalueof{/tikz/circuits/logic gate/height}}{0pt}}}
    \saveddimen{\dotradius}{\pgfpointscale{\pgfkeysvalueof{/tikz/circuits/scale}}{\pgfpoint{\pgfkeysvalueof{/tikz/circuits/logic gate/dot radius}}{0pt}}}
    \savedmacro{\inputdistribution}{\renewcommand{\inputdistribution}[0]{\pgfkeysvalueof{/tikz/circuits/logic gate/input distribution factor}}}
    \savedanchor{\output}{%
        \pgfpointadd%
        {\pgfpointorigin}%
        {\pgfpoint%
            {0.5 * \pgfkeysvalueof{/tikz/circuits/scale} * \pgfkeysvalueof{/tikz/circuits/logic gate/width}}%
            {0pt}
        }%
    }
    % electrical terminals (anchors)
    \anchor{in}{\pgfpoint{-\width/2}{0cm}}
    \anchor{out}{\pgfpoint{0.5 * \width + 2 * \dotradius}{0pt}}
    \anchor{in1}{\pgfpoint{-0.5 * \width}{0.5 * \height * \inputdistribution}}
    \anchor{in2}{\pgfpoint{-0.5 * \width}{-0.5 * \height * \inputdistribution}}
    % regular anchors
    \anchor{center}{\pgfpointorigin}
    \anchor{north}{\pgfpoint{0pt}{0.5 * \height}}
    \anchor{south}{\pgfpoint{0pt}{-0.5 * \height}}
    \anchor{west}{\pgfpoint{-0.5 * \width}{0pt}}
    \anchor{east}{\pgfpoint{0.5 * \width}{0pt}}
    \anchor{north east}{\pgfpoint{0.5 * \width}{0.5 * \height}}
    \anchor{north west}{\pgfpoint{0.5 * \width}{-0.5 * \height}}
    \anchor{south east}{\pgfpoint{-0.5 * \width}{0.5 * \height}}
    \anchor{south west}{\pgfpoint{-0.5 * \width}{-0.5 * \height}}
    \beforebackgroundpath{
        \pgfsetlinewidth{\pgfkeysvalueof{/tikz/circuits/line width}}
        \pgfpathmoveto{\pgfpointadd{\pgfpointorigin}{\pgfpoint{ 0.5 * \width - 0.5 * \height}{ 0.5 * \height}}}
        \pgfpathlineto{\pgfpointadd{\pgfpointorigin}{\pgfpoint{-0.5 * \width}{ 0.5 * \height}}}
        \pgfpathlineto{\pgfpointadd{\pgfpointorigin}{\pgfpoint{-0.5 * \width}{-0.5 * \height}}}
        \pgfpathlineto{\pgfpointadd{\pgfpointorigin}{\pgfpoint{ 0.5 * \width - 0.5 * \height}{-0.5 * \height}}}
        \pgfpatharc{-90}{90}{0.5 * \height}
        \pgfpathclose
        \pgfusepath{stroke}
        \if@circuits@logicgate@isinverting
            \pgfpathcircle{\pgfpointadd{\output}{\pgfpoint{0.5 * \dotradius + 1.5 * \pgfkeysvalueof{/tikz/circuits/line width}}{0pt}}}{\dotradius}
            \pgfusepath{stroke}
        \fi
    }
}

% vim: ft=plaintex nowrap

\tikzset{
    notgate/.style = {notgateshape}
}

\pgfdeclareshape{notgateshape}
{
    \saveddimen{\width}{\pgfpointscale{\pgfkeysvalueof{/tikz/circuits/scale}}{\pgfpoint{\pgfkeysvalueof{/tikz/circuits/logic gate/width}}{0pt}}}
    \saveddimen{\height}{\pgfpointscale{\pgfkeysvalueof{/tikz/circuits/scale}}{\pgfpoint{\pgfkeysvalueof{/tikz/circuits/logic gate/height}}{0pt}}}
    \saveddimen{\dotradius}{\pgfpointscale{\pgfkeysvalueof{/tikz/circuits/scale}}{\pgfpoint{\pgfkeysvalueof{/tikz/circuits/logic gate/dot radius}}{0pt}}}
    % electrical terminals (anchors)
    \anchor{in}{\pgfpoint{-0.5 * \width}{0pt}}
    \anchor{out}{\pgfpoint{0.5 * \width + 2 * \dotradius}{0pt}}
    \anchor{vdd}{\pgfpoint{0pt}{0.25 * \height}}
    \anchor{vss}{\pgfpoint{0pt}{-0.25 * \height}}
    % regular anchors
    \anchor{center}{\pgfpointorigin}
    \anchor{north}{\pgfpoint{0pt}{0.5 * \height}}
    \anchor{south}{\pgfpoint{0pt}{-0.5 * \height}}
    \anchor{west}{\pgfpoint{-0.5 * \width}{0pt}}
    \anchor{east}{\pgfpoint{0.5 * \width}{0pt}}
    \anchor{north east}{\pgfpoint{0.5 * \width}{0.5 * \height}}
    \anchor{north west}{\pgfpoint{0.5 * \width}{-0.5 * \height}}
    \anchor{south east}{\pgfpoint{-0.5 * \width}{0.5 * \height}}
    \anchor{south west}{\pgfpoint{-0.5 * \width}{-0.5 * \height}}
    \anchor{center}{\pgfpointorigin}
    \beforebackgroundpath{
        \pgfsetlinewidth{\pgfkeysvalueof{/tikz/circuits/line width}}
        \pgfsetmiterjoin
        \pgfsetmiterlimit{2}
        \pgfpathmoveto{\pgfpoint{-0.5 * \width}{-0.5 * \height}}
        \pgfpathlineto{\pgfpoint{-0.5 * \width}{ 0.5 * \height}}
        \pgfpathlineto{\pgfpoint{ 0.5 * \width}{0pt}}
        \pgfpathclose
        \color{white}
        \pgfsetstrokecolor{black}
        \pgfusepath{stroke, fill}
        % draw output circle
        \pgfpathcircle{\pgfpoint{0.5 * \width + \dotradius}{0pt}}{\dotradius}
        \color{white}
        \pgfsetstrokecolor{black}
        \pgfusepath{stroke, fill}
    }
}

% vim: ft=plaintex nowrap


% vim: ft=plaintex nowrap

\newif\if@circuits@sequentiallogic@drawreset
\tikzset{
    circuits/sequential logic/draw reset/.is if=@circuits@sequentiallogic@drawreset,
}
\newif\if@circuits@dff@negedge
\newif\if@circuits@dff@drawNQ
\tikzset{
    circuits/dflipflop/width/.initial = 1.0cm,
    circuits/dflipflop/height/.initial = 1.6cm,
    circuits/dflipflop/trianglesize/.initial = 3pt,
    circuits/dflipflop/dotradius/.initial = 2pt,
    circuits/dflipflop/input distribution factor/.initial = 0.4,
    circuits/dflipflop/draw NQ/.is if=@circuits@dff@drawNQ,
    circuits/dflipflop/negative clock edge/.is if=@circuits@dff@negedge,
    dflipflop/.style = {dflipflopshape}
}

\pgfdeclareshape{dflipflopshape}
{
    \saveddimen{\width}{\pgf@x=\pgfkeysvalueof{/tikz/circuits/dflipflop/width}}
    \saveddimen{\height}{\pgf@x=\pgfkeysvalueof{/tikz/circuits/dflipflop/height}}
    \saveddimen{\dotradius}{\pgf@x=\pgfkeysvalueof{/tikz/circuits/dflipflop/dotradius}}
    \saveddimen{\trianglesize}{\pgf@x=\pgfkeysvalueof{/tikz/circuits/dflipflop/trianglesize}}
    \savedmacro{\inputdistribution}{\renewcommand{\inputdistribution}[0]{\pgfkeysvalueof{/tikz/circuits/dflipflop/input distribution factor}}}
    \savedanchor{\CLK}{%
        \if@circuits@dff@negedge
            \pgfpoint{-0.5 * \pgfkeysvalueof{/tikz/circuits/dflipflop/width} - 2 * \pgfkeysvalueof{/tikz/circuits/dflipflop/dotradius}}{-0.5 * \pgfkeysvalueof{/tikz/circuits/dflipflop/input distribution factor} * \pgfkeysvalueof{/tikz/circuits/dflipflop/height}}
        \else
            \pgfpoint{-0.5 * \pgfkeysvalueof{/tikz/circuits/dflipflop/width}}{-0.5 * \pgfkeysvalueof{/tikz/circuits/dflipflop/input distribution factor} * \pgfkeysvalueof{/tikz/circuits/dflipflop/height}}
        \fi
    }
    % electrical terminals (anchors)
    \anchor{D}{\pgfpoint{-0.5 * \width}{0.5 * \inputdistribution * \height}}
    \deferredanchor{CLK}{\CLK}
    \anchor{Q}{\pgfpoint{0.5 * \width}{0.5 * \inputdistribution * \height}}
    \anchor{NQ}{\pgfpoint{0.5 * \width + 2 * \dotradius}{-0.5 * \inputdistribution * \height}}
    % regular anchors
    \anchor{text}
    {
        \pgfpoint{-0.5\wd\pgfnodeparttextbox}{-0.5\ht\pgfnodeparttextbox}
    }
    \anchor{center}{\pgfpointorigin}
    \anchor{north}{\pgfpointadd{\pgfpointorigin}{\pgfpoint{0.0 * \width}{ 0.5 * \height}}}
    \anchor{south}{\pgfpointadd{\pgfpointorigin}{\pgfpoint{0.0 * \width}{-0.5 * \height}}}
    \anchor{west}{\pgfpointadd{\pgfpointorigin}{\pgfpoint{-0.5 * \width}{ 0.0 * \height}}}
    \anchor{east}{\pgfpointadd{\pgfpointorigin}{\pgfpoint{ 0.5 * \width}{ 0.0 * \height}}}
    \anchor{north east}{\pgfpointadd{\pgfpointorigin}{\pgfpoint{ 0.5 * \width}{ 0.5 * \height}}}
    \anchor{south east}{\pgfpointadd{\pgfpointorigin}{\pgfpoint{ 0.5 * \width}{-0.5 * \height}}}
    \anchor{north west}{\pgfpointadd{\pgfpointorigin}{\pgfpoint{-0.5 * \width}{ 0.5 * \height}}}
    \anchor{south west}{\pgfpointadd{\pgfpointorigin}{\pgfpoint{-0.5 * \width}{-0.5 * \height}}}
    \anchorborder{
        \@tempdima=\pgf@x
        \@tempdimb=\pgf@y
        \pgfpointborderrectangle{\pgfpoint{\@tempdima}{\@tempdimb}}{\pgfpointadd{\pgfpointorigin}{\pgfpoint{ 0.5 * \width}{ 0.5 * \height}}}
    }
    \beforebackgroundpath{
        \pgfsetlinewidth{\pgfkeysvalueof{/tikz/circuits/line width}}
        \pgfpathrectangle %
            {\pgfpointadd{\pgfpointorigin}{\pgfpoint{-0.5 * \width}{-0.5 * \height}}}
            {\pgfpointadd{\pgfpointorigin}{\pgfpoint{\width}{\height}}}
        \pgfusepath{stroke}
        % draw clk triangle
        \pgfpathmoveto{\pgfpointadd{\pgfpoint{-0.5 * \width}{-0.5 * \inputdistribution * \height}}{\pgfpoint{0pt}{\trianglesize}}}
        \pgfpathlineto{\pgfpointadd{\pgfpoint{-0.5 * \width}{-0.5 * \inputdistribution * \height}}{\pgfpoint{\trianglesize}{0pt}}}
        \pgfpathlineto{\pgfpointadd{\pgfpoint{-0.5 * \width}{-0.5 * \inputdistribution * \height}}{\pgfpoint{0pt}{-\trianglesize}}}
        \pgfusepath{stroke}
        % draw clk circle
        \if@circuits@dff@negedge
            \pgfpathcircle{\pgfpoint{-0.5 * \width - \dotradius}{-0.5 * \inputdistribution * \height}}{\dotradius}
            \pgfusepath{stroke}
        \fi
        % draw NQ circle
        \if@circuits@dff@drawNQ
            \pgfpathcircle{\pgfpoint{0.5 * \width + \dotradius}{-0.5 * \inputdistribution * \height}}{\dotradius}
            \pgfusepath{stroke}
        \fi
    }
}

% vim: ft=plaintex nowrap

\tikzset{
    circuits/counter/width/.initial = 1.8cm,
    circuits/counter/height/.initial = 1.0cm,
    circuits/counter/trianglesize/.initial = 3pt,
    counter/.style = {countershape}
}

\pgfdeclareshape{countershape}
{
    \saveddimen{\width}{\pgf@x=\pgfkeysvalueof{/tikz/circuits/counter/width}}
    \saveddimen{\height}{\pgf@x=\pgfkeysvalueof{/tikz/circuits/counter/height}}
    \saveddimen{\dotradius}{\pgf@x=2pt}
    \saveddimen{\trianglesize}{\pgf@x=\pgfkeysvalueof{/tikz/circuits/counter/trianglesize}}
    \savedanchor{\count}{%
        \pgfpointadd%
        {\pgfpointorigin}%
        {\pgfpoint%
            {-0.5 * \pgfkeysvalueof{/tikz/circuits/counter/width}}%
            {0.15 * \pgfkeysvalueof{/tikz/circuits/counter/height}}
        }%
    }
    \savedanchor{\reset}{%
        \pgfpointadd%
        {\pgfpointorigin}%
        {\pgfpoint%
            {-0.5 * \pgfkeysvalueof{/tikz/circuits/counter/width}}%
            {-0.15 * \pgfkeysvalueof{/tikz/circuits/counter/height}}
        }%
    }
    \savedanchor{\Q}{%
        \pgfpointadd%
        {\pgfpointorigin}%
        {\pgfpoint%
            {0.5 * \pgfkeysvalueof{/tikz/circuits/counter/width}}%
            {0.15 * \pgfkeysvalueof{/tikz/circuits/counter/height}}
        }%
    }
    \savedanchor{\NQ}{%
        \pgfpointadd%
        {\pgfpointorigin}%
        {\pgfpoint%
            {0.5 * \pgfkeysvalueof{/tikz/circuits/counter/width}}%
            {-0.15 * \pgfkeysvalueof{/tikz/circuits/counter/height}}
        }%
    }
    % electrical terminals (anchors)
    \anchor{reset}{\pgfpoint{-0.5 * \width - 2 * \dotradius}{-0.15 * \height}}
    \anchor{count}{\count}
    \anchor{Q}{\Q}
    \anchor{NQ}{\pgfpointadd{\NQ}{\pgfpoint{2 * \dotradius}{0pt}}}
    % regular anchors
    \anchor{text}
    {
        \pgfpoint{-0.5\wd\pgfnodeparttextbox}{-0.5\ht\pgfnodeparttextbox}
    }
    \anchor{center}{\pgfpointorigin}
    \anchor{north}{\pgfpointadd{\pgfpointorigin}{\pgfpoint{0.0 * \width}{ 0.5 * \height}}}
    \anchor{south}{\pgfpointadd{\pgfpointorigin}{\pgfpoint{0.0 * \width}{-0.5 * \height}}}
    \anchor{west}{\pgfpointadd{\pgfpointorigin}{\pgfpoint{-0.5 * \width}{ 0.0 * \height}}}
    \anchor{east}{\pgfpointadd{\pgfpointorigin}{\pgfpoint{ 0.5 * \width}{ 0.0 * \height}}}
    \anchor{north east}{\pgfpointadd{\pgfpointorigin}{\pgfpoint{ 0.5 * \width}{ 0.5 * \height}}}
    \anchor{south east}{\pgfpointadd{\pgfpointorigin}{\pgfpoint{ 0.5 * \width}{-0.5 * \height}}}
    \anchor{north west}{\pgfpointadd{\pgfpointorigin}{\pgfpoint{-0.5 * \width}{ 0.5 * \height}}}
    \anchor{south west}{\pgfpointadd{\pgfpointorigin}{\pgfpoint{-0.5 * \width}{-0.5 * \height}}}
    \anchorborder{
        \@tempdima=\pgf@x
        \@tempdimb=\pgf@y
        \pgfpointborderrectangle{\pgfpoint{\@tempdima}{\@tempdimb}}{\pgfpointadd{\pgfpointorigin}{\pgfpoint{ 0.5 * \width}{ 0.5 * \height}}}
    }
    \beforebackgroundpath{
        \pgfsetlinewidth{\pgfkeysvalueof{/tikz/circuits/line width}}
        \pgfpathrectangle %
            {\pgfpointadd{\pgfpointorigin}{\pgfpoint{-0.5 * \width}{-0.5 * \height}}}
            {\pgfpointadd{\pgfpointorigin}{\pgfpoint{\width}{\height}}}
        \pgfusepath{stroke}
        % draw clk triangle
        \pgfpathmoveto{\pgfpointadd{\count}{\pgfpoint{0pt}{\trianglesize}}}
        \pgfpathlineto{\pgfpointadd{\count}{\pgfpoint{\trianglesize}{0pt}}}
        \pgfpathlineto{\pgfpointadd{\count}{\pgfpoint{0pt}{-\trianglesize}}}
        \pgfusepath{stroke}
        % draw reset circle
        \if@circuits@sequentiallogic@drawreset
            \pgfpathcircle{\pgfpoint{-0.5 * \width - \dotradius}{-0.15 * \height}}{\dotradius}
            \pgfusepath{stroke}
        \fi
    }
}

% vim: ft=plaintex nowrap

\tikzset{
    circuits/register/width/.initial = 1.3cm,
    circuits/register/height/.initial = 0.8cm,
    circuits/register/trianglesize/.initial = 3pt,
    register/.style = {registershape}
}

\pgfdeclareshape{registershape}
{
    \saveddimen{\width}{\pgf@x=\pgfkeysvalueof{/tikz/circuits/register/width}}
    \saveddimen{\height}{\pgf@x=\pgfkeysvalueof{/tikz/circuits/register/height}}
    \saveddimen{\dotradius}{\pgf@x=2pt}
    \saveddimen{\trianglesize}{\pgf@x=\pgfkeysvalueof{/tikz/circuits/register/trianglesize}}
    \savedanchor{\din}{%
        \pgfpointadd%
        {\pgfpointorigin}%
        {\pgfpoint%
            {-0.5 * \pgfkeysvalueof{/tikz/circuits/register/width}}%
            {0.30 * \pgfkeysvalueof{/tikz/circuits/register/height}}
        }%
    }
    \savedanchor{\clk}{%
        \pgfpointadd%
        {\pgfpointorigin}%
        {\pgfpoint%
            {-0.5 * \pgfkeysvalueof{/tikz/circuits/register/width}}%
            {0.00 * \pgfkeysvalueof{/tikz/circuits/register/height}}
        }%
    }
    \savedanchor{\Q}{%
        \pgfpointadd%
        {\pgfpointorigin}%
        {\pgfpoint%
            {0.5 * \pgfkeysvalueof{/tikz/circuits/register/width}}%
            {0.15 * \pgfkeysvalueof{/tikz/circuits/register/height}}
        }%
    }
    \savedanchor{\NQ}{%
        \pgfpointadd%
        {\pgfpointorigin}%
        {\pgfpoint%
            {0.5 * \pgfkeysvalueof{/tikz/circuits/register/width}}%
            {-0.15 * \pgfkeysvalueof{/tikz/circuits/register/height}}
        }%
    }
    % electrical terminals (anchors)
    \anchor{reset}{\pgfpoint{-0.5 * \width - 2 * \dotradius}{-0.30 * \height}}
    \anchor{DIN}{\din}
    \anchor{CLK}{\clk}
    \anchor{Q}{\Q}
    \anchor{NQ}{\pgfpointadd{\NQ}{\pgfpoint{2 * \dotradius}{0pt}}}
    % regular anchors
    \anchor{text}
    {
        \pgfpoint{-0.5\wd\pgfnodeparttextbox}{-0.5\ht\pgfnodeparttextbox}
    }
    \anchor{center}{\pgfpointorigin}
    \anchor{north}{\pgfpointadd{\pgfpointorigin}{\pgfpoint{0.0 * \width}{ 0.5 * \height}}}
    \anchor{south}{\pgfpointadd{\pgfpointorigin}{\pgfpoint{0.0 * \width}{-0.5 * \height}}}
    \anchor{west}{\pgfpointadd{\pgfpointorigin}{\pgfpoint{-0.5 * \width}{ 0.0 * \height}}}
    \anchor{east}{\pgfpointadd{\pgfpointorigin}{\pgfpoint{ 0.5 * \width}{ 0.0 * \height}}}
    \anchor{north east}{\pgfpointadd{\pgfpointorigin}{\pgfpoint{ 0.5 * \width}{ 0.5 * \height}}}
    \anchor{south east}{\pgfpointadd{\pgfpointorigin}{\pgfpoint{ 0.5 * \width}{-0.5 * \height}}}
    \anchor{north west}{\pgfpointadd{\pgfpointorigin}{\pgfpoint{-0.5 * \width}{ 0.5 * \height}}}
    \anchor{south west}{\pgfpointadd{\pgfpointorigin}{\pgfpoint{-0.5 * \width}{-0.5 * \height}}}
    \anchorborder{
        \@tempdima=\pgf@x
        \@tempdimb=\pgf@y
        \pgfpointborderrectangle{\pgfpoint{\@tempdima}{\@tempdimb}}{\pgfpointadd{\pgfpointorigin}{\pgfpoint{ 0.5 * \width}{ 0.5 * \height}}}
    }
    \beforebackgroundpath{
        \pgfsetlinewidth{\pgfkeysvalueof{/tikz/circuits/line width}}
        \pgfpathrectangle %
            {\pgfpointadd{\pgfpointorigin}{\pgfpoint{-0.5 * \width}{-0.5 * \height}}}
            {\pgfpointadd{\pgfpointorigin}{\pgfpoint{\width}{\height}}}
        \pgfusepath{stroke}
        % draw clk triangle
        \pgfpathmoveto{\pgfpointadd{\clk}{\pgfpoint{0pt}{\trianglesize}}}
        \pgfpathlineto{\pgfpointadd{\clk}{\pgfpoint{\trianglesize}{0pt}}}
        \pgfpathlineto{\pgfpointadd{\clk}{\pgfpoint{0pt}{-\trianglesize}}}
        \pgfusepath{stroke}
        % draw reset circle
        \if@circuits@sequentiallogic@drawreset
            \pgfpathcircle{\pgfpoint{-0.5 * \width - \dotradius}{-0.30 * \height}}{\dotradius}
            \pgfusepath{stroke}
        \fi
    }
}

% vim: ft=plaintex nowrap

\tikzset{
    circuits/statemachine/width/.initial = 1.3cm,
    circuits/statemachine/trianglesize/.initial = 3pt,
    circuits/statemachine/ports/.initial = 1,
    circuits/statemachine/port pitch/.initial = 0.30cm,
    statemachine/.style = {statemachineshape}
}

\pgfdeclareshape{statemachineshape}
{
    \saveddimen{\width}{\pgf@x=\pgfkeysvalueof{/tikz/circuits/statemachine/width}}
    \saveddimen{\trianglesize}{\pgf@x=\pgfkeysvalueof{/tikz/circuits/statemachine/trianglesize}}
    \saveddimen\height{
        \pgfmathsetlength\pgf@x{(\pgfkeysvalueof{/tikz/circuits/statemachine/ports}) * \pgfkeysvalueof{/tikz/circuits/statemachine/port pitch}}%
    }
    \savedmacro{\ports}{\renewcommand{\ports}[0]{\pgfkeysvalueof{/tikz/circuits/statemachine/ports}}}
    \saveddimen{\portpitch}{\pgf@x=\pgfkeysvalueof{/tikz/circuits/statemachine/port pitch}}
    \saveddimen{\dotradius}{\pgf@x=2pt}
    % electrical anchors
    \anchor{clk}{\pgfpoint{-0.5 * \width}{0pt + 1 * \portpitch}}
    \anchor{reset}{\pgfpoint{-0.5 * \width - 2 * \dotradius}{0pt + 0 * \portpitch}}
    % regular anchors
    \anchor{text}
    {
        \pgfpoint{-0.5\wd\pgfnodeparttextbox}{-0.5\ht\pgfnodeparttextbox}
    }
    \anchor{center}{\pgfpointorigin}
    \anchor{north}{\pgfpointadd{\pgfpointorigin}{\pgfpoint{0.0 * \width}{ 0.5 * \height}}}
    \anchor{south}{\pgfpointadd{\pgfpointorigin}{\pgfpoint{0.0 * \width}{-0.5 * \height}}}
    \anchor{west}{\pgfpointadd{\pgfpointorigin}{\pgfpoint{-0.5 * \width}{ 0.0 * \height}}}
    \anchor{east}{\pgfpointadd{\pgfpointorigin}{\pgfpoint{ 0.5 * \width}{ 0.0 * \height}}}
    \anchor{north east}{\pgfpointadd{\pgfpointorigin}{\pgfpoint{ 0.5 * \width}{ 0.5 * \height}}}
    \anchor{south east}{\pgfpointadd{\pgfpointorigin}{\pgfpoint{ 0.5 * \width}{-0.5 * \height}}}
    \anchor{north west}{\pgfpointadd{\pgfpointorigin}{\pgfpoint{-0.5 * \width}{ 0.5 * \height}}}
    \anchor{south west}{\pgfpointadd{\pgfpointorigin}{\pgfpoint{-0.5 * \width}{-0.5 * \height}}}
    \beforebackgroundpath{%
        % draw body
        \pgfsetlinewidth{\pgfkeysvalueof{/tikz/circuits/line width}}
        \pgfpathrectangle %
            {\pgfpointadd{\pgfpointorigin}{\pgfpoint{-0.5 * \width}{-0.5 * \height}}}
            {\pgfpointadd{\pgfpointorigin}{\pgfpoint{\width}{\height}}}
        \pgfusepath{stroke}
        % draw clk triangle
        \pgfpathmoveto{%
            \pgfpointadd%
                {\pgfpoint{-0.5 * \width}{0pt + 1 * \portpitch}}%
                {\pgfpoint{0pt}{\trianglesize}}%
        }
        \pgfpathlineto{%
            \pgfpointadd%
                {\pgfpoint{-0.5 * \width}{0pt + 1 * \portpitch}}%
                {\pgfpoint{\trianglesize}{0pt}}%
        }
        \pgfpathlineto{%
            \pgfpointadd%
                {\pgfpoint{-0.5 * \width}{0pt + 1 * \portpitch}}%
                {\pgfpoint{0pt}{-\trianglesize}}%
        }
        \pgfusepath{stroke}
        % draw reset circle
        \pgfpathcircle{\pgfpoint{-0.5 * \width - \dotradius}{0pt}}{\dotradius}
        \pgfusepath{stroke}
    }
    % \pgf@sh@s@statemachine contains all the code for the statemachine shape
    % and is executed just before a statemachine node is drawn.
    \pgfutil@g@addto@macro\pgf@sh@s@statemachineshape{%
        % Start with the maximum pin number and go backwards.
        % If the anchor is undefined, create it. Otherwise stop.
        \c@pgf@counta=\pgfkeysvalueof{/tikz/circuits/statemachine/ports}\relax%
        \pgfmathloop%
        \ifnum\c@pgf@counta>0\relax%
            \pgfutil@ifundefined{pgf@anchor@statemachineshape@port\space\the\c@pgf@counta}{%
                \expandafter\xdef\csname pgf@anchor@statemachineshape@port\space\the\c@pgf@counta\endcsname{%
                    \noexpand\statemachineportanchor{\the\c@pgf@counta}%
                }%
            }{\c@pgf@counta=0\relax}%
            \advance\c@pgf@counta-1\relax%
        \repeatpgfmathloop%  
    }%
}

\def\statemachineportanchor#1{%
    % When this macro is called,
    % \width, \ports and \portpitch will be defined
    \pgfpointadd{\pgfpointorigin}{\pgfpoint{-0.5 * \width}{(#1 - 0.5 * (\ports + 1) - 1) * \portpitch}}
}

% vim: ft=plaintex


% vim: ft=plaintex nowrap

\tikzset{
    circuits/mux/width/.initial = 1cm,
    circuits/mux/input height/.initial = 1cm,
    circuits/mux/output height/.initial = 0.6cm,
    circuits/mux/input distribution factor/.initial = 0.6,
    circuits/mux/output distribution factor/.initial = 0.8,
    circuits/mux/no pin labels/.is if=muxnopinlabels,
    mux/.style = {muxshape}
}

\pgfdeclareshape{muxshape}
{
    \saveddimen{\width}{\pgf@x=\pgfkeysvalueof{/tikz/circuits/mux/width}}
    \saveddimen{\inputheight}{\pgf@x=\pgfkeysvalueof{/tikz/circuits/mux/input height}}
    \saveddimen{\outputheight}{\pgf@x=\pgfkeysvalueof{/tikz/circuits/mux/output height}}
    \savedmacro{\inputdistribution}{\renewcommand{\inputdistribution}[0]{\pgfkeysvalueof{/tikz/circuits/mux/input distribution factor}}}
    \savedmacro{\outputdistribution}{\renewcommand{\outputdistribution}[0]{\pgfkeysvalueof{/tikz/circuits/mux/output distribution factor}}}
    \savedanchor{\centerpoint}{\pgfpointorigin}
    \savedanchor{\inputplus}{%
        \pgfpointadd%
        {\pgfpointorigin}%
        {\pgfpoint%
            {-0.5 * \pgfkeysvalueof{/tikz/circuits/mux/width}}%
            {0.5 * \pgfkeysvalueof{/tikz/circuits/mux/input height} * \pgfkeysvalueof{/tikz/circuits/mux/input distribution factor}}%
        }%
    }
    \savedanchor{\inputminus}{%
        \pgfpointadd%
        {\pgfpointorigin}%
        {\pgfpoint%
            {-0.5 * \pgfkeysvalueof{/tikz/circuits/mux/width}}%
            {-0.5 * \pgfkeysvalueof{/tikz/circuits/mux/input height} * \pgfkeysvalueof{/tikz/circuits/mux/input distribution factor}}%
        }%
    }
    \savedanchor{\outputminus}{%
        \pgfpointadd%
        {\pgfpointorigin}%
        {\pgfpoint%
            {0.5 * \pgfkeysvalueof{/tikz/circuits/mux/width}}%
            {0.5 * \pgfkeysvalueof{/tikz/circuits/mux/output height}}
        }%
    }
    \savedanchor{\outputplus}{%
        \pgfpointadd%
        {\pgfpointorigin}%
        {\pgfpoint%
            {0.5 * \pgfkeysvalueof{/tikz/circuits/mux/width}}%
            {0.5 * -\pgfkeysvalueof{/tikz/circuits/mux/output height}}
        }%
    }
    % electrical terminals (anchors)
    \anchor{in}{\pgfpointadd{\centerpoint}{\pgfpoint{-\width/2}{0cm}}}
    \anchor{out}{\pgfpointadd{\centerpoint}{\pgfpoint{\width/2}{0cm}}}
    \anchor{in1}{\inputplus}
    \anchor{in2}{\inputminus}
    \anchor{topcontrol}{
        \pgfpointlineattime%
        {0.5}%
        {\pgfpointadd{\inputplus}{\pgfpoint{0cm}{0.5 * \inputheight * (1 - \inputdistribution)}}}%
        {\outputminus}
    }
    \anchor{botcontrol}{
        \pgfpointlineattime%
        {0.5}%
        {\pgfpointadd{\inputminus}{\pgfpoint{0cm}{-0.5 * \inputheight * (1 - \inputdistribution)}}}%
        {\outputplus}
    }
    % regular anchors
    \anchor{center}{\centerpoint}
    \anchor{north}{
        \pgfpointadd{\inputplus}{\pgfpoint{\width/2}{0.5 * \inputheight * (1 - \inputdistribution)}}
    }
    \anchor{south}{
        \pgfpointadd{\inputminus}{\pgfpoint{\width/2}{-0.5 * \inputheight * (1 - \inputdistribution)}}
    }
    \anchor{west}{
        \pgfpoint{-0.5 * \width}{0cm}
    }
    \anchor{east}{
        \pgfpoint{0.5 * \width}{0cm}
    }
    \anchor{north west}{
        \pgfpointadd{\inputplus}{\pgfpoint{0cm}{0.5 * \inputheight * (1 - \inputdistribution)}}
    }
    \anchor{south west}{
        \pgfpointadd{\inputminus}{\pgfpoint{0cm}{-0.5 * \inputheight * (1 - \inputdistribution)}}
    }
    \anchor{north east}{
        \pgfpointadd{\inputplus}{\pgfpoint{\width}{0.5 * \inputheight * (1 - \inputdistribution)}}
    }
    \anchor{south east}{
        \pgfpointadd{\inputminus}{\pgfpoint{\width}{-0.5 * \inputheight * (1 - \inputdistribution)}}
    }
    \beforebackgroundpath{
        \pgfsetlinewidth{\pgfkeysvalueof{/tikz/circuits/line width}}
        \pgfpathmoveto{\pgfpointadd{\inputplus}{\pgfpoint{0cm}{0.5 * \inputheight * (1 - \inputdistribution)}}}
        \pgfpathlineto{\outputminus}
        \pgfpathlineto{\outputplus}
        \pgfpathlineto{\pgfpointadd{\inputminus}{\pgfpoint{0cm}{-0.5 * \inputheight * (1 - \inputdistribution)}}}
        \pgfpathclose
        \pgfusepath{stroke}
    }
}

% vim: ft=plaintex nowrap

\makeatother

\tikzset{
    circuits/mosfet/source length = 0.25cm,
    circuits/mosfet/drain length = 0.25cm,
    circuits/mosfet/channel length = 0.4cm,
    circuits/mosfet/channel height = 0.25cm,
    circuits/mosfet/gate width = 0.4cm,
    circuits/mosfet/arrow length/.initial = 1.25mm,
    circuits/mosfet/arrow width/.initial = 1.25mm,
    circuits/currentsource/radius = 0.2cm,
    circuits/capacitor/height = 0.4cm,
    circuits/resistor/height = 0.2cm,
    circuits/resistor/width = 0.5cm,
    circuits/resistor/segments = 3,
    circuits/inductor/height = 0.2cm,
    circuits/inductor/width = 0.8cm,
    circuits/inductor/segments = 4,
    circuits/switch/length = 0.2cm,
    circuits/switch/extension = 0.1cm,
    circuits/switch/circle radius = 0.03cm,
    circuits/switch/angle = 30,
    circuits/dot/radius = 0.042cm
}

% vim: ft=plaintex nowrap
